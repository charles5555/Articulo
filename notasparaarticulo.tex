\documentclass{article}
\usepackage[utf8]{inputenc}
\usepackage[english]{babel}
\usepackage{xcolor}
\usepackage{calrsfs}
\DeclareMathAlphabet{\pazocal}{OMS}{zplm}{m}{n}
\newcommand{\Sm}{\Ss_M}
\newcommand{\de}{\delta}
\newcommand{\ep}{\epsilon}
\newcommand{\Ss}{\mathcal{S}}


\title{Quantum H theorem, variational entropy, and relaxation processes}
\author{labfqot}
\date{March 2020}

\begin{document}

\maketitle

\section*{Abstract}
Starting from a functional, $\mathcal{H}$, defined in terms of non-homogeneous distribution functions, an alternative demonstration of the classical and quantum $H$ theorem is performed \textit{via} a variational procedure. This functional describes the time evolution of out of equilibrium classical or quantum gases, and it satisfies the condition $d\mathcal{H}/dt\leq0$. The equality condition is attained when the system reaches the thermodynamic equilibrium, and in this state -$\mathcal{H}$ becomes the entropy of an ideal  gas. The properties of the functionals $\mathcal{H}$ are investigated, including its correct behavior according to the correspondence principle. In addition, the relaxation of quantum gases towards equilibrium is explored, from which a time-evolution equation is obtained. Thus, the novelty of the approach presented here, is the application of the variational method to a functional defined in terms of non-homogeneous distribution functions, in the $\mu$-space and position-energy spaces respectively.







\section{Introduction}

In the literature, the theoretical basis that allows us to describe equilibrium systems in the classical scheme is well-established. The procedures to describe equilibrium systems are well-known and work in almost every system in nature (Thermodynamics for a phenomenological description and statistical mechanics for a microscopic description). In contrast, a unified mathematical model to describe the evolution of any system to an equilibrium state has not developed. Nevertheless, developments can be found in the literature and describe the evolution to equilibrium to almost any system (the internal behavior of glasses \cite{cristal1} and the entropy measurement \cite{cristal2, cristal3} are some examples where the recent developments do not work.) Examples of those developments can be the Onsager formulation (called linear thermodynamics) \cite{kei, onsager}, and the Kinetic Theory of dilute gases developed by Boltzmann.\\
In the Kinetic Theory of dilute gases, the behavior of a dilute classical gas was described by Boltzmann through the Boltzmann transport equation. Furthermore, the evolution of an equilibrium state for any system was proved by means the Boltzmann's $H$ theorem (or classical $H$ theorem) and the spatial-homogeneous distribution function hypothesis.
However, there are some aspects in the classical $H$ theorem and the Boltzmann's $H$ functional that deserve some review to accomplish better consistency, for example, the modification of the $H$ theorem in stochastic trajectories, violations in the second law of thermodynamics, and the relation between Shannon's measure of information and the Boltzmann's entropy  [6-10].\\ 
\\
To obtain the well-known correspondence principle between classical mechanics and quantum mechanics, a quantum version of the Boltzmann's $H$ theorem (or quantum $H$ theorem) and the quantum analogous to the Boltzmann transport equation is necessary such that in the appropriate limit, the  classical $H$ theorem and the Boltzmann transport equation should be recovered.\\
One of the first developments in this field was developed by Tolman \cite{tolman}. He proposes one of the first quantum versions of the Boltzmann's $H$ theorem using the probability transition relation, the random phases hypothesis, and a homogeneous distribution function. The equation of motion for the occupation number (necessary to prove the quantum $H$ theorem in the same way that Boltzmann's procedure) was obtained by applying the time perturbation theory. However, to obtain the flux relations in analogy to the Boltzmann's method, a quantum version of the transport equation that includes a non-homogeneous distribution function is required.\\
On the other hand, some works in the quantum operator formalism attempt to describe quantum transport phenomena using the Hamiltonian of the system and the master equation (analogous to the Boltzmann transport equation) [12-15]. 
Besides, some authors construct the $H$ functional and develop the proof of the quantum $H$ theorem \cite{htheorem2, quantum1, quantum2}. However, the homogeneous distribution function hypothesis and the correspondence principle are uncleared or not discussed. Moreover, the validity of the quantum $H$ theorem and the second law of thermodynamics, and the interpretation of the quantum entropy is still in discussion [19-26].\\ 
\\
In order to contribute to the construction of a consistent classical and quantum $H$ theorem, and a new formalism to describe out of equilibrium systems, we develop a new theoretical basis to describe them. In particular, we are interested in including a non-homogeneous distribution function in the mathematical model, and so including non-homogeneous systems in the $H$ theorem. Besides, we are interested in developing the quantum analogous of Boltzmann's $H$ theorem considering a non-spatial-homogeneous distribution function and solve the problem of the correspondence principle mentioned above.\\
\textcolor{red}{The first section of this work consists of a review of the Boltzmann transport equation, the Boltzmann $H$ theorem, the Maxwell-Boltzmann distribution, and our alternative form to prove the classical $H$ theorem. The second part consists of a review of the proposal from Tolman about the quantum version of the $H$ theorem, the Bose-Einstein and Fermi-Dirac distributions, and our alternative form to prove the quantum version of the $H$ theorem. The third section consists of a classical-quantum correspondence between the quantum $H$ functional and the classical $H$ functional. Finally, the last section consists of a proposal to explore relaxation processes in the quantum case obtaining the analogous quantum Boltzmann transport equation.} 





%----------------------------------
\section{Classical scheme}
\subsection{The Boltzmann transport equation}
In the classical version, the Boltzmann transport equation is required to show the $H$-theorem. That equation is
\begin{equation}
    \left( \frac{\partial }{\partial t}+\vec{v}_1 \cdot \nabla_r +\frac{\vec{F}}{m} \cdot \nabla_{v_1} \right)f_1=\int \mathrm{d}\Omega \int \mathrm{d}^{3}v_2\sigma(\Omega)|\vec{v}_1-\vec{v}_2|(f_2'f_1'-f_2f_1), \label{transport}
\end{equation}
where $\Omega$ is the solid angle, $\sigma$ is the scattering section, $F$ the external force applied to the system, $f_1, f_2$ are the distribution function of the particles 1 and 2 respectively before the collision, and $f'_1, f'_2$ are the distribution function of the particles 1 and 2 respectively after the collision.\\
\textcolor{red}{On the one hand, dynamics, inhomogeneities, and external forces are described by the left-hand side of the equation (\ref{transport}). On the other hand, the binary collisions among particles are involved in the right-hand side of the previous equation. However, the \textit{molecular chaos hypothesis} was applied to obtain the right-hand side of (\ref{transport}). The \textit{molecular chaos hypothesis} establishes that spatial and velocity coordinates of the particles are not correlated.}
%-------------------------
\subsection{H-theorem and the Maxwell-Boltzmann distribution}
The $H$ theorem can be proved using the $H$ functional defined by Boltzmann 
\begin{equation}
   H=\int f(\vec{v},t) \ln f(\vec{v},t) \mathrm{d}^{3}v,\label{hfunctional}
\end{equation} 
where $f(\vec{v},t)$ is a homogeneous distribution function. This $H$ functional describes a system with a global energy $E$, a global number of free
classical particles $N$, global volume $V$, and a temperature $T$. 
Besides, the Boltzmann's $H$ functional correspond to a density of entropy when the system arrives at the equilibrium state.\\
\textcolor{red}{The $H$ theorem establishes that if $f(\vec{v},t)$ satisfies the Boltzmann transport equation and the molecular chaos hypothesis, then $\frac{dH}{dt}\leq 0$ and the system arrives to the equilibrium state when $\frac{dH}{dt}=0$. The theorem can be proved by applying the time derivative to (\ref{hfunctional}) and using the Boltzmann transport equation (\ref{transport}). As a consequence of this theorem, the distribution function that describes the system in equilibrium corresponds to the Maxwell-Boltzmann distribution. This distribution can be obtained by solving a sufficient condition for $f$ to obtain the equilibrium distribution function $f_0$
\begin{equation}
    f_0(\vec{v}'_2)f_0(\vec{v}'_1)- f_0(\vec{v}_2)f_0(\vec{v}_1)=0.
\end{equation}
This condition can be obtained from the binary collision term of the Boltzmann transport equation (\ref{transport}).\\
The $H$-theorem fixes the time evolution direction of systems in nature, such as the second law of thermodynamics does. \footnote{The molecular hypothesis is fundamental to determine the physical meaning of the $H$ theorem. For more information, see \cite{huang}.}}
%---------------------------
\subsection{Out of equilibrium, non-homogeneous distributions}
The Boltzmann $H$-functional (\ref{hfunctional}) is defined with a spatial homogeneous distribution $f(\vec{v},t)$. However, we want to consider a $H$ functional with a non-homogeneous distribution function in order to generalize the Boltzmann $H$ theorem. Because of this, we introduce a modified $H$ functional (denoted as $H'$) defined as
\begin{equation}
   H'(t)=\sum_{M=1}^{K}\int_{}^{} f_M(\vec{v},t) \ln f_M(\vec{v},t)\mathrm{d}^3v  \label{CH2}.
\end{equation}
This $H'$ describes a system with a global energy $E$, a global number of free
classical particles $N$, global volume $V$, and a temperature $T$. Nevertheless, the global system, in turn, consists of a set of $K$ small cells with a constant volume $\delta V_M$, whose local homogeneous distribution functions is $f_{M}(\vec{v},t)$. The distribution function will depend on the position of the cells, the velocity $\vec{v}$, and time $t$. Without loss of generality, we choose these cells to have
local identical volumes (\textit{i.e.} $\delta V_M = V/K, \forall M$). Each cell has its local chemical potential, $\mu_M$, local
$H$ functional, $H'_M$, local number of particles, $\mathcal{N}'_M$, and local energy, $\mathcal{E}'_M$. In terms of the local distribution 
function and the energy $\epsilon(\vec{v})$, the last three properties are given by:
\begin{eqnarray}
    H'_M &=&  \int f_M(\vec{v},t) \ln f_{M}(\vec{v},t) \mathrm{d}^{3}v \label{Hcell},\\
    {\mathcal{N}}'_M&=& \int f_{M}(\vec{v} ,t) \mathrm{d}^{3}v, \nonumber \\
{\mathcal{E}}'_M&=& \int f_{M}(\vec{v},t)\epsilon(\vec{v}) \mathrm{d}^{3}v.
\end{eqnarray}
When the system is in equilibrium, the local number of particles and the local energy don't depend on the cell number, this is
\begin{equation}
   {\mathcal{N}}'_M=\mathcal{N}'\equiv \bar{\mathcal{N}}'; \ \ \ \  {\mathcal{E}}'_M=\mathcal{E}'\equiv \bar{\mathcal{E}}'.
\end{equation}
The particles in the system are considered to be free, \textit{i.e.} the particles do not interact with each other. Besides, the local distribution function must satisfy the following microcanonical restrictions 
\begin{equation}
    \sum_{M=1}^{K}\int_{}^{}f_M(\vec{v},t)\mathrm{d}^3v =N, \ \ \ \sum_{M=1}^{K}\int_{}^{}f_M(\vec{v},t)\epsilon(\vec{v})\mathrm{d}^3v =E \label{micro}.
\end{equation}
We can see that the functional (\ref{CH2}) contains a non-homogeneous distribution function but keeps the same form of the Boltzmann's $H$ functional.\\
Inspired by the Hamilton principle, $H'$ will be maximized with those restrictions to obtain the equilibrium distribution function for a classical gas using the Lagrange multipliers method obtaining
\begin{eqnarray}
    \frac{\delta H'}{\delta f_J(\vec{v}')}&=&\sum_{M=1}^{K}\int_{}^{}\frac{\delta}{\delta f_J(\vec{v}')}\left[f_M(\vec{v})\ln f_M(\vec{v})  \right]\mathrm{d}^3v -\sum_{M=1}^{K}\alpha_M\int_{}^{}\frac{\delta f_M(\vec{v})}{\delta f_J(\vec{v}')}\mathrm{d}^3v \nonumber \\
    &&-\sum_{M=1}^{K}\beta_M \int_{}^{}\epsilon(\vec{v})\frac{\delta f_M(\vec{v})}{\delta f_J(\vec{v}')}\mathrm{d}^3v \nonumber \\
    &=&\ln f_J(\vec{v}')+1-\alpha_J-\beta_J \epsilon(\vec{v}')=0.
\end{eqnarray}
Isolating the distribution function 
\begin{eqnarray}
    \ln f_J(\vec{v}')&=&\alpha_J+\beta_J \epsilon(\vec{v}')-1 \ \ \  \Rightarrow \ \ \ f_J(\vec{v}')=e^{\alpha_J +\beta_J \epsilon(\vec{v}')-1} \nonumber \\
    &=&Ce^{\alpha_J+\beta_J \epsilon(\vec{v}') } \label{relacion1},
\end{eqnarray}
where $C$ is a constant. It is trivial that using the variational procedures we obtain the form of the Maxwell-Boltzmann distribution. \\
As $H'$ in equilibrium is proportional to entropy, it is trivial to think that Lagrange multipliers do not depend on the position because of the system is in equilibrium and consequently, the distribution function is homogeneous, that is
\begin{equation}
    f(\vec{v})=Ce^{\alpha+\beta \epsilon(\vec{v})}\equiv \bar{f}(\vec{v}).
\end{equation}{} 
The constant $C$ can be omitted defining the following $H$ functional
\begin{equation}
   H''(t)=\sum_{M=1}^{K}\int_{}^{} \left[f_M(\vec{v},t) \ln f_M(\vec{v},t)-f_M(\vec{v},t)\right]\mathrm{d}^3v  \label{CH3},
\end{equation}
  but $\sum_{M=1}^{K} \int f_M(\vec{v},t)\mathrm{d}^3v =N$ where $N$ is the total particle number. As $N$ is a constant and any $H$ functional has sense when we calculate its time derivative, this constant can be omitted. \\
  On the other hand, if the distribution function is spatial-homogeneous, $f_M$ does not depend on the cell number $M$, then we rewrite (\ref{CH2}) as
  \begin{equation}
      H'(t)=\int \sum_{M=1}^{K} [f(\vec{v},t)\ln f(\vec{v},t)] \mathrm{d}^{3}v = K\int  f(\vec{v},t) \ln f(\vec{v},t) \mathrm{d}^{3}v= K H_{boltz}(t), 
  \end{equation}
  where
  \begin{equation}
      H_{boltz}(t)=\int  f(\vec{v},t) \ln f(\vec{v},t) \mathrm{d}^{3}v.
  \end{equation}
  Besides, if we analyze the following expression
  \begin{eqnarray}
      \int Kf(\vec{v},t) \ln [Kf(\vec{v},t)] \mathrm{d}^{3}v = \int [(K\ln K)f(\vec{v},t) + Kf(\vec{v},t) \ln f(\vec{v},t)]\mathrm{d}^{3}v, \label{sum-h}
  \end{eqnarray}
  we can observe if the number of particles in the $\mu$-space $f(\vec{v},t)$ is to big compare with the number of cells $P$, then the first term in (\ref{sum-h}) is negligible, and consequently, we conclude that
  \begin{equation}
      \int f'(\vec{v},t) \ln f'(\vec{v},t) \mathrm{d}^{3}v = K\int f(\vec{v},t) \ln f(\vec{v},t) \mathrm{d}^{3}v; \ \ \ f'(\vec{v},t)= Kf(\vec{v},t). \label{aditive-property} 
  \end{equation}
  We obtained the additive property when we supposed the distribution function is spatial-homogeneous.\\
  \\
To prove the Boltzmann's $H$ theorem (using (\ref{CH2}) as functional), we need to assume that the distribution function $f_M(\vec{v},t)$ must satisfy the following assumptions:
\begin{itemize}
    \item The local equilibrium hypothesis in each cell ($f_{M}$ must be spatial-homogeneous for each $M$).
    \item The non-homogeneous distribution function assumption in the total volume ($f_{M}$ must be different among all cells).
    \item The following restrictions
    \begin{eqnarray}
        \int f_{M}(\vec{v},t) \mathrm{d}^{3}v=\bar{\mathcal{N}}+\Delta_M(t); \ \ \ \ \int f_{M}(\vec{v},t) \epsilon(\vec{v}) \mathrm{d}^{3}v=\bar{\mathcal{E}}+ \delta_M(t), \label{restrictionoutsideclassical}
  \end{eqnarray}
  where $\bar {\mathcal{N}}$ and $\bar{\mathcal{E}}$ are the local particle number and the local energy in equilibrium
  \begin{equation}
      \bar{\mathcal{N}}= \int \bar{f}(\vec{v}) \mathrm{d}^{3}v ; \ \ \ \ \bar{\mathcal{E}}= \int \bar{f}(\vec{v})\epsilon(\vec{v}) \mathrm{d}^{3}v,
  \end{equation}
  and $\Delta_M,\delta_M$ could be seen as a deviation from $\bar{\mathcal{N}}$ and $\bar{\mathcal{E}}$ respectively, with $\Delta_M(t)\ll \bar{\mathcal{N}}$ and $\delta_M(t) \ll \bar{\mathcal{E}}$.
\end{itemize} 
 Those conditions described a system out of equilibrium but no so far from it, that is a perturbed system and will suffer a \textit{relaxation process}\footnote{The time evolution of a perturbated system to the equilibrium state.}.\\
  It is important to remark that the quantities $\Delta_M,\delta_M$ are sufficiently big to be different from fluctuations in the system but too small such that the system is not far from the equilibrium state.  \\
Performing the derivative of (\ref{CH2}) with respect to time, we obtain
\begin{equation}
    \frac{dH'}{dt}=\sum_{M=1}^{K}\int_{}^{}\left[ 1+\ln f_M(\vec{v},t) \right]\dot f_M(\vec{v},t) \mathrm{d}^3v  \label{dH1}.
\end{equation}{}
Using the first-order approximation
\begin{eqnarray}
   f_{Mn}(\vec{v},t)=\bar{f}_{n}(\vec{v})(1+g_{Mn}(\vec{v},t)): \ \ \ \bar{f}_{n}(\vec{v})\gg \bar{f}_{n}(\vec{v})|g_{Mn}(\vec{v},t)|, \label{firstorder}
\end{eqnarray}
(\ref{dH1}) yields
\begin{equation}
    \frac{dH'}{dt}=\sum_{M=1}^{K}\int_{}^{}\bar f(\vec{v}) \left [ 1+\ln \left\{ \bar f(\vec{v})+\bar f(\vec{v})g_M(\vec{v},t) \right\} \right]\dot g_M(\vec{v},t)\mathrm{d}^3v  \label{dH1,1}.
\end{equation}
We can approximate the logarithm function in (\ref{dH1,1}) to its first-order Taylor series around $\bar f_n(\vec{v}) g_{nM}(\vec{v},t)=0$ 
\begin{equation}
    \ln [\bar{f}(\vec{v})+\bar{f}(\vec{v}) g_{M}(\vec{v},t)] \approx \ln [\bar{f}(\vec{v})]+ g_{M}(\vec{v},t). \label{lnapproximationclassical}
\end{equation}
With the previous approximation, (\ref{dH1,1}) obtains the following form 
\begin{eqnarray}
    \frac{dH'}{dt}&=&\sum_{M=1}^{K} \int_{}^{} \bar f(\vec{v})\left[ 1+\ln \bar f(\vec{v})+g_M(\vec{v},t) \right]\dot g_M(\vec{v},t)\mathrm{d}^3v,
\end{eqnarray}{}
using (\ref{relacion1}) with homogeneous multipliers
\begin{eqnarray}
    \frac{dH'}{dt}&=&\sum_{M=1}^{K}\int_{}^{}\bar f(\vec{v})\left[ \alpha+\beta \epsilon(\vec{v}) \right]\dot g_M(\vec{v},t)\mathrm{d}^3v +\sum_{M=1}^{K}\int_{}^{}\bar f(\vec{v})g_M(\vec{v},t)\dot g_M(\vec{v},t)\mathrm{d}^3v \label{dH1,2}. \nonumber \\
\end{eqnarray}
On the other hand, we obtain from the restrictions the following expressions
\begin{eqnarray}
    \int \bar{f}(\vec{v}) g_{M}(\vec{v},t) \mathrm{d}^{3}v=\Delta_M(t) \ \  &\Rightarrow& \ \  \int \bar{f}(\vec{v}) \dot{g}_{M}(\vec{v},t)\mathrm{d}^{3}v=\dot{\Delta}_M(t), \nonumber \\
    \int  \bar{f}(\vec{v}) g_{M}(\vec{v},t)\epsilon(\vec{v}) \mathrm{d}^{3}v=\delta_M(t) \ \  &\Rightarrow& \ \  \int \bar{f}(\vec{v}) \dot{g}_{M}(\vec{v},t)\epsilon(\vec{v}) \mathrm{d}^{3}v=\dot{\delta}_M(t), \nonumber \\
\end{eqnarray}{}
and as a consequence of $\sum_{M=1}^{K}
\Delta_M(t)  =\sum_{M=1}^{K} \delta_M(t) =0$, we find
\begin{equation}
    \sum_{M=1}^{K} \dot{\Delta}_M(t)  =\sum_{M=1}^{K} \dot{\delta}_{M}(t) =0.
\end{equation}
Then using the previous result to (\ref{dH1,2}), we find
\begin{eqnarray}
\frac{dH'}{dt}&=&\sum_{M=1}^{K}\int_{}^{}\bar f(\vec{v})g_M(\vec{v},t)\dot g_M(\vec{v},t)\mathrm{d}^3v.
\end{eqnarray}
Now, the summation over $M$ will be expressed in two summations
\begin{equation}
    \frac{dH'}{dt}=\sum_J^{L}\int_{}^{}\bar f(\vec{v})g_J^{+}(\vec{v},t)\dot g_J^{+}(\vec{v},t)\mathrm{d}^3v +\sum_J^{P}\int_{}^{}\bar f(\vec{v})g_J^{-}(\vec{v},t)\dot g_J^{-}(\vec{v},t)\mathrm{d}^3v. \label{classicalH3}
\end{equation}
where $L+P=K$. The group of $L$ cells is such that cells have an excess of particles or/and energy (from the mean value in equilibrium). In contrast, the group of $P$ cells is such that cells have to miss particles or/and energy (from the mean value in equilibrium). Besides, $\dot{g}^{+}_{J}$ represents the change on the deviation on cells that have an excess of particles or energy while $\dot{g}^{-}_{J}$  represents the change on the deviations on cells that have missing particles or energy. 
On the other hand, $g^{+}_{J}$  represents the value of the deviation on cells that have an excess of particles or energy. In contrast, $g^{-}_{J}$ represents the value of the deviation on cells that have missing particles or energy.\\
Also, on the one hand, $\dot{g}^{+}_{J}<0$ describes the loss of particles and/or energy and so, $g^{+}_{J}>0$. On the other hand, $\dot{g}^{-}_{J}>0$ describes the gain of particles and/or energy and therefore $g^{-}_{J}<0$. \\
We sort the previous ideas in the following form
\begin{eqnarray}
   &&g^{+}_{J}=+|g^{+}_{J}|; \ \ \  \dot{g}^{+}_{J}=-|\dot{g}^{+}_{J}|, \nonumber \\
   &&g^{-}_{J}=-|g^{-}_{J}|; \ \ \ \dot{g}^{-}_{J}=+|\dot{g}^{-}_{J}| \label{separacionclassical},
\end{eqnarray}
and consequently, (\ref{classicalH3}) obtains the following form
\begin{equation}
    \frac{dH'}{dt}=-\left[
                               \sum_J^{L}\int_{}^{}\bar f(\vec{v})|g_J^{+}(\vec{v},t)||\dot g_J^{+}(\vec{v},t)|\mathrm{d}^3v +\sum_J^{P}\int_{}^{}\bar f(\vec{v})|g_J^{-}(\vec{v},t)||\dot g_J^{-}(\vec{v},t)|\mathrm{d}^3v 
    \right]. \label{classicalH4}
\end{equation}
We can observe that $\bar{f}(\vec{v})$ in (\ref{classicalH4}) is always positive. Also the deviation and its derivative are positive, then all the expression is positive. However, the global sign makes the derivative of the $H'$ with respect to time is always less to zero. With this we proved that $\frac{dH'}{dt}<0$. If the system is in equilibrium, $g_{J}(\vec{v},t)=\dot g_J(\vec{v},t)=0$, therefore $\frac{dH'}{dt}=0$.\\
Joining both results, we can say the following statement:\\
Consider a classical gas in a total volume $V$ (divided in $K$ cells of equal volume elements), total energy $E$ and total number of free particles $N$. Consider also the system has inhomogeneities and suffers a relaxation process. If we define the following functional
\begin{equation}
   H'(t)=\sum_{M=1}^{K}\int_{}^{} f_M(\vec{v},t) \ln f_M(\vec{v},t)\mathrm{d}^3v  \label{CH3},
\end{equation}
where $f_M(\vec{v},t)$ is the local distribution function of each cell in the system, then in the first-order approximation
\begin{equation}
    \frac{dH'}{dt} \leq 0.
\end{equation}
This statement will be the classical $H$ theorem with inhomogeneities.\\
\textcolor{red}{In the next section, we present a proposal quantum version of the $H$ theorem defined by Tolman and our proposal quantum version of the $H$ theorem applying the method of the volume divided into cells.}

%------------------------------

\section{Quantum scheme}
\subsection{H theorem and the Fermi-Dirac and Bose-Einstein distribution}
The $H$ functional in quantum mechanics presented by Tolman \cite{tolman} is defined as 
\begin{equation}
    H=\sum_i n_i \ln n_i -(n_i\pm g_i)\ln (g_i \pm n_i)\pm g_i\ln g_i, \label{quantumh}
\end{equation}
where $n_i$ is the occupation number and $g_i$ is an energy cell. Also, an equation for the rate of change in the number of particles in group $\kappa$ is
\begin{eqnarray}
    \frac{d n_{\kappa}}{dt}&=&-\sum_{\lambda,(\mu \nu)}A_{\kappa\lambda,\mu\nu} n_{\kappa}n_{\lambda}(g_{\mu}\pm n_{\mu})(g_{\nu}\pm n_{\nu})\nonumber \\
    &&+\sum_{\lambda,(\mu \nu)}A_{\mu\nu,\kappa\lambda} n_{\mu}n_{\nu}(g_{\kappa}\pm n_{\kappa})(g_{\lambda}\pm n_{\lambda}),\label{changen}
\end{eqnarray}
where $A_{\kappa\lambda,\mu\nu}$ is a tensor expressed as
\begin{equation}
  A_{\kappa\lambda,\mu\nu}=\frac{4\pi^{2}}{h}\frac{|I_1\pm I_2|^2}{\Delta \epsilon},
\end{equation}
where $\epsilon$ is the energy and $|I_1-I_2|^2=|V_{mn,kl}|^2$ where $V_{mn,kl}$ is the perturbation matrix of binary particles. Those expressions are obtained by the perturbation theory.\\
\textcolor{red}{Tolman's proposal for the quantum $H$-theorem establishes that if we have a quantum gas that satisfies the rate of change equation (\ref{changen})\footnote{It is important to remark that the equal a priori probabilities and random a priori phases hypothesis are implicit in the rate of change equation (\ref{changen}) and the random a priori phases hypothesis is considered as the quantum analogous to the molecular chaos hypothesis \cite{binary}}, then the $H$ functional (\ref{quantumh}) satisfies that
\begin{equation}
    \frac{dH}{dt}\leq 0,
\end{equation}
and as a consequence of this theorem, when the system is in equilibrium, $\frac{dH}{dt}=0$.\\
The proof of this theorem is in the same way as in the classical version. Also, when the system is in equilibrium, the distribution functions that correspond to the equilibrium state are the Fermi-Dirac and the Bose-Einstein distribution. Those distributions can be obtained from a sufficient condition for the equilibrium state
\begin{equation}
    \ln \frac{n_{\kappa}}{g_{\kappa}\pm n_{\kappa}}+\ln \frac{n_{\lambda}}{g_{\lambda}\pm n_{\lambda}}=\ln \frac{n_{\mu}}{g_{\mu}\pm n_{\nu}}+\ln \frac{n_{\nu}}{g_{\nu}\pm n_{\nu}},
\end{equation}
where the upper sign refers to the Bose-Einstein distribution and the lower sign refers to the Fermi-Dirac distribution.}\\
In addition to this, Tolman showed that (\ref{quantumh}) can be reduced in the limit of high-energy levels to
\begin{equation}
    H = \sum_{\kappa} (n_{\kappa} \ln n_{\kappa} - n_{\kappa} \ln g_{\kappa}). \label{reduce-h}
\end{equation}
This expression also can be obtained also from the Boltzmann $H$ functional through defining 
\begin{equation}
    f=\frac{n_{\kappa}}{ g_{\kappa}}.
\end{equation}
%--------------------------
\subsection{Out of equilibrium, non-homogeneous distributions for degenerated systems}

We begin by defining the form of the functional 
\begin{eqnarray}
    \Ss (t)&=&\sum_{M=1}^{K} \sum_{n} [ f_{Mn}(\epsilon_{n},t) \ln f_{Mn}(\epsilon_{n},t)\nonumber \\
    &&\pm (1 \mp f_{Mn}(\epsilon_{n},t)) \ln (1 \mp f_{Mn}(\epsilon_{n},t)) ]   \delta V_M \delta \epsilon_n \label{entropy}.
\end{eqnarray}
This functional will be named \textit{variational entropy}.
This functional describes, as in the classical case, a system with a global energy $E$, a global number of free
quantum particles (which may be fermions for upper sing or bosons for lower sing) $N$, global volume $V$, and a temperature $T$. 
Besides, the variational entropy should correspond to a density of entropy when the system arrives at the equilibrium state. The
global system, in turn, consists of a set of $K$ small cells with a constant volume $\delta V_M$, whose local homogeneous distribution functions is $f_{Mn}(\epsilon_{n},t)$. The distribution function will depend on the position of the cells, the energy per cell $\epsilon_{n}$, and time $t$. Also, the energy spectra will be divided into $\delta \epsilon_n$ volumes. We will use the uppercase Latin index to denote cell numbers, and unless stated otherwise, $M = 1, 2 . . . , K$. Without loss of generality, we choose these cells to have
local identical volumes (\textit{i.e.} $\delta V_M = V/K, \forall M$). Each cell has its local chemical potential, $\mu_M$, local
variational functional, $\Sm$, local number of particles, $\mathcal{N}_M$, and local energy, $\mathcal{E}_M$. In terms of the local distribution 
function and local energy spectra, $\{\epsilon_{n}\}$, the last three properties are given by:
\begin{eqnarray}
    \Sm &=&  \sum_{n} [ f_{Mn}(\epsilon_{n},t) \ln f_{Mn}(\epsilon_{n},t)\nonumber \\
    &&\pm (1 \mp f_{Mn}(\epsilon_{n},t)) \ln (1 \mp f_{Mn}(\epsilon_{n},t)) ] \de V_M \delta \epsilon_n \label{entropycell},\\
    {\mathcal{N}}_M&=& \sum_{n}f_{Mn}(\epsilon_{n} ,t) \de V_M \delta \epsilon_n, \nonumber \\
{\mathcal{E}}_M&=& \sum_{n}f_{Mn}(\epsilon_{n},t)\epsilon_{n} \de V_M \delta \epsilon_n.
\end{eqnarray}
When the system is in equilibrium, the local number of particles and the local energy don't depend on the cell number, this is
\begin{equation}
   {\mathcal{N}}_M=\mathcal{N}\equiv \bar{\mathcal{N}}; \ \ \ \  {\mathcal{E}}_M=\mathcal{E}\equiv \bar{\mathcal{E}}.
\end{equation}
The particles in the system are considered to be free, \textit{i.e.} the particles do not interact with each other.
Consequently, we can assume that the available quantum states do not depend on the cell features,
therefore the energy spectrum, denoted by $\epsilon_n$, is the same for all cells, \textit{i.e.} $\epsilon_{Mn} = \epsilon_n, \forall M$. Besides, the system is subjected to the microcanonical restrictions
\begin{eqnarray}
    &&\sum_{M=1}^{K} \left[ \sum_{n}f_{Mn}(\epsilon_{n} ,t)\de \epsilon_n\right] \de V_M=\sum_{M=1}^{K} {\mathcal{N}}_{M} \de V_M=N; \nonumber \\
    &&\sum_{M=1}^{K}\left[ \sum_{n}f_{Mn}(\epsilon_{n},t)\epsilon_{n}\delta \epsilon_n\right] \de V_M=\sum_{M=1}^{K} {\mathcal{E}}_M \delta V_M=E, \label{restriccions1}
\end{eqnarray}
where $N$ and $E$ are the total particle number and the total energy of the system respectively.\\ 
Maximizing the variational entropy $\Ss$ using the Lagrange multipliers method, we obtain
\begin{eqnarray}
&&\ln \left(\frac{1\mp f_{Mn}(\epsilon_{n},t)}{f_{Mn}(\epsilon_{n},t)} \right)=-\alpha_M(t)-\beta_M(t) \epsilon_{n}, \label{relation}\\ &&\Rightarrow \ \ f_{Mn}(\epsilon_{n},t)=\frac{1}{e^{-\alpha_M(t)-\beta_M(t) \epsilon_{n}}\pm 1} \equiv \bar{f}_{Mn}(\epsilon_{n},t) \label{distributionequilibrium}.
\end{eqnarray}
where $\alpha_M(t), \beta_M(t)$ are the Lagrange multipliers. We can obtain from the variational procedure that 
\begin{equation}
    \alpha_M\propto \frac{\partial \ln \Omega_M}{\partial \mathcal{N}_M}, \ \ \ \beta_M\propto \frac{\partial \ln \Omega_M}{\partial \mathcal{E}_M},\label{multipliers}
\end{equation}
where we used that the $H$ functional in equilibrium is proportional to the entropy of the system.
In other hand, in the expression (\ref{multipliers}), we expect that the variation of the logarithm with the number of particles by cells is indifferent from a particular cell. With this, we conclude  
\begin{equation}
    \alpha_M=\alpha; \ \ \ \ \beta_M=\beta.
\end{equation}
In equilibrium, the energy and the Lagrange multiplier do not depend on the cell number and time, therefore
\begin{equation}
    \bar f_{Mn}(\epsilon_{n},t)=\bar f_n(\epsilon_{n}) =\frac{1}{e^{-\alpha-\beta \epsilon_n}\pm 1}.
\end{equation}{}
If one compares those multipliers with the mean energy and the mean particle number from the statistical mechanic's results, we obtain
\begin{equation}
    \alpha=\frac{\mu}{kT}\equiv \bar{\alpha}; \ \ \ \ \beta=-\frac{1}{kT}\equiv \bar{\beta},
\end{equation}{}
and finally
\begin{eqnarray}
    \bar{f}_{n}(\epsilon_{n})&=&\frac{1}{e^{(\frac{{\epsilon_n}-\bar{\mu}}{kT})}\pm 1}\equiv \bar{f}_{n}.
\end{eqnarray}{}
We proved, when the system is in equilibrium, the Lagrange multipliers don't depend on the cell number. In other words, Lagrange multipliers are homogeneous in the system.\\
We can calculate the entropy from an ideal system, in other words, if one substitute (\ref{distributionequilibrium}) to (\ref{entropy}) we obtain
\begin{eqnarray}
      \Ss(t)&=&\sum_{M=1}^{K} \sum_n  \left[\left(\frac{1}{e^{-\bar{\alpha}-\bar{\beta}\epsilon_{n}}\pm 1} \right)\ln \left(\frac{1}{e^{-\bar{\alpha}-\bar{\beta}\epsilon_{n}}} \right) \right]\nonumber \\
      &&\pm  \ln \left[\prod_{M=1}^{K} \prod_{n}\left(1 \mp \frac{1}{e^{-\bar{\alpha}-\bar{\beta}\epsilon_{n}}\pm 1} \right) \right] \de V_M \delta \epsilon_n\label{H-entropy}.
  \end{eqnarray}
  This result allows us to calculate the entropy of a quantum ideal gases. This entropy could be used as a thermodynamic variable to describe those systems.\\
 %\subsection{The quantum analogous of the Boltzmann's $H$ theorem}
We rewrite the \textit{variational entropy} (\ref{entropy}) as
\begin{equation}
    \Ss(t)=\sum_{M=1}^{K} \sum_{n} \left[ f_{Mn} \ln f_{Mn} \pm (1 \mp f_{Mn}) \ln (1 \mp f_{Mn}) \right] \de V_M \delta \epsilon_n \label{entropy2},
\end{equation}
where, for simplicity, we have defined $f(\epsilon_{n},t)\equiv f_{Mn}$.
This distribution function must satisfy the local equilibrium hypothesis in each cell ($f_{nM}$ must be spatial-homogeneous for each $M$), the non-homogeneous distribution function assumption in the total volume ($f_{nM}$ must be different among all cells), and the following restrictions
\begin{eqnarray}
        \sum_{n}f_{Mn} \delta \epsilon_n=\bar{\mathcal{N}}+\Delta_M(t); \ \ \ \ \sum_{n}\epsilon_{n}f_{Mn} \delta \epsilon_n=\bar{\mathcal{E}}+ \delta_M(t), \label{restrictionoutside}
  \end{eqnarray}
  where $\bar {\mathcal{N}}$ and $\bar{\mathcal{E}}$ are the local particle number and the local energy in equilibrium
  \begin{equation}
      \bar{\mathcal{N}}= \sum_n \bar{f}_n \delta \epsilon_n; \ \ \ \ \bar{\mathcal{E}}= \sum_n \epsilon_n\bar{f}_n \delta \epsilon_n,
  \end{equation}
  and $\Delta_M,\delta_M$ could be seen as a deviation from $\bar{\mathcal{N}}$ and $\bar{\mathcal{E}}$ respectively, with $\Delta_M(t)\ll \bar{\mathcal{N}}$ and $\delta_M(t) \ll \bar{\mathcal{E}}$. The previous conditions and restrictions are the quantum analogous case of the classical case. Also, this system will suffer, as in the classical case, a \textit{relaxation process}.\\
  The quantities $\Delta_M,\delta_M$ are sufficiently big to be different from fluctuations in the system but sufficiently small to consider the system not far from the equilibrium state, as in the classical case.  \\
If we perform the derivative of the variational entropy with respect to time, we obtain
\begin{equation}
   \frac{d \Ss (t)}{dt}= \sum_n \sum_{M=1}^{K} \dot{f}_{nM}(t)\ln \left[ \frac{f_{nM}(t)}{1\mp f_{nM}(t)} \right] \de V_M \delta \epsilon_n.\label{deltaH}
\end{equation}{}
Using the first-order approximation (\ref{firstorder}), we get
\begin{eqnarray}
    \frac{d\Ss (t)}{dt}&=&\sum_n \sum_{M=1}^{K} \bar{f}_{n}\ln \left[ \frac{\bar{f}_{n}(1+g_{nM})}{1\mp \bar{f}_{n} (1+ g_{nM})} \right]\dot{g}_{nM} \de V_M\delta \epsilon_n \nonumber \\
    &=&\sum_n \sum_{M=1}^{K} \bar{f}_n \left \{ \ln [\bar{f}_n+\bar{f}_n g_{nM}]\dot{g}_{nM}-\ln [1\mp\bar{f}_n\mp\bar{f}_n g_{nM}]\dot{g}_{nM}  \right \}\de V_M \delta \epsilon_n.\nonumber \\
    \label{cambioH1}
\end{eqnarray}{}
It's necessary to remark that, in the case of fermions
\begin{eqnarray}
   1-\bar f_n -\bar f_n g_{nM}>0 \ \ \Rightarrow \ \ \frac{1}{\bar f_n}>1+g_{nM}. \label{fermionrestriction}
\end{eqnarray}{}
The previous expression establishes that the value of $g_{nM}$ is determined by $\bar f_n$. Furthermore, whether $\bar f_{n}=1$, this is, all lower energy levels are occupied, then the system will not have inhomogeneities because of the exclusion principle, and as a consequence of this,  $g_{nM}=0$. This result corresponds to the system in the zero temperature condition. We can observe that in this condition, (\ref{fermionrestriction}) is violated. Due to this, the zero temperature condition is excluded.\\
\\
We can observe in the case of bosons, if $\bar{f}_n \gg \bar{f}_n |g_{nM}|$ then $1+\bar{f}_n \gg \bar{f}_n |g_{nM}|$. In fermions, in the case of a non-extremely degenerated condition, it is true that $1-\bar{f}_n \gg \bar{f}_n |g_{nM}|$. With those relations, we can approximate the logarithm functions in (\ref{cambioH1}) to their first-order Taylor series around $\bar f_n g_{nM}=0$ 
\begin{equation}
    \ln [\bar{f}_n+\bar{f}_n g_{nM}] \approx \ln [\bar{f}_n]+ g_{nM}; \ \ \ \ \ln[1\mp\bar{f}_n\mp\bar{f}_n g_{nM}] \approx \ln[1\mp\bar{f}_n]\mp\frac{\bar{f}_n}{1\mp\bar{f}_{n}} g_{nM}. \label{lnapproximation}
\end{equation}{}
With the previous approximation, (\ref{cambioH1}) will be
\begin{eqnarray}
    \frac{d\Ss}{dt}&=&\sum_n \sum_{M=1}^{K} \bar{f}_n\left \{ (\ln \bar{f}_n+ g_{nM})\dot{g}_{nM}\right\} \de V_M \delta \epsilon_n \nonumber \\
    &&-\sum_{n}\sum_{M=1}^{K}\bar f_n\left\{ \left( \ln[1\mp\bar{f}_n]\mp \left[\frac{\bar{f}_n}{1\mp\bar{f}_n} \right] g_{nM}\right)\dot{g}_{nM} \right \}\de V_M \delta \epsilon_n.\label{cambioH2}
\end{eqnarray}{}
Making use of the expression (\ref{relation}),
(\ref{cambioH2}) casts into
\begin{eqnarray}
    \frac{d\Ss}{dt}&=&\sum_n \sum_{M=1}^{K} \bar{f}_n\left \{ (\bar{\alpha}+\bar{\beta}{\epsilon}_n)\dot{g}_{nM}+ g_{nM}\left(1\pm e^{\bar{\alpha}+\bar{\beta}{\epsilon}_n}\right)\dot{g}_{nM} \right \} \de V_M \delta \epsilon_n. \nonumber \\
    \label{cambioH3}
\end{eqnarray}{}
On the other hand, we obtain from the restrictions the following expressions
\begin{eqnarray}
    &&\sum_n \bar{f}_n g_{nM} \delta \epsilon_n=\Delta_M(t) \ \  \Rightarrow \ \  \sum_n \bar{f}_n \dot{g}_{nM} \delta \epsilon_n=\dot{\Delta}_M(t), \nonumber \\
    &&\sum_n  \bar{f}_n g_{nM}\epsilon_n \delta \epsilon_n=\delta_M(t) \ \  \Rightarrow \ \  \sum_n \bar{f}_n \dot{g}_{nM}\epsilon_n \delta \epsilon_n=\dot{\delta}_M(t)
\end{eqnarray}{}
and as a consequence of $\sum_{M=1}^{K} \Delta_M(t) \de V_M  =\sum_{M=1}^{K} \delta_M(t) \de V_M =0$, we find
\begin{equation}
    \sum_{M=1}^{K} \dot{\Delta}_M(t) \de V_M =\sum_{M=1}^{K} \dot{\delta}_{M}(t) \de V_M=0.
\end{equation}{}
Substituting the previous expression to (\ref{cambioH3}) we obtain
\begin{eqnarray}
   \frac{d\Ss}{dt}&=&  \sum_n e^{\bar{\alpha}+\bar{\beta}\epsilon_n}\sum_M  g_{nM}\dot{g}_{nM} \de V_M\delta \epsilon_n. \nonumber \\ \label{cambioH4}
\end{eqnarray}{}
Now, the summation over $M$ will be expressed in two summations, such as in the classical case
\begin{equation}
    \frac{d\Ss}{dt}=\sum_n  e^{\bar{\alpha}+\bar{\beta}\epsilon_n}\left(\sum_J ^{L} g^{+}_{nJ}\dot{g}^{+}_{nJ}\de V_J \delta \epsilon_n+\sum^{P}_J  g^{-}_{nJ}\dot{g}^{-}_{nJ} \de V_J\delta \epsilon_n \right), \label{cambioH5}
\end{equation}
where $L+P=K$. Besides, $\dot{g}^{+}_{J}$ represents the change on the deviation on cells that have an excess of particles or energy while $\dot{g}^{-}_{J}$  represents the change on the deviations on cells that have missing particles or energy. 
On the other hand, $g^{+}_{J}$  represents the value of the deviation on cells that have an excess of particles or energy. In contrast, $g^{-}_{J}$ represents the value of the deviation on cells that have missing particles or energy.\\
Also, on the one hand, $\dot{g}^{+}_{nJ}<0$ describes the loss of particles and/or energy and so, $g^{+}_{nJ}>0$. On the other hand, $\dot{g}^{-}_{nJ}>0$ describes the gain of particles and/or energy and therefore $g^{-}_{nJ}<0$. \\
We sort the previous ideas in the following form
\begin{eqnarray}
   &&g^{+}_{nJ}=+|g^{+}_{nJ}|; \ \ \  \dot{g}^{+}_{nJ}=-|\dot{g}^{+}_{nJ}| \nonumber \\
   &&g^{-}_{nJ}=-|g^{-}_{nJ}|; \ \ \ \dot{g}^{-}_{nJ}=+|\dot{g}^{-}_{nJ}| \label{separacion},
\end{eqnarray}{}
\\
and consequently, (\ref{cambioH5}) obtains the following form
\begin{equation}
    \frac{d\Ss}{dt}=-\sum_n  e^{\bar{\alpha}+\bar{\beta}\epsilon_n}\left(\sum_J ^{L} |g^{+}_{nJ}||\dot{g}^{+}_{nJ}|\de V_J \delta \epsilon_n+\sum^{P}_J  |g^{-}_{nJ}||\dot{g}^{-}_{nJ}| \de V_J \delta \epsilon_n \right), \label{cambioH6}
\end{equation}{}
and due to $e^{\bar{\alpha}+\bar{\beta}\epsilon_n}$ is always positive, then $\frac{d\Ss}{dt}<0$. With this, we proved that any quantum ideal gas perturbed (out of equilibrium but not so far from it), evolves such that $\frac{d\Ss}{dt}<0$.\\
When the system is in equilibrium, $g_{nM}=0$ and in consequence of that, $\dot g_{nM}=0$ and finally from (\ref{cambioH1}) $\frac{d\Ss}{dt}=0$. Then, we can say that any ideal quantum gas perturbed always evolves to the equilibrium state, such as the classical $H$ theorem. Then, we obtained the quantum version of the Boltzmann's $H$ theorem including inhomogeneities in the system.\\
We obtained some results from the Statistical Mechanic in Quantum Mechanics scheme. However, the correspondence principle is important to recover all results from Classical Mechanics. This correspondence principle will be discussed in the next section.

  %-----------------------------------------
  

%---------------------------------------------

\section{Quantum Classical correspondence}
%The variational entropy $H$ that was defined in (\ref{entropy2}) can be reduce to the Boltzmann's $H$ functional in the appropriated limit.\\
We begin defining the distribution function as the number of particles in a defined cell and an energy level $N_{nM}$ inside a volume $\de V_M \delta \epsilon_n$ as
\begin{equation}
    f_{nM}=\frac{N_{nM}}{ \de V_M \delta \epsilon_{n} }.
\end{equation}
Using this definition on the variational entropy (\ref{entropy2}) we obtain
\begin{eqnarray}
    \Ss&=& \sum_{M=1}^{K} \sum_n
    \left[  
           \frac{N_{nM}}{ \de V_M\delta \epsilon_{n}} \ln 
           \left( 
                  \frac{N_{nM}}{ \de V_M\delta \epsilon_{n}}
           \right)\pm 
           \left(  
                  1\mp \frac{N_{nM}}{ \de V_M \delta \epsilon_{n}}
           \right) \ln 
           \left(  
                   1\mp \frac{N_{nM}}{ \de V_M \delta \epsilon_{n}}
           \right)
    \right] \de V_M \delta \epsilon_{n}. \nonumber \\
    \label{h-quantic} 
\end{eqnarray}
In the classic limit, the particles occupy high energy levels, and the number of particles in most of the groups of state $n$ is small compared with the number of the elementary states $\epsilon_n$ of the group, that is
\begin{equation}
    \frac{N_{nM}}{ \de V_M \delta \epsilon_{n} } \approx 0,
\end{equation}
and with the previous approximation, the second term in (\ref{h-quantic}) is negligible, then (\ref{h-quantic}) yields
\begin{eqnarray}
    \Ss&=& \sum_{M=1}^{K} \sum_n
    \left[  
           \frac{N_{nM}}{ \de V_M \delta \epsilon_{n}} \ln 
           \left( 
                  \frac{N_{nM}}{ \de V_M \delta \epsilon_{n}}
           \right)
    \right] \de V_M \delta \epsilon_{n}. \label{h-quantic2}
\end{eqnarray}
In the case when the system is spatial-homogeneous, the definition of the distribution function is 
\begin{equation}
    f_{n}=\frac{N_{n}}{ \delta \epsilon_{n} },
\end{equation}
and using the same limit we obtain
\begin{eqnarray}
    \Ss&=& \sum_{M=1}^{K} \sum_n
    \left[  
           \frac{N_{n}}{ \delta \epsilon_{n}} \ln 
           \left( 
                  \frac{N_{n}}{ \delta \epsilon_{n}}
           \right)
    \right]  \de V_M \delta \epsilon_{n} = V \sum_n
    \left[  
           \frac{N_{n}}{ \delta \epsilon_{n}} \ln 
           \left( 
                  \frac{N_{n}}{ \delta \epsilon_{n}}
           \right)
    \right] \delta \epsilon_{n} \nonumber \\
    &=& V \sum_n \left[N_n \ln N_n - N_n \ln \delta \epsilon_n  \right]=\de V_M K H_{Boltz},\label{h-quantic4}
\end{eqnarray}
where $H_{Boltz}$ now is (\ref{CH2}) expressed is the form of (\ref{reduce-h}) shown by Tolman.\footnote{See \cite{tolman} eq. (102.6) for more information.}\\
We can observe that if the number of particles is too big compared to the number of cells $K$, then we can conclude that the sum of the local variational entropy is equal to the variational entropy of the entire system.\\
\textcolor{blue}{In the equilibrium case, we have the following variational entropy
\begin{eqnarray}
    \Ss&=& \sum_{M=1}^{K}\sum_{n}
        \left[
                \frac{1}{e^{-\alpha_M-\beta_M \epsilon}\pm 1} \ln 
                    \left(
                            \frac{1}{e^{-\alpha_M-\beta_M \epsilon}\pm 1}
                    \right)
        \right. \nonumber \\
          && \pm \left. \left(
                        1 \mp \frac{1}{e^{-\alpha_M-\beta_M \epsilon}\pm 1}
                  \right) \ln
                \left(
                        1 \mp \frac{1}{e^{-\alpha_M-\beta_M \epsilon}\pm 1}            
                \right) \right]. \label{equilibriumvariational}
\end{eqnarray}
To obtain the classical limit, that is, to recover the Maxwell-Boltzmann distribution from the Fermi-Dirac and Bose-Einstein distribution, it is necessary to hold the following approximation
\begin{equation}
    e^{-\alpha_M-\beta_M \epsilon}\gg 1, \label{classicalapproximation}
\end{equation}
and then
\begin{equation}
    \frac{1}{e^{-\alpha_M-\beta_M \epsilon}\pm 1} \approx e^{\alpha_M+\beta_M \epsilon}.
\end{equation}
If we apply (\ref{classicalapproximation}) to (\ref{equilibriumvariational}), we obtain
\begin{eqnarray}
    \Ss&=& \sum_{M=1}^{K}\sum_{n}
        \left[
                e^{\alpha_M+\beta_M \epsilon} \ln 
                    \left(
                            e^{\alpha_M+\beta_M \epsilon}
                    \right)
        \right. \nonumber \\
          && \pm \left. \left(
                        1 \mp e^{\alpha_M+\beta_M \epsilon}
                  \right) \ln
                \left(
                        1 \mp e^{\alpha_M+\beta_M \epsilon}            
                \right) \right],
\end{eqnarray}
but if (\ref{classicalapproximation}) holds, then 
\begin{equation}
     e^{\alpha_M+\beta_M \epsilon}\ll 1,
\end{equation}
consequently 
\begin{equation}
    \ln(1\mp e^{\alpha_M+\beta_M \epsilon}) \approx \ln 1 \approx 0,
\end{equation}
and finally
\begin{equation}
    \Ss=\sum_{M=1}^{K}\sum_{n}
        \left[
                e^{\alpha_M+\beta_M \epsilon} \ln 
                    \left(
                            e^{\alpha_M+\beta_M \epsilon}
                    \right)
        \right],
\end{equation}
that corresponds to the classical $H$ functional in equilibrium described in the energy space.}\\
\textcolor{red}{Finally, as a complementary topic, in the next section, we propose a method to obtain the quantum analogous from the Boltzmann transport equation for relaxation processes.}

%--------------------------------------

  \section{Relaxation processes in degenerated quantum gases}
Consider that the distribution function $f_{nM}$ is a function of the local number of particles $\mathcal{N}_M$, the local number of particles in a quantum energy state $\mathcal{N}_{M}^{n}$ and the time $t$ as follow
\begin{equation}
    f=f(\mathcal{N}_M,\mathcal{N}_M^{n},t),
\end{equation}
where $\mathcal{N}_{M}^{n}$ is defined as
\begin{equation}
    \mathcal{N}_{M}^{n}= f_{nM} \delta \epsilon_n.
\end{equation}
Besides the local number of particles is a function of time $\mathcal{N}_J=\mathcal{N}_J(t)$ and the local number of particles in a quantum energy state is a function of time and the local number of particles $\mathcal{N}_J^{j}=\mathcal{N}_J^{j}(\mathcal{N}_M,t)$. This is due to when a cell transfer particles to another cell, the particles in some defined state energy levels must change.\\
In order to obtain the analogous transport equation to this system, we apply the variation method to the functional $\Ss$. Due to $\Ss$ in equilibrium is proportional to entropy, and therefore a maximum, the variation must be zero. Then the equation of motion is
\begin{equation}
    0=\sum_{M=1}^{K} \sum_n \ln \left[ \frac{f_{nM}}{1\mp f_{nM}} \right] \left[ \frac{\partial f_{nM}}{\partial t}+\frac{\partial f_{nM}}{\partial \mathcal{N}_J}\dot{\mathcal{N}}_J+\frac{\partial f_{nM}}{\partial \mathcal{N}_J^{j}}\left( \frac{\partial \mathcal{N}_J^{j}}{\partial t}+\frac{\partial \mathcal{N}_J^{j}}{\partial \mathcal{N}_L}\dot{\mathcal{N}}_L \right) \right]. \label{transportequation1}
\end{equation}
This equation is valid in equilibrium. When the system is out of equilibrium, the variation is different from zero. Due to this, an expression has to be included in (\ref{transportequation1}). This term, in analogy to the Boltzmann transport equation, will be the collision term and will be expressed as $\left( \frac{df_{nM}}{dt} \right)_{coll}$. Moreover, we propose that this term contributes to the equation of motion per each cell.\footnote{The collision term is only in a defined cell and local group of energy. If we are describing the total system, we need to include all contributions (each cell and energy group).} Therefore, (\ref{transportequation1}) casts into
\begin{eqnarray}
    \sum_{M=1}^{K}\sum_n \left( \frac{df_{nM}}{dt} \right)_{coll}&=&\sum_{M=1}^{K} \sum_n \ln \left[ \frac{f_{nM}}{1\mp f_{nM}} \right] \nonumber \\
    &&\times \left[ \frac{\partial f_{nM}}{\partial t}+\frac{\partial f_{nM}}{\partial \mathcal{N}_J}\dot{\mathcal{N}}_J+\frac{\partial f_{nM}}{\partial \mathcal{N}_J^{j}}\left( \frac{\partial \mathcal{N}_J^{j}}{\partial t}+\frac{\partial \mathcal{N}_J^{j}}{\partial \mathcal{N}_L}\dot{\mathcal{N}}_L \right) \right].  \nonumber \\ \label{transportequation2}
\end{eqnarray}
The equation (\ref{transportequation2}) is the time evolution equation to a quantum ideal gas. \\


  
  %----------------------------------------


%-------------------------------------------------



\section{Discussions and remarks}

In the literature, the Boltzmann's $H$ theorem is proved using a spatially homogeneous distribution function. In this work, we used the local equilibrium hypothesis and as a consequence of this, it was necessary to include the homogeneous distribution function hypothesis. However, we included inhomogeneities in the system through the division of the volume $V$ in cells and the non-homogeneous distribution function assumption in the total volume. Moreover, the Boltzmann's $H$ theorem still holds when the system includes inhomogeneities, only if the local equilibrium hypothesis holds. We can include the fact that we proved the Boltzmann's $H$ theorem without using the Boltzmann's transport equation. It was only necessary to specify the behavior of the deviation $g$ and its derivative in time. Also, we obtained the Boltzmann's transport equation in equilibrium from the Boltzmann's $H$ functional through the variational method. Nevertheless, in order to obtain the complete Boltzmann's transport equation, it was necessary to include the collision term because of the variation of the $H$ functional is not zero when the system is out of equilibrium. This is a consequence of the $H$ theorem.\\
\\
In the quantum version, we used the same assumptions to prove the $H$ theorem, those were the behavior of the deviation $g_{nM}$ and its derivative, the local equilibrium hypothesis, and the non-homogeneous distribution function assumption. With the proof of the quantum version of the $H$ theorem, we can say that a quantum gas inside a volume $V$ that is out of equilibrium with inhomogeneities, in the first-order approximation the system evolves to the equilibrium state as long as the local equilibrium hypothesis holds. In addition to this, the system is in the equilibrium state when the derivative of the variational entropy with respect to time is equal to zero, and as a consequence of this, the distribution function corresponds to the Bose-Einstein or Fermi-Dirac distribution.\\
On the other hand, the equation of motion obtained from the variational entropy could be seen as a quantum transport equation. However, in order to develop the transport phenomena (in analogy to theoretical work developed by Boltzmann) is necessary to find an expression for the collision term. This problem may be future work.\\
\\
We also find that variational entropy is proportional to the Boltzmann's $H$ functional in the limit of high energy levels. This conclusion is trivial if we compare (\ref{h-quantic4}) with (\ref{reduce-h}). Moreover, if we see the expression for the Boltzmann's $H$ functional in equilibrium
\begin{equation}
    H_{Boltz}=-\frac{S}{Vk},
\end{equation}
where $S$ is the entropy of the system, and $k$ is the Boltzmann's constant, we can identify that variational entropy is proportional to the entropy of the system.\\
\begin{equation}
    \Ss\propto S.
\end{equation}
\textcolor{red}{Besides, we find an expression for the entropy density (\ref{H-entropy}). This expression will be useful to describe quantum gases in terms of entropy.\\
It is important to remark that we find that $\frac{dH'}{dt}\leq 0$ and $\frac{d\Ss}{dt}\leq 0$ setting the expecting behavior of the deviation $g$. This correct behavior involves the molecular chaos hypothesis in the classical case and the random a priori phases hypothesis in the quantum case. Specifically those hypothesis are included in $\dot{g}_{nJ}^{+}$ and $\dot{g}_{nJ}^{-}$. Those time derivatives correspond to the expected behavior in a meantime, that is, $\dot{g}_{nJ}^{+}$ always decreases and $\dot{g}_{nJ}^{-}$ always increases.\\ 
According to the classical analysis of the $H$ theorem, the expected behavior of the distribution function is presented in the system if the molecular chaos hypothesis holds, but in a meantime, called \textit{relaxation time}, the distribution function behaves as we expect, such as the distribution function evolves to the equilibrium state keeping that $\frac{dH}{dt}\leq 0$.\\
In the same way as the previous statement, we proposed that $\dot{g}_{nJ}^{+}$ and $\dot{g}_{nJ}^{-}$ have the expected behavior in a relaxation time. Also molecular chaos hypothesis and a random a priori phases hypothesis are implicit in the expected behavior of the deviation such as $\frac{dH'}{dt}\leq 0$ and $\frac{d\Ss}{dt}\leq 0$.}

%-----------------------------------------------

\section{Conclusions}
We conclude the following statements:\\
\begin{enumerate}
    \item Any system (classical or quantum) out of equilibrium (including inhomogeneities) evolves to the equilibrium state without using the transport equation as long as the local equilibrium hypothesis holds.
    \item We can obtain the transport equation from the $H$ functional.
    \item The variational entropy is proportional to the Boltzmann's $H$ functional in the limit of the high energy levels.
    \item We find an expression for the entropy to quantum gases.
    \end{enumerate}
  
%  \begin{equation}
%      \mathcal{H}
%  \end{equation}
  
\clearpage

\begin{thebibliography}{}
%\bibitem{goldstein}
%H. Goldstein, {\it Mec\'anica Cl\'asica}, Ed. 1987, Espa\~na: Editorial Revert\'e (2002).

%\bibitem{sakurai}
%J. J. Sakurai, J. Napolitano {\it Modern Quantum Mechanics}, Second Edition, Addison-Wesley, 1994.

\bibitem{huang}
K. Huang, \emph{Statistical mechanics}, John Wiley \& Sons, Inc., (1963).

\bibitem{reif}
F. Reif {\it Fundamentals of Statistical and Thermal Physics.},Waveland Press Inc, 2009.

\bibitem{patrick}
R. Fitzpatrick, {\it Thermodynamics and Statistical Mechanics: An intermediate level.},University Reprints 2012 (2012).

\bibitem{cristal1}
Zanotto, E.D.; Mauro, J.C., {\it Comment on “Glass Transition, Crystallization of Glass-Forming Melts, and Entropy”} Entropy 2018, 20, 103.. Entropy 2018, 20, 703.

\bibitem{cristal2}
Schmelzer, J.W.P.; Tropin, T.V. {\it Glass Transition, Crystallization of Glass-Forming Melts, and Entropy}, Entropy 2018, 20, 103.

\bibitem{cristal3}
Nemilov, S.V. {\it On the Possibility of Calculating Entropy, Free Energy, and Enthalpy of Vitreous Substances}, Entropy 2018, 20, 187.

\bibitem{kei}
Keizer J., \emph{Statistical thermodynamics of nonequilibrium processes}, Springer-Verlag, New York, 1987.

\bibitem{onsager}
Onsager, L. (1931), Physical Review 37 (4): 405-426.

\bibitem{paradox1}
Harvey R. Brown and Wayne Myrvold, {\it Boltzmann's H-theorem, its limitations, and the birth of (fully) statistical mechanics}, physics.hist-ph,0809.1304,2008.

\bibitem{paradox2}
Dragoljub A. Cucic, {\it Paradoxes of Thermodynamics and Statistical Physics}, physics.gen-ph,0912.1756,2009.

\bibitem{hamiltonian}
Wang, Q.A.; El Kaabouchiu, A. {\it From Random Motion of Hamiltonian Systems to Boltzmann’s H Theorem and Second Law of Thermodynamics: a Pathway by Path Probability}, Entropy 2014, 16, 885-894.

\bibitem{H-theorem-violation}
Gorban, A.N. {\it General H-theorem and Entropies that Violate the Second Law}, Entropy 2014, 16, 2408-2432.

\bibitem{H-theorem-and-entropy}
Ben-Naim, A. {\it Entropy, Shannon’s Measure of Information and Boltzmann’s H-Theorem}, Entropy 2017, 19, 48.

\bibitem{tolman} Richard C. Tolman, \emph{The principles of statistical mechanics}, Oxford, 1938.

\bibitem{entropic-framework}
Li, S.-N.; Cao, B.-Y. {\it On Entropic Framework Based on Standard and Fractional Phonon Boltzmann Transport Equations}, Entropy 2019, 21, 204.


\bibitem{thermal-harmonic}
F. Nicacio, A. Ferraro, A. Imparato, M. Paternostro, and F. L. Semião,
Phys. Rev. E 91, 042116 – 15 April 2015.

\bibitem{quantum-transport}
Robert Hussein and Sigmund Kohler,
Phys. Rev. B 89, 205424, 21 May 2014

\bibitem{quantum-transport-reservoirs}
Giulio Amato, Heinz-Peter Breuer, Sandro Wimberger, Alberto Rodríguez, and Andreas Buchleitner
Phys. Rev. A 102, 022207, 11 August 2020.

\bibitem{htheorem2}
R. Silva {\it et al} EPL 89 10004 (2010).

\bibitem{quantum1}
J. Math. Phys. 47, 073303 (2006).

\bibitem{quantum2}
Z. Physik 268, 139-143 (1974).

\bibitem{quantum-entropy}
J. Acharya, I. Issa, N. V. Shende and A. B. Wagner, {\it Measuring Quantum Entropy}, 2019 IEEE International Symposium on Information Theory (ISIT), Paris, France, 2019, pp. 3012-3016

\bibitem{quantum-colapse}
Kastner, R.E. {\it On Quantum Collapse as a Basis for the Second Law of Thermodynamics}, Entropy 2017, 19, 106.



\bibitem{quantum3}
Gring, M. and Kuhnert, M. and Langen, T. and Kitagawa, T. and Rauer, B. and Schreitl, M. and Mazets, I. and Smith, D. Adu and Demler, E. and Schmiedmayer, J.; {\it Relaxation and Prethermalization in an Isolated Quantum System}; 337; 6100; 1318-1322 (2012).

\bibitem{quantum4}
Physical Review E 91, 062106 (2015).

%\bibitem{relaxationphenomena}
%Journal of Statistical Physics, 30; 2 (1983).

%\bibitem{htheorem}
%Review of Modern Physics; 74 (2002).


\bibitem{binary}
Das, Bandita and Biswas, Shyamal; Journal of Statistical Mechanics: Theory and Experiment;2018, 10, 1742-5468. 

\bibitem{contradictions}
Lesovik, G. B. and Lebedev, A. V. and Sadovskyy, I. A. and Suslov, M. V. and Vinokur, V. M., Scientific Reports, Vol. 6, \# 1, 2016.

\bibitem{ergodic-theorem}
Eur. Phys. J. H 35, 201–237 (2010).

\bibitem{hydrodynamic-simulation}
Review of Modern Physics 74(4):1203-1220.




\end{thebibliography}{}




\end{document}

