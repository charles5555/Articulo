%  LaTeX support: latex@mdpi.com 
%  For support, please attach all files needed for compiling as well as the log file, and specify your operating system, LaTeX version, and LaTeX editor.

%=================================================================
\documentclass[entropy,article,submit,moreauthors,pdftex]{Definitions/mdpi} 

% For posting an early version of this manuscript as a preprint, you may use "preprints" as the journal and change "submit" to "accept". The document class line would be, e.g., \documentclass[preprints,article,accept,moreauthors,pdftex]{mdpi}. This is especially recommended for submission to arXiv, where line numbers should be removed before posting. For preprints.org, the editorial staff will make this change immediately prior to posting.

%--------------------
% Class Options:
%--------------------
%----------
% journal
%----------
% Choose between the following MDPI journals:
% acoustics, actuators, addictions, admsci, adolescents, aerospace, agriculture, agriengineering, agronomy, ai, algorithms, allergies, analytica, animals, antibiotics, antibodies, antioxidants, appliedchem, applmech, applmicrobiol, applnano, applsci, arts, asi, atmosphere, atoms, audiolres, automation, axioms, batteries, bdcc, behavsci, beverages, biochem, bioengineering, biologics, biology, biomechanics, biomedicines, biomedinformatics, biomimetics, biomolecules, biophysica, biosensors, biotech, birds, bloods, brainsci, buildings, businesses, cancers, carbon, cardiogenetics, catalysts, cells, ceramics, challenges, chemengineering, chemistry, chemosensors, chemproc, children, civileng, cleantechnol, climate, clinpract, clockssleep, cmd, coatings, colloids, compounds, computation, computers, condensedmatter, conservation, constrmater, cosmetics, crops, cryptography, crystals, curroncol, cyber, dairy, data, dentistry, dermato, dermatopathology, designs, diabetology, diagnostics, digital, disabilities, diseases, diversity, dna, drones, dynamics, earth, ebj, ecologies, econometrics, economies, education, ejihpe, electricity, electrochem, electronicmat, electronics, encyclopedia, endocrines, energies, eng, engproc, entropy, environments, environsciproc, epidemiologia, epigenomes, fermentation, fibers, fire, fishes, fluids, foods, forecasting, forensicsci, forests, fractalfract, fuels, futureinternet, futuretransp, futurepharmacol, futurephys, galaxies, games, gases, gastroent, gastrointestdisord, gels, genealogy, genes, geographies, geohazards, geomatics, geosciences, geotechnics, geriatrics, hazardousmatters, healthcare, hearts, hemato, heritage, highthroughput, histories, horticulturae, humanities, hydrogen, hydrology, hygiene, idr, ijerph, ijfs, ijgi, ijms, ijns, ijtm, ijtpp, immuno, informatics, information, infrastructures, inorganics, insects, instruments, inventions, iot, j, jcdd, jcm, jcp, jcs, jdb, jfb, jfmk, jimaging, jintelligence, jlpea, jmmp, jmp, jmse, jne, jnt, jof, joitmc, jor, journalmedia, jox, jpm, jrfm, jsan, jtaer, jzbg, kidney, land, languages, laws, life, liquids, literature, livers, logistics, lubricants, machines, macromol, magnetism, magnetochemistry, make, marinedrugs, materials, materproc, mathematics, mca, measurements, medicina, medicines, medsci, membranes, metabolites, metals, metrology, micro, microarrays, microbiolres, micromachines, microorganisms, minerals, mining, modelling, molbank, molecules, mps, mti, nanoenergyadv, nanomanufacturing, nanomaterials, ncrna, network, neuroglia, neurolint, neurosci, nitrogen, notspecified, nri, nursrep, nutrients, obesities, oceans, ohbm, onco, oncopathology, optics, oral, organics, osteology, oxygen, parasites, parasitologia, particles, pathogens, pathophysiology, pediatrrep, pharmaceuticals, pharmaceutics, pharmacy, philosophies, photochem, photonics, physchem, physics, physiolsci, plants, plasma, pollutants, polymers, polysaccharides, proceedings, processes, prosthesis, proteomes, psych, psychiatryint, publications, quantumrep, quaternary, qubs, radiation, reactions, recycling, regeneration, religions, remotesensing, reports, reprodmed, resources, risks, robotics, safety, sci, scipharm, sensors, separations, sexes, signals, sinusitis, smartcities, sna, societies, socsci, soilsystems, solids, sports, standards, stats, stresses, surfaces, surgeries, suschem, sustainability, symmetry, systems, taxonomy, technologies, telecom, textiles, thermo, tourismhosp, toxics, toxins, transplantology, traumas, tropicalmed, universe, urbansci, uro, vaccines, vehicles, vetsci, vibration, viruses, vision, water, wevj, women, world 

%---------
% article
%---------
% The default type of manuscript is "article", but can be replaced by: 
% abstract, addendum, article, benchmark, book, bookreview, briefreport, casereport, changes, comment, commentary, communication, conceptpaper, conferenceproceedings, correction, conferencereport, entry, expressionofconcern, extendedabstract, meetingreport, creative, datadescriptor, discussion, editorial, essay, erratum, hypothesis, interestingimages, letter, meetingreport, newbookreceived, obituary, opinion, projectreport, reply, retraction, review, perspective, protocol, shortnote, studyprotocol, systematicreview, supfile, technicalnote, viewpoint
% supfile = supplementary materials

%----------
% submit
%----------
% The class option "submit" will be changed to "accept" by the Editorial Office when the paper is accepted. This will only make changes to the frontpage (e.g., the logo of the journal will get visible), the headings, and the copyright information. Also, line numbering will be removed. Journal info and pagination for accepted papers will also be assigned by the Editorial Office.

%------------------
% moreauthors
%------------------
% If there is only one author the class option oneauthor should be used. Otherwise use the class option moreauthors.

%---------
% pdftex
%---------
% The option pdftex is for use with pdfLaTeX. If eps figures are used, remove the option pdftex and use LaTeX and dvi2pdf.

%=================================================================
%\usepackage{subfigure}
%\usepackage{amsmath} %needed for subequation
\usepackage{verbatim}
%\usepackage[marginal,inner]{showlabels}
%\usepackage{showkeys}
%\usepackage{calrsfs}
%\usepackage{amsthm}
%\DeclareMathAlphabet{\pazocal}{OMS}{zplm}{m}{n}
\newcommand{\commb}[1]{\textcolor{blue}{#1}}
\newcommand{\commr}[1]{\textcolor{red}{#1}}

\newcommand{\NM}{\mathcal{N}_M}
\newcommand{\EM}{\mathcal{E}_M}
\newcommand{\SM}{\mathcal{S}_M}
\newcommand{\muM}{\mu_M}
\newcommand{\fMn}{f_{Mn}}
\newcommand{\epsn}{\varepsilon_{n}}
\newcommand{\sigMn}{\sum_{M}\sum_n}
\newcommand{\Sm}{\mathcal{S}_M}
\newcommand{\Ss}{\mathcal{S}}

\newcommand{\de}{\delta}
\newcommand{\ep}{\epsilon}

\maxdeadcycles=500

\firstpage{1} 
\makeatletter 
\setcounter{page}{\@firstpage} 
\makeatother
\pubvolume{1}
\issuenum{1}
\articlenumber{0}
\pubyear{2021}
\copyrightyear{2020}
%\externaleditor{Academic Editor: Firstname Lastname}
\datereceived{} 
\dateaccepted{} 
\datepublished{}

%------------------------------------------------------------------
% The following line should be uncommented if the LaTeX file is uploaded to arXiv.org
%\pdfoutput=1

%=================================================================
% Add packages and commands here. The following packages are loaded in our class file: fontenc, inputenc, calc, indentfirst, fancyhdr, graphicx, epstopdf, lastpage, ifthen, lineno, float, amsmath, setspace, enumitem, mathpazo, booktabs, titlesec, etoolbox, tabto, xcolor, soul, multirow, microtype, tikz, totcount, changepage, paracol, attrib, upgreek, cleveref, amsthm, hyphenat, natbib, hyperref, footmisc, url, geometry, newfloat, caption

%=================================================================
%% Please use the following mathematics environments: Theorem, Lemma, Corollary, Proposition, Characterization, Property, Problem, Example, ExamplesandDefinitions, Hypothesis, Remark, Definition, Notation, Assumption
%% For proofs, please use the proof environment (the amsthm package is loaded by the MDPI class).

%=================================================================
% Full title of the paper (Capitalized)
\Title{Clasical and Quantum H-Theorem Revisited: Variational Entropy and Relaxation Processes}

% MDPI internal command: Title for citation in the left column
\TitleCitation{Title}

% Author Orchid ID: enter ID or remove command
\newcommand{\orcidauthorA}{0000-0001-8773-4215} % Add \orcidA{} behind the author's name
%\newcommand{\orcidauthorB}{0000-0000-0000-000X} % Add \orcidB{} behind the author's name

% Authors, for the paper (add full first names)
\Author{C. Medel-Portugal$^{1}$, J. M. Solano-Altamirano$^{2}$\orcidA{} and J. L. Carrillo-Estrada$^{1}$*}

% MDPI internal command: Authors, for metadata in PDF
\AuthorNames{Firstname Lastname, Firstname Lastname and Firstname Lastname}

% MDPI internal command: Authors, for citation in the left column
\AuthorCitation{Medel-Portugal, C.; Solano-Altamirano, J.M.; Carrillo-Estrada, J.L.}

% Affiliations / Addresses (Add [1] after \address if there is only one affiliation.)
\address{%
$^{1}$ \quad Instituto de F\'{i}sica, Benem\'erita Universidad Aut\'onoma de Puebla,
Apdo. Postal. J-48, Puebla 72570, M\'exico; cmedel@ifuap.buap.mx (C.M.P.),
carrillo@ifuap.buap.mx (J.L.C.E.) \\
$^{2}$ \quad Facultad de Ciencias Qu\'{\i}micas, Benem\'erita Universidad Aut\'onoma de Puebla, 14 Sur y Av. San Claudio, Col. San Manuel, 72520 Puebla, M\'exico; jmanuel.solano@correo.buap.mx (J.M.S.A.)}

% Contact information of the corresponding author
%\corres{Correspondence: e-mail@e-mail.com; Tel.: (optional; include country code; if there are multiple corresponding authors, add author initials) +xx-xxxx-xxx-xxxx (F.L.)}
\corres{Correspondence: carrillo@ifuap.buap.mx}

% Current address and/or shared authorship
%\secondnote{These authors contributed equally to this work.}
% The commands \thirdnote{} till \eighthnote{} are available for further notes

%\simplesumm{} % Simple summary

%\conference{} % An extended version of a conference paper

% Abstract (Do not insert blank lines, i.e. \\) 
\abstract{%
Starting from a functional, $\mathcal{H}$, defined in terms of non-homogeneous distribution
functions, alternative demonstrations of both classical and quantum H theorems are performed
\textit{via} a variational procedure. This functional describes the time evolution of out of
equilibrium classical or quantum gases, and it satisfies the condition $d\mathcal{H}/dt\leq0$. The
equality condition is attained when the system reaches the thermodynamic equilibrium, and in this
state -$\mathcal{H}$ becomes the entropy of an ideal  gas. Some properties of the functional
$\mathcal{H}$ are investigated, including its correct behavior according to the correspondence
principle. In addition, the relaxation process of quantum gases towards equilibrium is explored. The novelty of the approach presented here is the
application of a variational method to a functional defined in terms of a non-homogeneous
distribution functions, in $\mu$ and position-energy spaces respectively.%
}

% Keywords
%(List three to ten pertinent keywords specific to the article; yet reasonably
%common within the subject discipline.)
\keyword{Non-equilibrium thermodynamics; Entropy; Variational entropy} 

% The fields PACS, MSC, and JEL may be left empty or commented out if not applicable
%\PACS{J0101}
%\MSC{}
%\JEL{}

\begin{document}
%%%%%%%%%%%%%%%%%%%%%%%%%%%%%%%%%%%%%%%%%%
\color{blue}
\section{Introduction}

The theoretical basis and the procedures that allow to describe equilibrium systems
are well-stablished. These procedures can be applied to treat a wide range of the systems
present in nature, and include both the macroscopic phenomenological methods (thermodynamics)
as well as the microscopic description (statistical mechanics).
(Of course, regarding out-of equilibrium systems, the present situation is still a challenge.) 
For instance,
in the Kinetic Theory of gases, the behavior of a dilute classical gas
is described through the Boltzmann transport equation \cite{bib:huang},
and the time-evolution of a system towards the equilibrium is proved by
means of the Boltzmann $H$-theorem, wherein the
spatial-homogeneous distribution function hypothesis plays a crucial role.

On the other hand, for quantum equilibrium systems, the construction of a framework,
with the same level of
success and universality as the classical version, is still an open problem. For instance,
in order to obtain a complete correspondence principle, between classical mechanics
and quantum mechanics, the quantum analogous of both the Boltzmann $H$-theorem
and the Boltzmann transport equation have not attained a fully accepted form.
In this context, Tolman was one of the earliest physicists to propose a quantum
$H$-theorem \cite{bib:tolman}, using probability transition relations, the random
phases hypothesis, and a spatially homogeneous distribution function.
Also, Tolman proposed a potential quantum analogous of the transport equation,
in terms of the occupation number,
by applying time perturbation theory. Further attempts, under the quantum operator
formalism, have addressed the description of quantum transport phenomena by means of the Hamiltonian
of the system and the master equation (which here is the analogous
of the Boltzmann transport equation)
\cite{bib:grabert1974,bib:wang2014,%
bib:angel2017,
bib:amato2020,bib:nicacio2015,bib:hussein2014}. However, these approaches 
are not consistent with the classical-quantum correspondence principle.
In the same line of thought, some authors have proposed $H$-functionals
and attempted to proof a
quantum $H$-theorem \cite{bib:gorban2014,bib:bennaim2017,bib:silva2010,%
bib:deroeck2006,bib:acharya2019,bib:kastner2017,bib:han2015,bib:das2018,bib:vonneumann2010}.
However, whether or not the homogeneous distribution function hypothesis is assumed, or
if their framework obeys the correspondence principle is unclear or not discussed whatsoever.
Furthermore, the general validity
of the quantum $H$-theorem, the second law of thermodynamics, and the
interpretation of the quantum entropy is still in discussion %[19-26].
\cite{bib:silva2010,bib:acharya2019,bib:kastner2017,bib:han2015,bib:brown2008,%
bib:dragoljub2009,bib:vonneumann2010,%
bib:syros1999,bib:lesovik2016,
bib:lesovik2019}.

On the other hand, the framework to describe spatially non-homogeneous systems is still
under construction, although several approaches have been developed. For instance,
the celebrated Onsager formulation (\textit{aka} linear thermodynamics)
\cite{bib:keizer1987,bib:onsager1931} have been successful in describing irreversible chemical 
and physical phenomena. Nonetheless, some aspects such as describing the internal behavior of glasses
\cite{bib:zanotto2018} and the entropy measurement
\cite{bib:schmelzer2018,bib:nemilov2018} cannot be completely addressed with linear thermodynamics.

In addition, some aspects regarding the classical $H$-theorem and the
Boltzmann $H$-functional require to be revised, for the purpose of improving their mutual consistency.
For example, the modification of the $H$-theorem to include phenomena stemming from stochastic
trajectories, violations of the second law of thermodynamics, the relation
between Shannon's measure of information and the Boltzmann's entropy, 
and the calculation of thermodynamical quantities and thermalization of certain systems
\cite{bib:gorban2014,bib:li2019,bib:gring2012,bib:nemilov2018,bib:wang2014}.

In order to contribute to the construction of a consistent classical and
quantum $H$-theorem, within a formalism that describes out-of-equilibrium
non-homogeneous systems, in this article we propose a new theoretical framework.
Specifically, for both classical and quantum systems, we include non-homogeneous distribution functions in
the $H$-functionals, and consider non-homogeneous systems in the proofs of the resulting
$H$-theorems. Our proposed $H$-functionals obey
the correspondence principle, but more importantly, these functionals describe
the time-evolution of spatially non-homogeneous
systems towards equilibrium.

The organization of this article is as follows.
In Section~\ref{sec:classical}, we highlight, to our purposes, the most
fundamental assumptions required to proof the Boltzmann $H$-theorem,
we provide an alternative method to obtain the Maxwell-Boltzmann
distribution using the variational method, followed by our proposal of
an alternative $H$-functional for classical systems, and the demonstration of
the respective $H$-theorem . In Section~\ref{sec:quantum}
we review the Tolman proposal for the quantum
version of the $H$-theorem (quantum $H$-theorem)
and how the Bose-Einstein and Fermi-Dirac distributions are treated within
this framework, subsequently we present our proposal for a quantum $H$-functional
and the proof of the corresponding quantum $H$-theorem. In 
Section~\ref{sec:qccorrespondence}
the classical-quantum correspondence between the
quantum and classical $H$-functionals is analyzed.
In Section~\ref{sec:relaxproc} we explore how relaxation processes occur
in a quantum ideal gas; \commb{here we provide a time-evolution equation that could be
considered the analogous of the classical Boltzmann transport equation.} Finally,
we discuss some key ideas derived from our approach and close with a brief summary
in Section~\ref{sec:disscussion}.



\end{paracol}

\color{black}
\begin{paracol}{2}
\switchcolumn\linenumbers
%\section{Classical}
%\subsection{Boltzmann transport equation}
%\subsection{H theorem and the Maxwell-Boltzmann distribution}
%Todo esto de arriba en la forma tradicional y muy resumido. Solo remarcando lo fundamental.
%\subsection{Out of equilibrium, non-homogeneous distributions}.
%Entrar a la generalizacion, ahi si de manera muy detallada y con amplitud discutida
%fisicamente, del tratamiento con las celdas. Es decir, la demostracion del teorema H
%clasico pero con el enfoque de funciones de distribucion no homogeneas y por tanto
%con el tratamiento de celdas.

%----------------------------------

\section{Classical scheme}

\subsection{The Boltzmann transport equation}

In the classical version, the Boltzmann transport equation is required to show
the $H$-theorem. That equation is
%
\begin{equation}\label{eq:transport}
  \left(
    \frac{\partial}{\partial t}+\vec{v}_1 \cdot\nabla_r
    +\frac{\vec{F}}{m}\cdot\nabla_{v_1}
  \right)f_1=\int\mathrm{d}\Omega\int
    \mathrm{d}^{3}v_2\sigma(\Omega)|\vec{v}_1-\vec{v}_2|(f_2'f_1'-f_2f_1),
\end{equation}
%
where $\Omega$ is the solid angle, $\sigma$ is the scattering section, $F$ the
external force applied to the system, $f_1$, $f_2$ are the distribution
functions of the particles 1 and 2 respectively before the collision,
and $f'_1$, $f'_2$ are the distribution function of the particles 1 and 2
respectively after the collision.

On the one hand, dynamics, inhomogeneities, and external forces are described
by the left-hand side of the equation (\ref{eq:transport}). On the other hand, the
binary collisions among particles are involved in the right-hand side of the
previous equation. However, the \textit{molecular chaos hypothesis} was applied
to obtain the right-hand side of (\ref{eq:transport}). The \textit{molecular chaos
hypothesis} establishes that spatial and velocity coordinates of the particles
are not correlated.

%-------------------------
\subsection{H-theorem and the Maxwell-Boltzmann distribution}
The $H$ theorem can be proved using the $H$ functional defined by Boltzmann 
%
\begin{equation}\label{eq:hfunctional}
   H=\int f(\vec{v},t) \ln f(\vec{v},t) \mathrm{d}^{3}v,
\end{equation} 
%
where $f(\vec{v},t)$ is a homogeneous distribution function.
This $H$ functional describes a system with a global energy $E$, a global
number of free classical particles $N$, global volume $V$, and a temperature $T$.
Besides, the Boltzmann's $H$ functional correspond to a density of entropy
when the system arrives at the equilibrium state.

The $H$ theorem establishes that if $f(\vec{v},t)$ satisfies the Boltzmann
transport equation and the molecular chaos hypothesis, then
$dH/dt\leq0$ and the system arrives to the equilibrium state when
$dH/dt=0$.
The theorem can be proved by applying the time derivative
to Eq. (\ref{eq:hfunctional}) and using the Boltzmann transport equation
(\ref{eq:transport}). As a consequence of this theorem, the distribution function
that describes the system in equilibrium corresponds to the Maxwell-Boltzmann
distribution. This distribution can be obtained by solving a sufficient
condition for $f$ to obtain the equilibrium distribution function $f_0$:
%
\begin{equation}
    f_0(\vec{v}'_2)f_0(\vec{v}'_1)- f_0(\vec{v}_2)f_0(\vec{v}_1)=0.
\end{equation}
%

This condition can be obtained from the binary collision term of the Boltzmann
transport equation (\ref{eq:transport}).

The $H$-theorem fixes the time evolution direction of systems in nature, such
as the second law of thermodynamics does. {\color{blue}footnote: }The molecular hypothesis is %
fundamental to determine the physical meaning of the $H$ theorem. For more %
information, see \cite{bib:huang}.

%---------------------------
\subsection{Out of equilibrium, non-homogeneous distributions}

The Boltzmann $H$-functional (\ref{eq:hfunctional}) is defined with a spatially
homogeneous distribution $f(\vec{v},t)$. However, we want to consider a $H$
functional with a non-homogeneous distribution function in order to generalize
the Boltzmann $H$ theorem. Because of this, we introduce a modified $H$
functional (denoted as $H'$) defined as
%
\begin{equation}\label{eq:CH2}
   H'(t)=\sum_{M=1}^{K}\int f_M(\vec{v},t) \ln f_M(\vec{v},t)\mathrm{d}^3v.
\end{equation}
%
This $H'$ describes a system with a global energy $E$, a global number of free
classical particles $N$, global volume $V$, and a temperature $T$. Nevertheless,
the global system, in turn, consists of a set of $K$ small cells with a constant
volume $\delta V_M$, whose local homogeneous distribution functions is
$f_{M}(\vec{v},t)$.

The distribution function will depend on the position of the cells, the velocity
$\vec{v}$, and time $t$. Without loss of generality, we choose these cells to
have local identical volumes (\textit{i.e.} $\delta V_M = V/K, \forall M$).
Each cell has its local chemical potential, $\mu_M$, local
$H$ functional, $H'_M$, local number of particles, $\mathcal{N}'_M$, and local
energy, $\mathcal{E}'_M$. In terms of the local distribution 
function and the energy $\epsilon(\vec{v})$, the last three properties are
given by:
%
\begin{eqnarray}
    H_M' & = &  \int f_M(\vec{v},t) \ln f_{M}(\vec{v},t)
      \mathrm{d}^{3}v \label{Hcell},\nonumber\\
    \mathcal{N}_M' & = & \int f_{M}(\vec{v} ,t) \mathrm{d}^{3}v, \nonumber\\
    \mathcal{E}_M' & = & \int f_{M}(\vec{v},t)\epsilon(\vec{v}) \mathrm{d}^{3}v.
\end{eqnarray}
%

When the system is in equilibrium, the local number of particles and the local
energy do not depend on the cell number, this is
%
\begin{equation}
   \mathcal{N}'_M=\mathcal{N}'\equiv \bar{\mathcal{N}}';
   \quad\mathcal{E}'_M=\mathcal{E}'\equiv\bar{\mathcal{E}}'.
\end{equation}
%
The particles in the system are considered to be free, \textit{i.e.} the
particles do not interact with each other. Besides, the local distribution
function must satisfy the following microcanonical restrictions 
%
\begin{equation}\label{micro}
    \sum_{M=1}^{K}\int f_M(\vec{v},t)\mathrm{d}^3v =N,
    \quad\sum_{M=1}^{K}\int f_M(\vec{v},t)\epsilon(\vec{v})\mathrm{d}^3v=E.
\end{equation}
%

We can see that the functional (\ref{eq:CH2}) contains a non-homogeneous
distribution function but keeps the same form of the Boltzmann's $H$ functional.

Inspired by the Hamilton principle, $H'$ will be maximized with those
restrictions to obtain the equilibrium distribution function for a classical gas
using the Lagrange multipliers method obtaining
%
\begin{eqnarray}
    \frac{\delta H'}{\delta f_J(\vec{v}')} & = & \sum_{M=1}^{K}\int
      \frac{\delta}{\delta f_J(\vec{v}')}\left[
        f_M(\vec{v})\ln f_M(\vec{v})
        \right]
       \mathrm{d}^3v -\sum_{M=1}^{K}\alpha_M\int
       	\frac{\delta f_M(\vec{v})}{\delta f_J(\vec{v}')}
      \mathrm{d}^3v\nonumber\\
    & & -\sum_{M=1}^{K}\beta_M\int\epsilon(\vec{v})
    	\frac{\delta f_M(\vec{v})}{\delta f_J(\vec{v}')}
      \mathrm{d}^3v \nonumber\\
    & = & \ln f_J(\vec{v}')+1-\alpha_J-\beta_J \epsilon(\vec{v}')=0.
\end{eqnarray}
%
Isolating the distribution function 
%
\begin{eqnarray}
    \ln f_J(\vec{v}') & = & \alpha_J+\beta_J \epsilon(\vec{v}')
       -1\quad\Rightarrow\quad f_J(\vec{v}')\\
    & = & e^{\alpha_J +\beta_J \epsilon(\vec{v}')-1}\nonumber\\
    & = & Ce^{\alpha_J+\beta_J \epsilon(\vec{v}') } \label{relacion1},
\end{eqnarray}
%
where $C$ is a constant. It is trivial that using the variational procedures we
obtain the form of the Maxwell-Boltzmann distribution. 

As $H'$ in equilibrium is proportional to entropy, it is trivial to think that
Lagrange multipliers do not depend on the position because of the system is in
equilibrium and consequently, the distribution function is homogeneous, that is
%
\begin{equation}
    f(\vec{v})=Ce^{\alpha+\beta \epsilon(\vec{v})}\equiv \bar{f}(\vec{v}).
\end{equation} 
%
The constant $C$ can be omitted defining the following $H$ functional
%
\begin{equation}
   H''(t)=\sum_{M=1}^{K}\int \left[f_M(\vec{v},t)
    \ln f_M(\vec{v},t)-f_M(\vec{v},t)\right]\mathrm{d}^3v  \label{CH3},
\end{equation}
%
but $\sum_{M=1}^{K} \int f_M(\vec{v},t)\mathrm{d}^3v =N$ where $N$ is the
total particle number. As $N$ is a constant and any $H$ functional has sense
when we calculate its time derivative, this constant can be omitted. 

On the other hand, if the distribution function is spatial-homogeneous, $f_M$ does not depend on the cell number $M$, then we rewrite (\ref{eq:CH2}) as
%
\begin{equation}
    H'(t)=\int \sum_{M=1}^{K} [f(\vec{v},t)\ln f(\vec{v},t)] \mathrm{d}^{3}v
         = K\int  f(\vec{v},t) \ln f(\vec{v},t) \mathrm{d}^{3}v= K H_{boltz}(t), 
\end{equation}
%
where
%
\begin{equation}
    H_{boltz}(t)=\int f(\vec{v},t) \ln f(\vec{v},t) \mathrm{d}^{3}v.
\end{equation}
%
Besides, if we analyze the following expression
%
\begin{equation}
    \int Kf(\vec{v},t) \ln [Kf(\vec{v},t)] \mathrm{d}^{3}v = \int [(K\ln K)f(\vec{v},t) + Kf(\vec{v},t) \ln f(\vec{v},t)]\mathrm{d}^{3}v, \label{sum-h}
\end{equation}
%
we can observe if the number of particles in the $\mu$-space $f(\vec{v},t)$ is to big compare with the number of cells $P$, then the first term in (\ref{sum-h}) is negligible, and consequently, we conclude that
%
\begin{equation}
    \int f'(\vec{v},t) \ln f'(\vec{v},t) \mathrm{d}^{3}v = K\int f(\vec{v},t) \ln f(\vec{v},t) \mathrm{d}^{3}v; \ \ \ f'(\vec{v},t)= Kf(\vec{v},t). \label{aditive-property} 
\end{equation}
%
We obtained the additive property when we supposed the distribution function is
spatial-homogeneous.


To prove the Boltzmann's $H$ theorem (using (\ref{eq:CH2}) as functional), we need
to assume that the distribution function $f_M(\vec{v},t)$ must satisfy the
following assumptions:
%
\begin{itemize}
  \item The local equilibrium hypothesis in each cell ($f_{M}$ must be
    spatial-homogeneous for each $M$).
  \item The non-homogeneous distribution function assumption in the total
    volume ($f_{M}$ must be different among all cells).
  \item The following restrictions
    %
    \begin{equation}\label{restrictionoutsideclassical}
      \int f_{M}(\vec{v},t) \mathrm{d}^{3}v=
      \bar{\mathcal{N}}+\Delta_M(t); \ \ \ \ 
      \int f_{M}(\vec{v},t) \epsilon(\vec{v}) \mathrm{d}^{3}v=
      \bar{\mathcal{E}}+ \delta_M(t),
    \end{equation}
    %
    where $\bar {\mathcal{N}}$ and $\bar{\mathcal{E}}$ are
    the local particle number and the local energy in equilibrium
    %
    \begin{equation}
      \bar{\mathcal{N}}=
      \int \bar{f}(\vec{v}) \mathrm{d}^{3}v ; \quad
      \bar{\mathcal{E}}=
      \int \bar{f}(\vec{v})\epsilon(\vec{v}) \mathrm{d}^{3}v,
    \end{equation}
    %
    and $\Delta_M$, $\delta_M$ could be seen as a deviation from
    $\bar{\mathcal{N}}$ and $\bar{\mathcal{E}}$ respectively,
    with $\Delta_M(t)\ll\bar{\mathcal{N}}$ and
    $\delta_M(t)\ll\bar{\mathcal{E}}$.
\end{itemize} 
%
Those conditions described a system out of equilibrium but no so far from it,
that is a perturbed system and will suffer a \textit{relaxation process, i.e.}
The time evolution of a perturbated system to the equilibrium state.

It is important to remark that the quantities $\Delta_M$, $\delta_M$ are
sufficiently big to be different from fluctuations in the system but too small
such that the system is not far from the equilibrium state.  

Performing the derivative of (\ref{eq:CH2}) with respect to time, we obtain
%
\begin{equation}\label{dH1}
    \frac{dH'}{dt}=\sum_{M=1}^{K}\int\left[
      1+\ln f_M(\vec{v},t)
    \right]\dot f_M(\vec{v},t)\mathrm{d}^3v.
\end{equation}
%
Using the first-order approximation
%
\begin{equation}\label{firstorder}
   f_{Mn}(\vec{v},t)=\bar{f}_{n}(\vec{v})(1+g_{Mn}(\vec{v},t)): \ \ \ 
   \bar{f}_{n}(\vec{v})\gg \bar{f}_{n}(\vec{v})|g_{Mn}(\vec{v},t)|,
\end{equation}
%
(\ref{dH1}) yields
%
\begin{equation}\label{dH1-1}
    \frac{dH'}{dt}=\sum_{M=1}^{K}\int\bar f(\vec{v}) \left[
      1+\ln \left\{
        \bar f(\vec{v})+\bar f(\vec{v})g_M(\vec{v},t)
      \right\}
    \right]\dot g_M(\vec{v},t)\mathrm{d}^3v.
\end{equation}
%

We can approximate the logarithm function in (\ref{dH1-1}) to its first-order
Taylor series around $\bar f_n(\vec{v}) g_{nM}(\vec{v},t)=0$ 
%
\begin{equation}\label{lnapproximationclassical}
    \ln [\bar{f}(\vec{v})+\bar{f}(\vec{v}) g_{M}(\vec{v},t)] \approx
    \ln [\bar{f}(\vec{v})]+ g_{M}(\vec{v},t).
\end{equation}
%
With the previous approximation, (\ref{dH1-1}) obtains the following form 
%
\begin{eqnarray}
    \frac{dH'}{dt}&=&\sum_{M=1}^{K} \int \bar f(\vec{v})\left[
      1+\ln \bar f(\vec{v})+g_M(\vec{v},t)
    \right]\dot g_M(\vec{v},t)\mathrm{d}^3v,
\end{eqnarray}
%
using (\ref{relacion1}) with homogeneous multipliers
%
\begin{equation}\label{dH1-2}
    \frac{dH'}{dt} = \sum_{M=1}^{K}\int\bar f(\vec{v})\left[
      \alpha+\beta \epsilon(\vec{v})
    \right]\dot g_M(\vec{v},t)\mathrm{d}^3v +\sum_{M=1}^{K}
    \int\bar f(\vec{v})g_M(\vec{v},t)\dot g_M(\vec{v},t)\mathrm{d}^3v . \nonumber 
\end{equation}
%
On the other hand, we obtain from the restrictions the following expressions
%
\begin{eqnarray}
    \int \bar{f}(\vec{v}) g_{M}(\vec{v},t) \mathrm{d}^{3}v=\Delta_M(t) \ \  &\Rightarrow&
    \ \  \int \bar{f}(\vec{v}) \dot{g}_{M}(\vec{v},t)\mathrm{d}^{3}v=\dot{\Delta}_M(t), \nonumber \\
    \int  \bar{f}(\vec{v}) g_{M}(\vec{v},t)\epsilon(\vec{v}) \mathrm{d}^{3}v=\delta_M(t) \ \  &\Rightarrow&
    \ \  \int \bar{f}(\vec{v}) \dot{g}_{M}(\vec{v},t)\epsilon(\vec{v}) \mathrm{d}^{3}v=\dot{\delta}_M(t), \nonumber 
\end{eqnarray}
%
and as a consequence of $\sum_{M=1}^{K} \Delta_M(t) =\sum_{M=1}^{K} \delta_M(t) =0$,
we find
%
\begin{equation}
    \sum_{M=1}^{K} \dot{\Delta}_M(t)  =\sum_{M=1}^{K} \dot{\delta}_{M}(t) =0.
\end{equation}
%
Then using the previous result to (\ref{dH1-2}), we find
%
\begin{equation}
  \frac{dH'}{dt} = \sum_{M=1}^{K}\int\bar f(\vec{v})g_M(\vec{v},t)\dot g_M(\vec{v},t)\mathrm{d}^3v.
\end{equation}
%
Now, the summation over $M$ will be expressed in two summations
%
\begin{equation}\label{classicalH3}
    \frac{dH'}{dt}=\sum_J^{L}\int
      \bar f(\vec{v})g_J^{+}(\vec{v},t)\dot g_J^{+}(\vec{v},t)\mathrm{d}^3v
      +\sum_J^{P}\int
        \bar f(\vec{v})g_J^{-}(\vec{v},t)\dot g_J^{-}(\vec{v},t)\mathrm{d}^3v.
\end{equation}
%
where $L+P=K$. The group of $L$ cells is such that cells have an excess of
particles or/and energy (from the mean value in equilibrium). In contrast,
the group of $P$ cells is such that cells have to miss particles or/and energy
(from the mean value in equilibrium). Besides, $\dot{g}^{+}_{J}$ represents the
change on the deviation on cells that have an excess of particles or energy
while $\dot{g}^{-}_{J}$  represents the change on the deviations on cells that
have missing particles or energy. 
On the other hand, $g^{+}_{J}$  represents the value of the deviation on cells
that have an excess of particles or energy. In contrast, $g^{-}_{J}$ represents
the value of the deviation on cells that have missing particles or energy.

Also, on the one hand, $\dot{g}^{+}_{J}<0$ describes the loss of particles
and/or energy and so, $g^{+}_{J}>0$. On the other hand, $\dot{g}^{-}_{J}>0$
describes the gain of particles and/or energy and therefore $g^{-}_{J}<0$. 

We sort the previous ideas in the following form
%
\begin{equation}\label{separacionclassical}
\begin{array}{rl}
  g^{+}_{J}=+|g^{+}_{J}|; & \dot{g}^{+}_{J}=-|\dot{g}^{+}_{J}|,\\
  g^{-}_{J}=-|g^{-}_{J}|; & \dot{g}^{-}_{J}=+|\dot{g}^{-}_{J}|,
 \end{array}
\end{equation}
%
and consequently, (\ref{classicalH3}) obtains the following form
%
\begin{equation}
    \frac{dH'}{dt}=-\left[
      \sum_J^{L}\int\bar f(\vec{v})|g_J^{+}(\vec{v},t)|
        |\dot g_J^{+}(\vec{v},t)|\mathrm{d}^3v
      +\sum_J^{P}\int\bar f(\vec{v})|g_J^{-}(\vec{v},t)|
      |\dot g_J^{-}(\vec{v},t)|\mathrm{d}^3v 
    \right]. \label{classicalH4}
\end{equation}
%
We can observe that $\bar{f}(\vec{v})$ in (\ref{classicalH4}) is always
positive. Also the deviation and its derivative are positive, then all the
expression is positive. However, the global sign makes the derivative of the
$H'$ with respect to time is always less to zero. With this we proved that
$\frac{dH'}{dt}<0$. If the system is in equilibrium,
$g_{J}(\vec{v},t)=\dot g_J(\vec{v},t)=0$, therefore $\frac{dH'}{dt}=0$.

Joining both results, we can say the following statement:
Consider a classical gas in a total volume $V$ (divided in $K$ cells of equal
volume elements), total energy $E$ and total number of free particles $N$.
Consider also the system has inhomogeneities and suffers a relaxation process.
If we define the following functional
%
\begin{equation}\label{CH3}
   H'(t)=\sum_{M=1}^{K}\int f_M(\vec{v},t) \ln f_M(\vec{v},t)\mathrm{d}^3v,
\end{equation}
%
where $f_M(\vec{v},t)$ is the local distribution function of each cell in the
system, then in the first-order approximation
%
\begin{equation}
    \frac{dH'}{dt} \leq 0.
\end{equation}
%
This statement will be the classical $H$ theorem with inhomogeneities.

In the next section, we present a proposal quantum version of the
$H$ theorem defined by Tolman and our proposal quantum version of the $H$
theorem applying the method of the volume divided into cells.



\end{paracol}

\begin{paracol}{2}
\switchcolumn\linenumbers
\section{Quantum scheme}\label{sec:quantum}

The classical $H$-theorem is even to these days one of the pillars whereon the classical statistical
physics is founded. Unfortunately, regardless of several attempts have been made \cite{bib:silva2010,bib:deroeck2006,bib:grabert1974,bib:han2015,bib:das2018,bib:vonneumann2010}, 
the generality of the classical $H$-theorem has no an equally robust quantum match. In this section, we shall
propose and analize an alternative quantum $H$-functional using the variational method.
We start by briefly outline a typical textbook demonstration of the quantum $H$-theorem
\cite{bib:tolman}, then, we present the analysis of our proposed $H$-functional.

\subsection{$H$-theorem and the Fermi-Dirac and Bose-Einstein distribution functions}

Consider a diluted gas of $N$ non-interacting quantum particles (either bosons or fermions),
contained by a vessel of volume
$V$, at some temperature $T$, and total energy E. Starting from the Boltzmann definition
of entropy, the quantum $H$ functional is
%
\begin{equation}
   H_T=-\ln G,
\end{equation}
%
where $G$ describes the total number of accessible quantum states of the gas that satisfy the
above mentioned conditions \cite{bib:tolman}.
The quantum $H$-theorem can be demostrated as follows.  $G$ can be divided into groups of neighboring states,
$g_k$, and a certain occupation numbers, $n_k$, can be associated with each of these
groups. Thus, the above functional takes the form:
%
\begin{equation}\label{eq:quantumh}
    H_T=\sum_i n_i \ln n_i -(n_i\pm g_i)\ln (g_i \pm n_i)\pm g_i\ln g_i, 
\end{equation}
%
where the upper and lower signs are for bosons and fermions, respectively.
Thus the time derivative of Eq.~(\ref{eq:quantumh}) is
%
\begin{equation}\label{eq:quantumdHdt}
\frac{dH_T}{dt}=\sum_{\kappa}\left[\ln n_{\kappa}-\ln\left(g_{\kappa}\pm n_{\kappa}\right)\right]
\frac{dn_{\kappa}}{dt}.
\end{equation}
%

Assuming that the energy exchange between particles is produced by collisions between particles.
By using perturbation theory, the rate of change in the number of particles in a group $\kappa$ is
%
\begin{eqnarray}\label{eq:changen}
    \frac{d n_{\kappa}}{dt}&=&-\sum_{\lambda,(\mu \nu)}A_{\kappa\lambda,\mu\nu} n_{\kappa}n_{\lambda}(g_{\mu}\pm n_{\mu})(g_{\nu}\pm n_{\nu})\nonumber \\
    &&+\sum_{\lambda,(\mu \nu)}A_{\mu\nu,\kappa\lambda} n_{\mu}n_{\nu}(g_{\kappa}\pm n_{\kappa})(g_{\lambda}\pm n_{\lambda}).
\end{eqnarray}
%
Here $A_{\kappa\lambda,\mu\nu}n_{\kappa}n_{\lambda}(g_{\mu}\pm n_{\mu})(g_{\nu}\pm n_{\nu})$ is the
expected number of collisions per unit time, in which two particles will be moved from groups
($\kappa$, $\lambda$) to ($\mu$, $\nu$), and the tensor $A_{\kappa\lambda,\mu\nu}$ is given by
%
\begin{equation}\label{eq:amnkldef}
  A_{\kappa\lambda,\mu\nu}=\frac{4\pi^{2}}{h}\frac{|I_1\pm I_2|^2}{\Delta \epsilon}.
\end{equation}
%
In Eq.~(\ref{eq:amnkldef}), $\Delta\epsilon$ is the net energy change occuring during the
collision and $|I_1-I_2|^2=|V_{mn,kl}|^2$ where
$V_{mn,kl}$ is the element of the transition matrix of a binary collision.
It is important to remark that in deriving Eq.~(\ref{eq:changen}), the equal a priori
probabilities and the random a priori phases hypotheses were taken valid. The 
the random a priori phases hypothesis can be
considered an analogous to the molecular chaos hypothesis
\cite{bib:das2018} as the mechanism to introduce stochasticity into the system.

Substituting Eq.~(\ref{eq:changen}) into Eq.~(\ref{eq:quantumdHdt}) after some algebra, it can be proven that
%
\begin{equation}
    \frac{dH_T}{dt}\leq 0. \label{eq:H-theorem-tolman}
\end{equation}
%

In equilibrium (at $t\to\infty$), $dn_{\kappa}/dt=0$, hence from Eq.~(\ref{eq:changen})
%
\begin{equation}\label{eq:boseiferdirdistfnctsrc}
    \ln \frac{n_{\kappa}}{g_{\kappa}\pm n_{\kappa}}+\ln \frac{n_{\lambda}}{g_{\lambda}\pm n_{\lambda}}=\ln \frac{n_{\mu}}{g_{\mu}\pm n_{\nu}}+\ln \frac{n_{\nu}}{g_{\nu}\pm n_{\nu}}.
\end{equation}
%
Considering that the energy is conserved during the collsion, the Bose-Einstein or Fermi-Dirac distribution
functions can be recovered from Eq.~(\ref{eq:boseiferdirdistfnctsrc}):
%
\begin{equation}
   n_{\kappa}=\frac{g_{\kappa}}{\exp(\alpha+\beta\epsilon_{\kappa})\mp1}.
\end{equation}
%

\textit{I.e.}, in equilibrium $dH_T/dt=0$ and the distribution function obtained from
Eq.~(\ref{eq:quantumh}) is the expected distribution function.

%----------------------------------------------
\begin{comment}
In addition, Tolman also showed that (\ref{eq:quantumh}) can be reduced, in the
high-energy limit, to
%
\begin{equation}
    H_T = \sum_{\kappa} (n_{\kappa} \ln n_{\kappa} - n_{\kappa} \ln g_{\kappa}). \label{reduce-h}
\end{equation}
%
This expression also can be obtained also from the Boltzmann $H$-functional through defining
%
\begin{equation}
    f=\frac{n_{\kappa}}{ g_{\kappa}}.
\end{equation}
\end{comment}
%
%--------------------------



\subsection{\textcolor{red}{Out of} equilibrium, non-homogeneous quantum systems}

Consider a diluted gas enclosed by a perfectly isolated vessel of volume $V$, with
total energy $E$, and total number of quantum particles $N$, which can be free fermions or
bosons. For our purposes, the volume $V$ is divided into $K$ small cells, each of which
has a constant volume $\delta V_M=V/K$ ($M=1,\,\dots\,,K$), a temperature $T_M$, an energy $\epsilon_M$,
a number of particles $\mathcal{N}_M$, and a distribution function,
$\{f_{Mn}(t)\}$. Hereafter we will use a short-hand notation:
%
\begin{equation}
   f_{Mn}(t)\equiv f(\vec r_M,\epsilon_{n},t),
\end{equation}
%
where $\vec r_M$ is the radius vector pointing at the center of the $M$-th cell.
$f_{Mn}(t)$ represent the number
of particles contained in the $M$-th cell that occupy the energy level $\epsilon_n$ at
time $t$. Since the particles are considered to be free, the energy levels should not
depend on the cell properties, \textit{i.e.} the energy spectrum, $\{\epsilon_n\}$, is the same for all
cells; thus there is no need to label $\epsilon_n$ with an index $M$.

We propose the following functional as an alternative $H$-functional
for quantum non-homogeneous diluted gases:
%
\begin{eqnarray}\label{eq:qHdef}
    \mathcal{H} (t)&=&\sum_{M=1}^{K} \sum_{n} \bigg[ f_{Mn}(t) \ln f_{Mn}(t)\bigg.\nonumber \\
    &&\qquad\qquad\bigg.\pm \Big(1 \mp f_{Mn}(t)) \ln (1 \mp f_{Mn}(t)\Big) \Big]   \delta V_M.
\end{eqnarray}
%
Here, the upper and lower sign refer to fermions and bosons respectively.
In addition, when needed, each cell will have an associated local chemical potential,
$\alpha_M$, and a local $H$-functional, which is defined by:
%
\begin{equation}\label{eq:qHMdef}
   \mathcal{H}_M(t)=\sum_{n} \bigg[ f_{Mn}(t) \ln f_{Mn}(t)
   \pm \Big(1 \mp f_{Mn}(t)\Big) \ln \Big(1 \mp f_{Mn}(t)\Big) \bigg] \de V_M.
\end{equation}
%
Therefore, $\mathcal{N}_M$ and $\mathcal{E}_M$ as functions of time are given by:
%
\begin{subequations}
\begin{equation}
    {\mathcal{N}}_M(t) = \sum_{n}f_{Mn}(t) \delta V_M
\end{equation}
%
and
%
\begin{equation}
\mathcal{E}_M(t) = \sum_{n}f_{Mn}(t)\epsilon_{n} \de V_M,
\end{equation}
\end{subequations}
%
which are, for the whole system, constrained by the microcanonical restrictions
%
\begin{subequations}\label{eq:qGlobRest}
\begin{equation}\label{eq:qGlobRestN}
    \sum_{M=1}^{K}\left[\sum_{n}f_{Mn}(t)\right]\delta  V_M
    =\sum_{M=1}^{K}\mathcal{N}_{M}(t)\delta V_M=N,
\end{equation}
%
and
%
\begin{equation}\label{eq:qGlobRestE}
    \sum_{M=1}^{K}\left[\sum_{n}f_{Mn}(t)\epsilon_{n}\right]\delta V_M
    =\sum_{M=1}^{K}\mathcal{E}_M(t)\delta V_M=E. 
\end{equation}
\end{subequations}
%

Applying the variational method to $\mathcal{H}$, and using the
Lagrange multipliers
$\{\alpha_M\}$ and $\{\beta_M\}$, it is straightforward to obtain
(see also the discussion of Eq.~(\ref{eq:deltaHpdeltafpj})):
%
\begin{equation}\label{eq:relation}
\ln \left(\frac{1\mp f_{Mn}(t)}{f_{Mn}(t)} \right)=-\alpha_M(t)-\beta_M(t) \epsilon_{n},
\end{equation}
%
and solving for $f_{Mn}(t)$ yields
%
\begin{equation}\label{eq:qfMn}
f_{Mn}(t)=\frac{1}{\exp\big(-\alpha_M(t)-\beta_M(t) \epsilon_{n}\big)\pm 1}.
\end{equation}
%
Thus, in this zero-order approximation the form of equilibrium distribution functions are conserved.

%============================================================
\subsubsection{Properties of $\mathcal{H}$ for systems in equilibrium}
%============================================================
If the system is in equilibrium (we will show that it will also be valid at $t\to\infty$), the temperature becomes homogeneous throughout
the complete system. Also, the local number of particles, the local
energy, and the Lagrange multipliers do not depend on the cell number and they should be homogeneous,
this is 
%
\begin{subequations}\label{eq:qEqRestrictions}
\begin{equation}
   {\mathcal{N}}_M(t\to\infty)\equiv \bar{\mathcal{N}}=N/K,
\end{equation}
%
%
\begin{equation}
	{\mathcal{E}}_M(t\to\infty)\equiv \bar{\mathcal{E}}=E/K,
\end{equation}
\end{subequations}
%
%
\begin{subequations}\label{eq:qEqAlphaBeta}
\begin{equation}
	\alpha_M=\bar\alpha,\quad\forall M,
\end{equation}
%
and
\begin{equation}
	\beta_M=\bar\beta, \quad\forall M.
\end{equation}
\end{subequations}
%
Substituting Eqs.~(\ref{eq:qEqAlphaBeta}) into Eq.~(\ref{eq:qfMn}) yields the distribution
function of each cell in equilibrium to be
%
\begin{equation}\label{eq:qfneq}
    \bar f_{Mn}=\bar f_n =\frac{1}{\exp\big(-\bar\alpha-\bar\beta \epsilon_n\big)\pm 1},\quad\forall M.
\end{equation}
%
By means of the above equation, we can recover the distribution function and the entropy of a diluted quantum gas in equilibrium as follows. Setting $\bar\alpha=\mu/kT$ and $\bar\beta=-1/kT$,
and substituting them in Eq.~(\ref{eq:qfneq}), it renders the Fermi-Dirac and Bose-Einstein
distribution functions:
%
\begin{equation}\label{eq:qfneqtwo}
    \bar{f}_{n}=\frac{1}{\exp\big(\frac{{\epsilon_n}-\mu}{kT}\big)\pm 1},
\end{equation}
%
and substituting Eq.~(\ref{eq:qfneq}) into the negative of Eq.(\ref{eq:qHdef}), 
the entropy of a quantum ideal gas is obtained
%
\begin{eqnarray}\label{eq:varEntropyDef}
      S&=&\sum_{M=1}^{K}
        \sum_n  \left[
        		\left(\frac{1}{\exp\big(-\bar{\alpha}-\bar{\beta}\epsilon_{n}\big)\pm 1} \right)
           \commr{\ln \left(\frac{1}{\exp\big(-\bar{\alpha}-\bar{\beta}\epsilon_{n}\big)\pm 1} \right)}
         \right]\nonumber \\
      &&\quad\pm  \ln \left[\prod_{M=1}^{K} \prod_{n}
         \left(1 \mp \frac{1}{\exp\big(-\bar{\alpha}-\bar{\beta}\epsilon_{n}\big)\pm 1} \right)
         \right] \delta V_M.
\end{eqnarray}
%
This quantity is what we refer to as ``variational entropy'',
and this name reflects the fact that it was obtained \textit{via}
the variational method. {\color{blue} We will use it at the end of this article to analyze relaxation processes.}

%=============================================
\subsubsection{Proof of the quantum $H$-theorem for non-homogeneous systems}
%=============================================

For quantum systems, we also accept the validity of the local equilibrium hypothesis
for every cell in the system. This allows us to define non-homogeneous systems,
wherein thermodynamic quantities are well-defined on a cell-per-cell basis,
and in terms of the equilibrium properties, we have

%
\begin{subequations}\label{eq:restrictionoutside}
\begin{equation}
        \mathcal{N}_M(t)=\sum_{n}f_{Mn}(t)=\bar{\mathcal{N}}+\Delta_M(t) 
\end{equation}
  and
\begin{equation}
        \mathcal{E}_M(t)=\sum_{n}\epsilon_{n}f_{Mn}(t)=\bar{\mathcal{E}}+ \delta_M(t).
\end{equation}
\end{subequations}
%
In Eqs.~(\ref{eq:restrictionoutside}) $\bar {\mathcal{N}}$ and $\bar{\mathcal{E}}$ are the cell particle
number and the cell energy in equilibrium, which are given by Eqs.~(\ref{eq:qEqRestrictions}),
and $\Delta_M$ and $\delta_M$ are deviations from $\bar{\mathcal{N}}$
and $\bar{\mathcal{E}}$, respectively, with $\Delta_M(t)\ll \bar{\mathcal{N}}$
and $\delta_M(t) \ll \bar{\mathcal{E}}$. 


In the present context, $|\Delta_M|$ and $|\delta_M|$ are sufficiently large so they are not
fluctuations of the system, but enough small so that the local equilibrium hypothesis is
valid for $t>0$ (we set $t_0=0$, and $t_0$ is the initial time at which the system is prepared).
Therefore, it is reasonable to re-write the distribution functions as:
%
\begin{equation}\label{eq:qFirstOrd}
   f_{Mn}(t)=\bar{f}_n(1+g_{Mn}(t)),\quad
   1\gg|g_{Mn}(t)|,
\end{equation}
%
from which it follows, by substituting Eq.~(\ref{eq:qFirstOrd}) into
Eq.~(\ref{eq:restrictionoutside}), that $\Delta_M$ and $\delta_M$ satisfy:
%
\begin{subequations}
\begin{equation}
    \Delta_M(t)=\sum_n \bar{f}_n g_{nM}
\end{equation}
%
and
%
\begin{equation}
	\delta_M(t)=\sum_n  \bar{f}_n g_{nM}\epsilon_n.
\end{equation}
\end{subequations}
%

\commr{
An additional consideration is necessary for treating gases composed of fermions,
which must obey the relation $f_{Mn}(t)\leq1$, hence
%
\begin{eqnarray}\label{eq:fermionrestriction}
   1-\bar f_n -\bar f_n g_{nM}\geq0 \ \ \Rightarrow \ \ \frac{1}{\bar f_n}\geq1+g_{nM}.
\end{eqnarray}
%
$\bar f_{n}=1$ is certainly satisfied if the system is at zero temperature. In this state,
all energy levels below and including the Fermi energy are occupied because of
the exclusion principle, therefore the system will
necessarily be homogeneous , and consequently $g_{nM}=0$. In this article, we will
omit this scenario and will only discuss Fermi gases whose temperature is non zero.
}


%\begin{proof}
To proof the quantum $H$-theorem, start by taking the time-derivative of Eq. (\ref{eq:qHdef}):
%
\begin{equation}\label{eq:deltaH}
   \frac{d \mathcal{H} (t)}{dt}= \sum_n \sum_{M=1}^{K} \dot{f}_{nM}(t)\ln \left[ \frac{f_{nM}(t)}{1\mp f_{nM}(t)} \right] \de V_M.
\end{equation}
%
Subsequently, substitute Eq.~(\ref{eq:qFirstOrd}) in the above equation to obtain:
%
\begin{eqnarray}\label{eq:cambioH1}
    \frac{d\mathcal{H} (t)}{dt}&=&
      \sum_n \sum_{M=1}^{K} \bar{f}_{n}\ln \left[
        \frac{\bar{f}_{n}(1+g_{nM})}{1\mp \bar{f}_{n} (1+ g_{nM})}
      \right]\dot{g}_{nM} \delta V_M \nonumber \\
    &=&\sum_n \sum_{M=1}^{K} \bar{f}_n \left \{
      \ln [\bar{f}_n+\bar{f}_n g_{nM}]\dot{g}_{nM}
      -\ln [1\mp\bar{f}_n\mp\bar{f}_n g_{nM}]\dot{g}_{nM}
    \right \}\delta V_M.
\end{eqnarray}
%
Approximate the logarithmic terms, corresponding to
Fermi and Bose gases, through a Taylor series around $\bar f_n g_{nM}=0$:
%
\begin{subequations}\label{eq:qlnApprox}
\begin{equation}\label{eq:qlnApproxFermions}
	\ln[1\mp\bar{f}_n\mp\bar{f}_n g_{nM}]
    	\approx \ln[1\mp\bar{f}_n]\mp\frac{\bar{f}_n}{1\mp\bar{f}_{n}} g_{nM}
\end{equation}
%
and
%
\begin{equation}\label{eq:qlnApproxBosons}
    \ln [\bar{f}_n+\bar{f}_n g_{nM}] \approx \ln [\bar{f}_n]+ g_{nM},
\end{equation}
\end{subequations}
%
respectively. Eq.~(\ref{eq:qlnApproxFermions}) is valid because, for non-extremely degenerated Fermi gases, 
$1-\bar{f}_n \gg \bar{f}_n|g_{nM}|$, and Eq.~(\ref{eq:qlnApproxBosons}) is fulfilled because,
for Boson gases,
$1+\bar{f}_n \gg \bar{f}_n |g_{nM}|$ when $\bar{f}_n \gg \bar{f}_n |g_{nM}|$.

Combine Eqs. (\ref{eq:cambioH1}) and (\ref{eq:qlnApprox}):
%
\begin{eqnarray}\label{eq:cambioH2}
    \frac{d\mathcal{H}}{dt}&=&\sum_n \sum_{M=1}^{K} \bar{f}_n\left \{ (\ln \bar{f}_n+ g_{nM})\dot{g}_{nM}\right\} \de V_M \delta \epsilon_n \nonumber \\
    &&-\sum_{n}\sum_{M=1}^{K}\bar f_n\left\{ \left( \ln[1\mp\bar{f}_n]\mp \left[\frac{\bar{f}_n}{1\mp\bar{f}_n} \right] g_{nM}\right)\dot{g}_{nM} \right \}\de V_M,
\end{eqnarray}
%
and substitute Eq.~(\ref{eq:qfneq}) into Eq.~(\ref{eq:cambioH2}):
%
\begin{equation}\label{eq:cambioH3}
    \frac{d\mathcal{H}}{dt}=
       \sum_n \sum_{M=1}^{K} \bar{f}_n\left \{
          (\bar{\alpha}+\bar{\beta}{\epsilon}_n)\dot{g}_{nM}
           +g_{nM}\left(1\pm e^{\bar{\alpha}+\bar{\beta}{\epsilon}_n}\right)\dot{g}_{nM}
       \right \} \de V_M.
\end{equation}
%

Since both the total number of particles and the total energy 
of the system are constant, it follows,
from Eqs.~(\ref{eq:qGlobRest}) and (\ref{eq:restrictionoutside}) that:
%
\begin{subequations}
\begin{equation}
\frac{dN}{dt}=\sum_{M=1}^K\dot{\mathcal{N}}_M\delta V_M
   =\sum_{M=1}^K\sum_n \bar{f}_n \dot{g}_{nM}\delta V_M=\sum_{M=1}^K\dot{\Delta}_M(t)\delta V_M=0
\end{equation}
%
and
%
\begin{equation}
\frac{dE}{dt}=\sum_{M=1}^K\dot{\mathcal{E}}_M\delta V_M
   =\sum_{M=1}^K\sum_n \bar{f}_n \dot{g}_{nM}\epsilon_n\delta V_M=\sum_{M=1}^K\dot{\delta}_M(t)\delta V_M=0.
\end{equation}
\end{subequations}
%

Substitute the previous expression in Eq.~(\ref{eq:cambioH3}) to obtain:
%
\begin{equation}\label{eq:cambioH4}
   \frac{d\mathcal{H}}{dt}=\sum_n e^{\bar{\alpha}+\bar{\beta}\epsilon_n}
   \sum_M  g_{nM}\dot{g}_{nM} \delta V_M \leq 0. \qquad\qquad\textrm{QED.}
\end{equation}
%\end{proof}
%

For obtaining the right-most hand side of Eq.~(\ref{eq:cambioH4}), 
we have used the relation $g_{nM}\dot{g}_{nM}\leq0$ for $t>0$. This can be proven 
by simply arguing that, in the initial
state, if a cell is described by $g_{nM}(t_0)>0$ then $g_{nM}(t)\geq0$ and $\dot g_{nM}(t)\leq0$, and if
$g_{nM}(t_0)<0$ then $g_{nM}(t)\leq0$ and $\dot g_{nM}(t)\geq0$. Here we have used the facts that
the system in equilibrium is homogeneous, and that, by accepting the local equilibrium hypothesis, $g_{Mn}(t)$
is a monotonic function and $g_{nM}\to0$ as $t\to\infty$ as the system aproaches to the equilibrium state.
Another approach to prove Eq.~(\ref{eq:cambioH4}) consists of splitting the cells into two subsets,
just as we did in the classical scenario.

Succinctly, considering a diluted quantum gas contained in a vessel of volume $V$ (divided into
$K$ small cells), with total energy $E$ and $N$ quantum free particles, which initially
is out of equilibrium ---but in such a manner that the local equilibrium hypothesis is valid---,
the functional
%
\begin{eqnarray}\label{eq:qHfunctTheo}
    \mathcal{H} (t)&=&\sum_{M=1}^{K} \sum_{n} \bigg[ f_{Mn}(t) \ln f_{Mn}(t)\pm \Big(1 \mp f_{Mn}(t)) \ln (1 \mp f_{Mn}(t)\Big) \Big]   \delta V_M,
\end{eqnarray}
%
where $f_{Mn}$ is the $M$-th cell distribution function,
evolves in time such that $d\mathcal{H}/dt\leq0$, and the equality 
condition is attained when the system reaches
the equilibrium state. In Eq.~(\ref{eq:qHfunctTheo}), and for a Fermi (Bose)
gas, $f_{Mn}$ corresponds to the Fermi-Dirac (Bose-Einstein) distrubution function for each cell.
Locally, each cell is in equilibrium, although the complete system may be non-homogeneous,
and is caracterized by the respective $f_{Mn}$, number of particles $\mathcal{N}_M$, energy $\mathcal{E}_M$,
temperature $T_M$, and Legendre Multipliers $\alpha_M$ and $\beta_M$.



\end{paracol}

\begin{paracol}{2}
\switchcolumn\linenumbers

\section{Quantum Classical correspondence}
%The variational entropy $H$ that was defined in (\ref{entropy2}) can be reduce to the Boltzmann's $H$ functional in the appropriated limit.\\
We begin defining the distribution function as the number of particles in a defined cell and an energy level $N_{nM}$ inside a volume $\de V_M \delta \epsilon_n$ as
\begin{equation}
    f_{nM}=\frac{N_{nM}}{ \de V_M \delta \epsilon_{n} }.
\end{equation}
Using this definition on the variational entropy (\ref{entropy2}) we obtain
\begin{eqnarray}
    \Ss&=& \sum_{M=1}^{K} \sum_n
    \left[  
           \frac{N_{nM}}{ \de V_M\delta \epsilon_{n}} \ln 
           \left( 
                  \frac{N_{nM}}{ \de V_M\delta \epsilon_{n}}
           \right)\pm 
           \left(  
                  1\mp \frac{N_{nM}}{ \de V_M \delta \epsilon_{n}}
           \right) \ln 
           \left(  
                   1\mp \frac{N_{nM}}{ \de V_M \delta \epsilon_{n}}
           \right)
    \right] \de V_M \delta \epsilon_{n}. \nonumber \\
    \label{h-quantic} 
\end{eqnarray}
In the classic limit, the particles occupy high energy levels, and the number of particles in most of the groups of state $n$ is small compared with the number of the elementary states $\epsilon_n$ of the group, that is
\begin{equation}
    \frac{N_{nM}}{ \de V_M \delta \epsilon_{n} } \approx 0,
\end{equation}
and with the previous approximation, the second term in (\ref{h-quantic}) is negligible, then (\ref{h-quantic}) yields
\begin{eqnarray}
    \Ss&=& \sum_{M=1}^{K} \sum_n
    \left[  
           \frac{N_{nM}}{ \de V_M \delta \epsilon_{n}} \ln 
           \left( 
                  \frac{N_{nM}}{ \de V_M \delta \epsilon_{n}}
           \right)
    \right] \de V_M \delta \epsilon_{n}. \label{h-quantic2}
\end{eqnarray}
In the case when the system is spatial-homogeneous, the definition of the distribution function is 
\begin{equation}
    f_{n}=\frac{N_{n}}{ \delta \epsilon_{n} },
\end{equation}
and using the same limit we obtain
\begin{eqnarray}
    \Ss&=& \sum_{M=1}^{K} \sum_n
    \left[  
           \frac{N_{n}}{ \delta \epsilon_{n}} \ln 
           \left( 
                  \frac{N_{n}}{ \delta \epsilon_{n}}
           \right)
    \right]  \de V_M \delta \epsilon_{n} = V \sum_n
    \left[  
           \frac{N_{n}}{ \delta \epsilon_{n}} \ln 
           \left( 
                  \frac{N_{n}}{ \delta \epsilon_{n}}
           \right)
    \right] \delta \epsilon_{n} \nonumber \\
    &=& V \sum_n \left[N_n \ln N_n - N_n \ln \delta \epsilon_n  \right]=\de V_M K H_{Boltz},\label{h-quantic4}
\end{eqnarray}
where $H_{Boltz}$ now is (\ref{CH2}) expressed is the form of (\ref{reduce-h}) shown by Tolman.\footnote{See \cite{bib:tolman} eq. (102.6) for more information.}\\
We can observe that if the number of particles is too big compared to the number of cells $K$, then we can conclude that the sum of the local variational entropy is equal to the variational entropy of the entire system.\\
\textcolor{blue}{In the equilibrium case, we have the following variational entropy
\begin{eqnarray}
    \Ss&=& \sum_{M=1}^{K}\sum_{n}
        \left[
                \frac{1}{e^{-\alpha_M-\beta_M \epsilon}\pm 1} \ln 
                    \left(
                            \frac{1}{e^{-\alpha_M-\beta_M \epsilon}\pm 1}
                    \right)
        \right. \nonumber \\
          && \pm \left. \left(
                        1 \mp \frac{1}{e^{-\alpha_M-\beta_M \epsilon}\pm 1}
                  \right) \ln
                \left(
                        1 \mp \frac{1}{e^{-\alpha_M-\beta_M \epsilon}\pm 1}            
                \right) \right]. \label{equilibriumvariational}
\end{eqnarray}
To obtain the classical limit, that is, to recover the Maxwell-Boltzmann distribution from the Fermi-Dirac and Bose-Einstein distribution, it is necessary to hold the following approximation
\begin{equation}
    e^{-\alpha_M-\beta_M \epsilon}\gg 1, \label{classicalapproximation}
\end{equation}
and then
\begin{equation}
    \frac{1}{e^{-\alpha_M-\beta_M \epsilon}\pm 1} \approx e^{\alpha_M+\beta_M \epsilon}.
\end{equation}
If we apply (\ref{classicalapproximation}) to (\ref{equilibriumvariational}), we obtain
\begin{eqnarray}
    \Ss&=& \sum_{M=1}^{K}\sum_{n}
        \left[
                e^{\alpha_M+\beta_M \epsilon} \ln 
                    \left(
                            e^{\alpha_M+\beta_M \epsilon}
                    \right)
        \right. \nonumber \\
          && \pm \left. \left(
                        1 \mp e^{\alpha_M+\beta_M \epsilon}
                  \right) \ln
                \left(
                        1 \mp e^{\alpha_M+\beta_M \epsilon}            
                \right) \right],
\end{eqnarray}
but if (\ref{classicalapproximation}) holds, then 
\begin{equation}
     e^{\alpha_M+\beta_M \epsilon}\ll 1,
\end{equation}
consequently 
\begin{equation}
    \ln(1\mp e^{\alpha_M+\beta_M \epsilon}) \approx \ln 1 \approx 0,
\end{equation}
and finally
\begin{equation}
    \Ss=\sum_{M=1}^{K}\sum_{n}
        \left[
                e^{\alpha_M+\beta_M \epsilon} \ln 
                    \left(
                            e^{\alpha_M+\beta_M \epsilon}
                    \right)
        \right],
\end{equation}
that corresponds to the classical $H$ functional in equilibrium described in the energy space.}\\
\textcolor{red}{Finally, as a complementary topic, in the next section, we propose a method to obtain the quantum analogous from the Boltzmann transport equation for relaxation processes.}

%--------------------------------------

\end{paracol}

\color{blue}
\begin{paracol}{2}
\switchcolumn\linenumbers

\section{Relaxation processes in degenerated quantum gases}\label{sec:relaxproc}

In order to obtain a time-evolution equation for an out-of-equilibrium quantum gas,
we propose the following approach. We start by evaluating $\Delta\mathcal{H}=
\mathcal{H}(t_2)-\mathcal{H}(t_1)$,
where our quantum $H$-functional ---Eq.~(\ref{eq:qHdef})--- is evaluated at diferent times $t_1$ and $t_2$, with
$t_2 > t_1$. This yields:
%
\begin{eqnarray}\label{eq:deltaHdef}
	\Delta\mathcal H &=& \sum_{M=1}^{K}\sum_n [f''_{nM} \ln f''_{nM} - f'_{nM}	\ln f'_{nM}\nonumber \\ 
			&&\pm (1\mp f''_{nM}) \ln(1\mp f''_{nM}) \mp (1\mp f'_{nM}) \ln (1\mp f'_{nM})]\delta V_M. 
\end{eqnarray}
%
In the above equation, and for the rest of this section, we use the short-hand notation
$f_{nM}(t_2)\equiv f''_{nM}$ and 
$f_{nM}(t_1) \equiv f'_{nM}$.
Subsequently, in Eq.~(\ref{eq:deltaHdef}), we replace the distribution functions by their expresions in terms
of deviations from equilibrium ---Eq.~(\ref{eq:qFirstOrd})---, which renders:
\begin{eqnarray}\label{eq:deltaH1}
	\Delta\mathcal H &=& \sum_{M=1}^{K} \sum_n\Big[\bar{f}_n(1+g''_{nM}) \ln \bar{f}_{n}(1+g''_{nM})  -  
			\bar{f}_n(1+g'_{nM}) \ln \bar{f}_n(1+g'_{nM})\Big. \nonumber \\
		    &&  \pm \big(1 \mp \bar{f}_n \{1+g''_{nM}\}\big) \ln \big(1\mp \bar{f}_n\{1+g''_{nM} \}\big)\nonumber \\
		    &&  \mp \big(1\mp \bar{f}_n \{ 1+g'_{nM} \}\big) \ln\big(1\mp \bar{f}_n \{1+g'_{nM} \}\big)\Big.\Big]
		    \delta V_M.
\end{eqnarray}
Afterwards, we expand the logarithmic terms up to the first-order in $g'_{nM}$ and $g''_{nM}$
and rearrange the result, which gives:
%
\begin{equation}\label{eq:deltaH2}
\Delta\mathcal H = \sum_{M}^{K}\sum_n [\bar{f}_n(1
	+\ln \bar{f}_n)-\bar{f}_n \{\ln(1\mp \bar{f}_n)\mp 1 \}](g''_{nM}-g'_{nM})\delta V_M.
\end{equation}
%
Finally we divide Eq.~(\ref{eq:deltaH2}) by $\Delta t\equiv t_2-t_1$, and take the limit
$\Delta t\to0$ to obtain:
%
\begin{equation}\label{eq:deltaH3}
\frac{d\mathcal H}{dt} = \sum_{M}^{K}\sum_n [\bar{f}_n(1
	+\ln \bar{f}_n)-\bar{f}_n \{\ln(1\mp \bar{f}_n)\mp 1 \}]\left(\frac{g_{nM}}{dt}\right)\delta V_M.
\end{equation}
%

Eq.~(\ref{eq:deltaH3}) is, within our framework, the time-evolution equation for $g_{nM}$.
Clearly, for describing a concrete situation, it is required to provide a specific approximation for the deviation function $g_{nM}$, however this will be the subject of future work.

  
  %----------------------------------------


%-------------------------------------------------


\end{paracol}

\begin{paracol}{2}
\switchcolumn\linenumbers
%--------------------------------------
% Me parece que sera necesario incluir la discusin sobre los procesos de relajacin.
%Sin esa parte el contenido original del artculo podria parecer insuficiente para
%una buena publicacion%

\color{red}
\section{Discussions and remarks}\label{sec:disscussion}

Traditionally, the proof of the classical $H$-theorem involves assuming that the
distribution function is spatially homogeneous, and the proof starts with
considering particular systems that are out of equilibrium, but in such a manner
that their distribution functions are still homogeneous. This out-of-equilibrium systems
are conceived considering the micro-states of the system, and the $H$-theorem determines
the microscopic evolution of the system towards a state with the most-probable distribution function,
given a thermodynamic state. 
This approach does not cover systems with spatial inhomogeneities.
To address this issue, we
proposed a framework that may be useful to describe the time-evolution of systems
initially non-homogeneous. For this, we divided the
system into small cells, so as to conceive a system wherein we accept the local equilibrium
hypothesis to be valid for each cell, but in such a manner that the global system is not homogeneous. Systems
that satisfy the previous conditions will evolve towards global equilibrium, and the evolution is
determined by the relations $d\mathcal{H}'/dt\leq0$, Eq.~(\ref{eq:cHdef}),
and $d\mathcal{H}/dt\leq0$, Eq.~(\ref{eq:qHdef}), for classical and quantum gases, respectively.
In some aspects, this can be considered as an extension or as a potential equivalent of the $H$ theorem
for systems out-of-equilibrium in the spatial sense, \textit{i.e.} systems whose thermodynamic
variables are not spatially homogeneous.

The classical and quantum $H$-functionals, $\mathcal{H}'$ and $\mathcal{H}$, respectively,
correctly recover the most-probable distribution
functions both in out-of-equilibrium states (locally) and when the system attains the global
equilibrium state. This enables the use of all the machinery of thermodynamics, including the
proper definition of state variables as functions of time for each cell. In addition, our approach
does not require to know the specific mechanisms by which energy or particles are transferred from one
cell to its neighbors. Instead, the time-evolution from the out-of-equilibrium to equilibrium states are
governed by monotonic functions that account for deviations from the global equilibrium.

The specific forms of the functions $g'_M$ and $g_{nM}$ for classical and quantum systems will be the
subject of our future work. However, some of their properties can be foreseen, \textit{e.g.} they must
be consistent with the system relaxation times, they must be consistent with the mechanisms of 
energy transfer between open systems, there must be related to the cell thermodynamic variables, etc.

An important aspect of the framework proposed in this work is related to the entropy of systems
out-of-equlibrium. Because the functionals $\mathcal{H}'$ and $\mathcal{H}$ can be related to
the entropy of diluted gases, either classical or quantum, the fact that these functionals are defined
over a system divided into cells enables their use for defining the entropy
of other out-of-equilibrium systems. In particular, and 
derived from our previous work (\textit{e.g.} \cite{bib:nicolas2020,bib:nicolas2016}), the
$\mathcal H'$ and $\mathcal H$ functionals may serve to describe the entropy,
as well as the entropy generation, occurring during the
growth of complex physical systems, such as fractals. Possibly, studying these systems might also
shed light on the functional form of $g'_M$ and $g_{nM}$.

In conclusion, we have proposed a framework for studying classical and quantum out-of-equilibrium
diluted gases, whose initial
state is inhomogeneous in nature. This framework consists, briefly, of dividing the system into
a set of small cells, and assuming that each cell is in equilibrium. Subsequently,
$H$-functionals, $\mathcal H$, can be defined
such that the time evolution towards equilibrium resembles the Boltzmann $H$-theorem: $d\mathcal H/dt\leq0$,
and the equality condition is satisfied when the system attains the global equlibrium.
The distribution functions,
stemming from the $\mathcal H$ functionals, at any time, $t$, are the respective equilibrium distribution functions,
which can be Maxwell-Boltzmann, Fermi-Dirac or Bose-Einstein, depending on the classical or quantum nature
of the gas under consideration. Furthermore,
the distribution function of each cell evolves so as to match the global distribution function in equilibrium.
A proof of the relation $d\mathcal H/dt\leq0$ is performed through the variational method,
the functional form of the distribution
functions, in terms of a monotonic function $g$, are defined, and some of the properties of $g$ are discussed.
The functionals $\mathcal H$ can be associated with the entropy of the system under consideration, hence we
refer to this entropy as the variational entropy of the system. Finally,
a potential application of this framework is outlined briefly. 



\section{Remarks and conclusions}\label{sec:disscussion}

In the literature, the Boltzmann's $H$-theorem is proved using a spatially homogeneous distribution function. 
Here, we considered inhomogeneities in the system through the division of the volume $V$ in cells 
and the non-homogeneous distribution function assumption in the total volume. 
We have shown that the Boltzmann's $H$-theorem still holds when the system includes 
inhomogeneities, of course assuming that the local equilibrium hypothesis holds. 
We demostrated the classical Boltzmann's $H$-theorem without using explicity the Boltzmann's transport equation. 
Instead of that, it needed to specify the behavior of the deviation $g$ and its time derivative. 
\\
In the quantum version, we used the same assumptions to prove the $H$-theorem, those were the 
behavior of the deviation $g_{nM}$ and its derivative, the local equilibrium hypothesis, and the 
non-homogeneous distribution function assumption. 
We have corroborated the correctness of the $f$ expression in equilibrium recovering 
the MB, FD and BE distributions and in recovering the expression of the entropy of an ideal gas. 
We demostrated as well that the functional $H$ leads to the proper fulfilling of the quantum classical correspondence.
\\
\\
We also have shown that variational entropy is proportional to the Boltzmann's $H$-functional in the non-degenerated limit. 
This is a straightfowars conclusion. One may infer from the expression for the Boltzmann's $H$-functional in equilibrium that
\begin{equation}
    H_{B}=-\frac{S}{Vk},
\end{equation}
where $S$ is the entropy of the system, and $k$ is the Boltzmann's constant, we can identify that variational entropy is proportional to the entropy of the system.\\
\begin{equation}
    \Ss\propto S.
\end{equation}
It is important to remark that we find that $\frac{d\mathcal{H}'}{dt}\leq 0$ and $\frac{d\mathcal{H}}{dt}\leq 0$ 
setting the expecting behavior of the deviation $g$. 
This correct behavior results from the molecular chaos hypothesis in the classical case and 
the random a priori phases hypothesis in the quantum case. 
Specifically those hypothesis are included in the assumed behavior of $\dot{g}_{nJ}^{+}$ and $\dot{g}_{nJ}^{-}$. 
Those time derivatives correspond to the expected behavior in a meantime, that 
is, $\dot{g}_{nJ}^{+}$ always decreases and $\dot{g}_{nJ}^{-}$ always increases.\\ 
According to the classical $H$-theorem, if the molecular chaos hypothesis holds, 
although within small times of the order of the \textit{relaxation time}, as a consequence of the 
stochastic nature of the collisions, $H$ function evolves through a sequential series of 
molecular chaos and non-molecular chaos, but a larger times $\frac{dH}{dt}\leq 0$.\\
In analogy with this, we see that in our procedure this behavior is introduced by 
$\dot{g}_{nJ}^{+}$ and $\dot{g}_{nJ}^{-}$ to obtain the expected behavior in a relaxation time. 
So, molecular chaos hypothesis and a random a priori phases hypothesis are implicit
 in the expected behavior of the deviation from the equilibrium of the $H$ function such that 
$\frac{d\mathcal{H}'}{dt}\leq 0$ and $\frac{d\mathcal{H}}{dt}\leq 0$.
\\
\\
In summaty, we propose a variational procedure to demostrate the classical and quantum $H$ theorem, that
make possible to describe at a mesoscopic local view (cell-scale), the time evolution 
of an out-of-equilibrium, spatially non-homogeneous, system to the equilibrium condition.
In principle, this approach would make possible investigate the transport phenomena inherent to the
equilibrium process, starting from an arbitrarily far-from-equilibrium initial condition.

%-----------------------------------------------


\end{paracol}

\begin{paracol}{2}
\switchcolumn\linenumbers
\section{Conclusions}\label{sec:conclusions}
We conclude the following statements:\\
\begin{enumerate}
    \item Any system (classical or quantum) out of equilibrium (including inhomogeneities) evolves to the equilibrium state without using the transport equation as long as the local equilibrium hypothesis holds.
    \item We can obtain the transport equation from the $H$ functional.
    \item The variational entropy is proportional to the Boltzmann's $H$ functional in the limit of the high energy levels.
    \item We find an expression for the entropy to quantum gases.
    \end{enumerate}
  
%  \begin{equation}
%      \mathcal{H}
%  \end{equation}
  
\clearpage




\end{paracol}

%%%%%%%%%%%%%%%%%%%%%%%%%%%%%%%%%%%%%%%%%%
\vspace{6pt} 

%%%%%%%%%%%%%%%%%%%%%%%%%%%%%%%%%%%%%%%%%%
%% optional
%\supplementary{The following are available online at \linksupplementary{s1}, Figure S1: title, Table S1: title, Video S1: title.}

% Only for the journal Methods and Protocols:
% If you wish to submit a video article, please do so with any other supplementary material.
% \supplementary{The following are available at \linksupplementary{s1}, Figure S1: title, Table S1: title, Video S1: title. A supporting video article is available at doi: link.} 

%%%%%%%%%%%%%%%%%%%%%%%%%%%%%%%%%%%%%%%%%%
\begin{paracol}{2}
\switchcolumn\linenumbers
\authorcontributions{Conceptualization, J.L.C.E.; methodology, C.M.P., J.M.S.A. and J.L.C.E.; formal analysis, C.M.P.; validation, J.M.S.A. and J.L.C.E.; writing---original draft preparation, C.M.P.; writing---review and editing, J.M.S.A. and J.L.C.E.; visualization, J.M.S.A.; supervision, J.M.S.A. and J.L.C.E.; project administration J.M.S.A and J.L.C.E.; funding acquisition, J.L.C.E. All authors have read and agreed to the published version of the manuscript.}

\funding{\color{blue}This research was funded by NAME OF FUNDER grant number XXX.}

%\dataavailability{Please refer to suggested Data Availability Statements in section “MDPI Research Data Policies” at \href{https://www.mdpi.com/ethics}{https://www.mdpi.com/ethics}.} 

\acknowledgments{CMP acknowledges Conacyt for his PhD scholarship (Registration Number: {\color{red}XXXXXX}).}

\conflictsofinterest{The authors declare no conflict of interest. The funders had no role in the design of the study; in the collection, analyses, or interpretation of data; in the writing of the manuscript, or in the decision to publish the~results.} 

%% Optional
%\sampleavailability{Samples of the compounds ... are available from the authors.}

%%%%%%%%%%%%%%%%%%%%%%%%%%%%%%%%%%%%%%%%%%
%% Only for journal Encyclopedia
%\entrylink{The Link to this entry published on the encyclopedia platform.}

%%%%%%%%%%%%%%%%%%%%%%%%%%%%%%%%%%%%%%%%%%
%% Optional
%\begin{comment}
\abbreviations{The following abbreviations are used in this manuscript:\\

\noindent 
\begin{tabular}{@{}ll}
$H_B$ & The original Boltzmann $H$-functional \\
$H'$  & Our $H$-functional for a classical diluted gas \\
$f_B$ & The Maxwell-Boltzmann distribution function \\
$f_M$ & The distribution function of a cell
\end{tabular}}
%\end{comment}

%%%%%%%%%%%%%%%%%%%%%%%%%%%%%%%%%%%%%%%%%%
\begin{comment}
%% Optional
\appendixtitles{no} % Leave argument "no" if all appendix headings stay EMPTY (then no dot is printed after "Appendix A"). If the appendix sections contain a heading then change the argument to "yes".
\appendix
\section{}
\subsection{}
The appendix is an optional section that can contain details and data supplemental to the main text---for example, explanations of experimental details that would disrupt the flow of the main text but nonetheless remain crucial to understanding and reproducing the research shown; figures of replicates for experiments of which representative data are shown in the main text can be added here if brief, or as Supplementary Data. Mathematical proofs of results not central to the paper can be added as an appendix.

\section{}
All appendix sections must be cited in the main text. In the appendices, Figures, Tables, etc. should be labeled, starting with ``A''---e.g., Figure A1, Figure A2, etc. 
\end{comment}
%%%%%%%%%%%%%%%%%%%%%%%%%%%%%%%%%%%%%%%%%%
\end{paracol}

\reftitle{References}

% Please provide either the correct journal abbreviation (e.g. according to the “List of Title Word Abbreviations” http://www.issn.org/services/online-services/access-to-the-ltwa/) or the full name of the journal.
% Citations and References in Supplementary files are permitted provided that they also appear in the reference list here. 

%=====================================
% References, variant A: external bibliography
%=====================================
%\externalbibliography{yes}
\bibliography{bibliography}

% The following MDPI journals use author-date citation: Arts, Econometrics, Economies, Genealogy, Humanities, IJFS, JRFM, Laws, Religions, Risks, Social Sciences. For those journals, please follow the formatting guidelines on http://www.mdpi.com/authors/references
% To cite two works by the same author: \citeauthor{ref-journal-1a} (\citeyear{ref-journal-1a}, \citeyear{ref-journal-1b}). This produces: Whittaker (1967, 1975)
% To cite two works by the same author with specific pages: \citeauthor{ref-journal-3a} (\citeyear{ref-journal-3a}, p. 328; \citeyear{ref-journal-3b}, p.475). This produces: Wong (1999, p. 328; 2000, p. 475)

%%%%%%%%%%%%%%%%%%%%%%%%%%%%%%%%%%%%%%%%%%
%% for journal Sci
%\reviewreports{\\
%Reviewer 1 comments and authors’ response\\
%Reviewer 2 comments and authors’ response\\
%Reviewer 3 comments and authors’ response
%}
%%%%%%%%%%%%%%%%%%%%%%%%%%%%%%%%%%%%%%%%%%
\end{document}

