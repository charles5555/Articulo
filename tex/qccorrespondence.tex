

\section{Quantum Classical correspondence}\label{sec:qccorrespondence}

In sections \ref{sec:classical} and \ref{sec:quantum}, we have seen that the variational
method can be applied to $H$-functionals, which correctly describes the behavior of
classical and quantum diluted gases, as to their respective time-evolution.
Both $H$-functionals defined in Eqs.~(\ref{eq:cHdef}) and
(\ref{eq:qHdef}) also recovers the well-known distribution functions, either Maxwell-Boltzmann
for a classical gas, Fermi-Dirac for a Fermi gas, or Bose-Einstein for a Bose gas. Nevertheless,
the functionals (\ref{eq:cHdef}) and
(\ref{eq:qHdef}) are seemingly different, and in this section we will show that both are
related by the correspondence principle.

We start by arguing that, in the classic limit, $f_{nM}\approx0$ for the following reasons.
Systems at very low temperatures, where the quantum effects cannot be ignored, are obviously excluded
from the current analysis. For systems at sufficently large temperature, the particles
occupy almost exclusively high-energy levels. 
Furthermore, the energy spectrum approaches to be a continuum, as is expected by taking the limit
$\hbar\to0$, and the number of particles per level are very close to zero.

Subsequently, we substitute $f_{nM}\approx0$ into Eq.~(\ref{eq:qHdef}) to obtain:
%
\begin{equation}\label{h-quantic2}
    \mathcal{H}= \sum_{M=1}^{K} \sum_n
    \left[ f_{nM}(t)\ln \left(f_{nM}(t)\right)\right] \delta V_M
    =\sum_{M=1}^{K} \sum_n
    \left[ f_{M}(\epsilon_n,t) \ln \left( f_{M}(\epsilon_n,t)\right)\right] \delta V_M.
\end{equation}
%



\color{blue}



In the case when the system is spatial-homogeneous, the definition of the distribution function is 
\begin{equation}
    f_{n}=\frac{N_{n}}{ \delta \epsilon_{n} },
\end{equation}
and using the same limit we obtain
\begin{eqnarray}
    \Ss&=& \sum_{M=1}^{K} \sum_n
    \left[  
           \frac{N_{n}}{ \delta \epsilon_{n}} \ln 
           \left( 
                  \frac{N_{n}}{ \delta \epsilon_{n}}
           \right)
    \right]  \delta V_M \delta \epsilon_{n} = V \sum_n
    \left[  
           \frac{N_{n}}{ \delta \epsilon_{n}} \ln 
           \left( 
                  \frac{N_{n}}{ \delta \epsilon_{n}}
           \right)
    \right] \delta \epsilon_{n} \nonumber \\
    &=& V \sum_n \left[N_n \ln N_n - N_n \ln \delta \epsilon_n  \right]=\delta V_M K H_{B},\label{h-quantic4}
\end{eqnarray}
where $H_{B}$ now is (\ref{eq:cHdef}) expressed is the form of (\ref{reduce-h}) shown by Tolman.\footnote{See \cite{bib:tolman} eq. (102.6) for more information.}\\
We can observe that if the number of particles is too big compared to the number of cells $K$, then we can conclude that the sum of the local variational entropy is equal to the variational entropy of the entire system.

In the equilibrium case, we have the following variational entropy
\begin{eqnarray}
    \Ss&=& \sum_{M=1}^{K}\sum_{n}
        \left[
                \frac{1}{e^{-\alpha_M-\beta_M \epsilon}\pm 1} \ln 
                    \left(
                            \frac{1}{e^{-\alpha_M-\beta_M \epsilon}\pm 1}
                    \right)
        \right. \nonumber \\
          && \pm \left. \left(
                        1 \mp \frac{1}{e^{-\alpha_M-\beta_M \epsilon}\pm 1}
                  \right) \ln
                \left(
                        1 \mp \frac{1}{e^{-\alpha_M-\beta_M \epsilon}\pm 1}            
                \right) \right]. \label{equilibriumvariational}
\end{eqnarray}
To obtain the classical limit, that is, to recover the Maxwell-Boltzmann distribution from the Fermi-Dirac and Bose-Einstein distribution, it is necessary to hold the following approximation
\begin{equation}
    e^{-\alpha_M-\beta_M \epsilon}\gg 1, \label{classicalapproximation}
\end{equation}
and then
\begin{equation}
    \frac{1}{e^{-\alpha_M-\beta_M \epsilon}\pm 1} \approx e^{\alpha_M+\beta_M \epsilon}.
\end{equation}
If we apply (\ref{classicalapproximation}) to (\ref{equilibriumvariational}), we obtain
\begin{eqnarray}
    \Ss&=& \sum_{M=1}^{K}\sum_{n}
        \left[
                e^{\alpha_M+\beta_M \epsilon} \ln 
                    \left(
                            e^{\alpha_M+\beta_M \epsilon}
                    \right)
        \right. \nonumber \\
          && \pm \left. \left(
                        1 \mp e^{\alpha_M+\beta_M \epsilon}
                  \right) \ln
                \left(
                        1 \mp e^{\alpha_M+\beta_M \epsilon}            
                \right) \right],
\end{eqnarray}
but if (\ref{classicalapproximation}) holds, then 
\begin{equation}
     e^{\alpha_M+\beta_M \epsilon}\ll 1,
\end{equation}
consequently 
\begin{equation}
    \ln(1\mp e^{\alpha_M+\beta_M \epsilon}) \approx \ln 1 \approx 0,
\end{equation}
and finally
\begin{equation}
    \Ss=\sum_{M=1}^{K}\sum_{n}
        \left[
                e^{\alpha_M+\beta_M \epsilon} \ln 
                    \left(
                            e^{\alpha_M+\beta_M \epsilon}
                    \right)
        \right],
\end{equation}
that corresponds to the classical $H$-functional in equilibrium described in the energy space.


Finally, as a complementary topic, in the next section, we propose a method to obtain the quantum analogous from the Boltzmann transport equation for relaxation processes.

%--------------------------------------
