

\section{Quantum Classical correspondence}\label{sec:qccorrespondence}

In sections \ref{sec:classical} and \ref{sec:quantum}, we have seen that the variational
method can be applied to $H$-functionals, which correctly describes the behavior of
classical and quantum diluted gases, as to their respective time-evolution.
Both $H$-functionals defined in Eqs.~(\ref{eq:cHdef}) and
(\ref{eq:qHdef}) also recovers the well-known distribution functions, either Maxwell-Boltzmann
for a classical gas, Fermi-Dirac for a Fermi gas, or Bose-Einstein for a Bose gas. Nevertheless,
the functionals (\ref{eq:cHdef}) and
(\ref{eq:qHdef}) are seemingly different, and in this section we will show that both are
related by the correspondence principle.

We start by arguing that, in equilibrium, it is a trivial task to proof that expression (\ref{eq:qfneqtwo}) collapses
into expression (\ref{eq:fbardef}) by simply letting that the 
degeneration parameter $\xi=e^{-\frac{1}{K_B T}(\epsilon-\mu)} \gg 1$.
A more general approach to see the quantum classical correspondence would be if the quantum functional 
$H$ collapsed  into the classical one in the appropiated limit.
This limit in our case will be $f_{nM} \approx 0$ because the following reasons.
Systems at very low temperatures, where the quantum effects cannot be ignored, are obviously excluded
from the current analysis. For systems at sufficently large temperature, the particles
occupy almost exclusively high-energy levels. 
Furthermore, the energy spectrum approaches to be a continuum, as is expected by taking the limit
$\hbar\to0$, and the number of particles per level are very close to zero.

Subsequently, we substitute $f_{nM}\approx0$ into Eq.~(\ref{eq:qHdef}) to obtain:
%
\begin{equation}\label{h-quantic2}
    \mathcal{H}= \sum_{M=1}^{K} \sum_n
    \left[ f_{nM}(t)\ln \left(f_{nM}(t)\right)\right] \delta V_M
    =\sum_{M=1}^{K} \sum_n
    \left[ f_{M}(\epsilon_n,t) \ln \left( f_{M}(\epsilon_n,t)\right)\right] \delta V_M.
\end{equation}
%

Finally, the sum over the quatum energy levels can be replaced by an integral over the velocities
by invoking both the uncertainty principle and the fact that, for free particles, the continuum
energy spectrum
can be written as a function of the velocity. Hence the quantum $H$-functional
transforms, in the classical limit, to:
%
\begin{equation}\label{eq:qHclassLim}
    \mathcal{H}= \sum_{M=1}^{K}\int C'f_M(\vec v,t)\ln\big[C' f_M(\vec v,t)\big] d\vec v,
\end{equation}
%
where $C'$ collects the appropriate constants stemming from writing the energy spectrum as
a function of $\vec v$.






%--------------------------------------
