\section{Quantum Classical correspondence}\label{sec:qccorrespondence}


We obtained some results from the Statistical Mechanic in Quantum Mechanics
scheme. However, the correspondence principle is important to recover all
results from Classical Mechanics. This correspondence principle will be
discussed in the next section.

%The variational entropy $H$ that was defined in (\ref{entropy2}) can be reduce to the Boltzmann's $H$-functional in the appropriated limit.\\
We begin defining the distribution function as the number of particles in a defined cell and an energy level $N_{nM}$ inside a volume $\de V_M \delta \epsilon_n$ as
\begin{equation}
    f_{nM}=\frac{N_{nM}}{ \de V_M \delta \epsilon_{n} }.
\end{equation}
Using this definition on the variational entropy (\ref{entropy2}) we obtain
\begin{eqnarray}
    \Ss&=& \sum_{M=1}^{K} \sum_n
    \left[  
           \frac{N_{nM}}{ \de V_M\delta \epsilon_{n}} \ln 
           \left( 
                  \frac{N_{nM}}{ \de V_M\delta \epsilon_{n}}
           \right)\pm 
           \left(  
                  1\mp \frac{N_{nM}}{ \de V_M \delta \epsilon_{n}}
           \right) \ln 
           \left(  
                   1\mp \frac{N_{nM}}{ \de V_M \delta \epsilon_{n}}
           \right)
    \right] \de V_M \delta \epsilon_{n}. \nonumber \\
    \label{h-quantic} 
\end{eqnarray}
In the classic limit, the particles occupy high energy levels, and the number of particles in most of the groups of state $n$ is small compared with the number of the elementary states $\epsilon_n$ of the group, that is
\begin{equation}
    \frac{N_{nM}}{ \de V_M \delta \epsilon_{n} } \approx 0,
\end{equation}
and with the previous approximation, the second term in (\ref{h-quantic}) is negligible, then (\ref{h-quantic}) yields
\begin{eqnarray}
    \Ss&=& \sum_{M=1}^{K} \sum_n
    \left[  
           \frac{N_{nM}}{ \de V_M \delta \epsilon_{n}} \ln 
           \left( 
                  \frac{N_{nM}}{ \de V_M \delta \epsilon_{n}}
           \right)
    \right] \de V_M \delta \epsilon_{n}. \label{h-quantic2}
\end{eqnarray}
In the case when the system is spatial-homogeneous, the definition of the distribution function is 
\begin{equation}
    f_{n}=\frac{N_{n}}{ \delta \epsilon_{n} },
\end{equation}
and using the same limit we obtain
\begin{eqnarray}
    \Ss&=& \sum_{M=1}^{K} \sum_n
    \left[  
           \frac{N_{n}}{ \delta \epsilon_{n}} \ln 
           \left( 
                  \frac{N_{n}}{ \delta \epsilon_{n}}
           \right)
    \right]  \de V_M \delta \epsilon_{n} = V \sum_n
    \left[  
           \frac{N_{n}}{ \delta \epsilon_{n}} \ln 
           \left( 
                  \frac{N_{n}}{ \delta \epsilon_{n}}
           \right)
    \right] \delta \epsilon_{n} \nonumber \\
    &=& V \sum_n \left[N_n \ln N_n - N_n \ln \delta \epsilon_n  \right]=\de V_M K H_{B},\label{h-quantic4}
\end{eqnarray}
where $H_{B}$ now is (\ref{eq:CH2}) expressed is the form of (\ref{reduce-h}) shown by Tolman.\footnote{See \cite{bib:tolman} eq. (102.6) for more information.}\\
We can observe that if the number of particles is too big compared to the number of cells $K$, then we can conclude that the sum of the local variational entropy is equal to the variational entropy of the entire system.

In the equilibrium case, we have the following variational entropy
\begin{eqnarray}
    \Ss&=& \sum_{M=1}^{K}\sum_{n}
        \left[
                \frac{1}{e^{-\alpha_M-\beta_M \epsilon}\pm 1} \ln 
                    \left(
                            \frac{1}{e^{-\alpha_M-\beta_M \epsilon}\pm 1}
                    \right)
        \right. \nonumber \\
          && \pm \left. \left(
                        1 \mp \frac{1}{e^{-\alpha_M-\beta_M \epsilon}\pm 1}
                  \right) \ln
                \left(
                        1 \mp \frac{1}{e^{-\alpha_M-\beta_M \epsilon}\pm 1}            
                \right) \right]. \label{equilibriumvariational}
\end{eqnarray}
To obtain the classical limit, that is, to recover the Maxwell-Boltzmann distribution from the Fermi-Dirac and Bose-Einstein distribution, it is necessary to hold the following approximation
\begin{equation}
    e^{-\alpha_M-\beta_M \epsilon}\gg 1, \label{classicalapproximation}
\end{equation}
and then
\begin{equation}
    \frac{1}{e^{-\alpha_M-\beta_M \epsilon}\pm 1} \approx e^{\alpha_M+\beta_M \epsilon}.
\end{equation}
If we apply (\ref{classicalapproximation}) to (\ref{equilibriumvariational}), we obtain
\begin{eqnarray}
    \Ss&=& \sum_{M=1}^{K}\sum_{n}
        \left[
                e^{\alpha_M+\beta_M \epsilon} \ln 
                    \left(
                            e^{\alpha_M+\beta_M \epsilon}
                    \right)
        \right. \nonumber \\
          && \pm \left. \left(
                        1 \mp e^{\alpha_M+\beta_M \epsilon}
                  \right) \ln
                \left(
                        1 \mp e^{\alpha_M+\beta_M \epsilon}            
                \right) \right],
\end{eqnarray}
but if (\ref{classicalapproximation}) holds, then 
\begin{equation}
     e^{\alpha_M+\beta_M \epsilon}\ll 1,
\end{equation}
consequently 
\begin{equation}
    \ln(1\mp e^{\alpha_M+\beta_M \epsilon}) \approx \ln 1 \approx 0,
\end{equation}
and finally
\begin{equation}
    \Ss=\sum_{M=1}^{K}\sum_{n}
        \left[
                e^{\alpha_M+\beta_M \epsilon} \ln 
                    \left(
                            e^{\alpha_M+\beta_M \epsilon}
                    \right)
        \right],
\end{equation}
that corresponds to the classical $H$-functional in equilibrium described in the energy space.


Finally, as a complementary topic, in the next section, we propose a method to obtain the quantum analogous from the Boltzmann transport equation for relaxation processes.

%--------------------------------------
