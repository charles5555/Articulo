%--------------------------------------
% Me parece que sera necesario incluir la discusin sobre los procesos de relajacin.
%Sin esa parte el contenido original del artculo podria parecer insuficiente para
%una buena publicacion%

\section{Remarks and conclusions}\label{sec:disscussion}

In the literature, the Boltzmann's $H$-theorem is proved using a spatially homogeneous distribution function. 
Here, we considered inhomogeneities in the system through the division of the volume $V$ in cells 
and the non-homogeneous distribution function assumption in the total volume. 
We have shown that the Boltzmann's $H$-theorem still holds when the system includes 
inhomogeneities, of course assuming that the local equilibrium hypothesis holds. 
We demostrated the classical Boltzmann's $H$-theorem without using explicity the Boltzmann's transport equation. 
Instead of that, it needed to specify the behavior of the deviation $g$ and its time derivative. 
\\
In the quantum version, we used the same assumptions to prove the $H$-theorem, those were the 
behavior of the deviation $g_{nM}$ and its derivative, the local equilibrium hypothesis, and the 
non-homogeneous distribution function assumption. 
We have corroborated the correctness of the $f$ expression in equilibrium recovering 
the MB, FD and BE distributions and in recovering the expression of the entropy of an ideal gas. 
We demostrated as well that the functional $H$ leads to the proper fulfilling of the quantum classical correspondence.
\\
\\
We also have shown that variational entropy is proportional to the Boltzmann's $H$-functional in the non-degenerated limit. 
This is a straightfowars conclusion. One may infer from the expression for the Boltzmann's $H$-functional in equilibrium that
\begin{equation}
    H_{B}=-\frac{S}{Vk},
\end{equation}
where $S$ is the entropy of the system, and $k$ is the Boltzmann's constant, we can identify that variational entropy is proportional to the entropy of the system.\\
\begin{equation}
    \Ss\propto S.
\end{equation}
It is important to remark that we find that $\frac{d\mathcal{H}'}{dt}\leq 0$ and $\frac{d\mathcal{H}}{dt}\leq 0$ 
setting the expecting behavior of the deviation $g$. 
This correct behavior results from the molecular chaos hypothesis in the classical case and 
the random a priori phases hypothesis in the quantum case. 
Specifically those hypothesis are included in the assumed behavior of $\dot{g}_{nJ}^{+}$ and $\dot{g}_{nJ}^{-}$. 
Those time derivatives correspond to the expected behavior in a meantime, that 
is, $\dot{g}_{nJ}^{+}$ always decreases and $\dot{g}_{nJ}^{-}$ always increases.\\ 
According to the classical $H$-theorem, if the molecular chaos hypothesis holds, 
although within small times of the order of the \textit{relaxation time}, as a consequence of the 
stochastic nature of the collisions, $H$ function evolves through a sequential series of 
molecular chaos and non-molecular chaos, but a larger times $\frac{dH}{dt}\leq 0$.\\
In analogy with this, we see that in our procedure this behavior is introduced by 
$\dot{g}_{nJ}^{+}$ and $\dot{g}_{nJ}^{-}$ to obtain the expected behavior in a relaxation time. 
So, molecular chaos hypothesis and a random a priori phases hypothesis are implicit
 in the expected behavior of the deviation from the equilibrium of the $H$ function such that 
$\frac{d\mathcal{H}'}{dt}\leq 0$ and $\frac{d\mathcal{H}}{dt}\leq 0$.
\\
\\
In summaty, we propose a variational procedure to demostrate the classical and quantum $H$ theorem, that
make possible to describe at a mesoscopic local view (cell-scale), the time evolution 
of an out-of-equilibrium, spatially non-homogeneous, system to the equilibrium condition.
In principle, this approach would make possible investigate the transport phenomena inherent to the
equilibrium process, starting from an arbitrarily far-from-equilibrium initial condition.

%-----------------------------------------------

