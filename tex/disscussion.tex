\section{Comments and remarks}\label{sec:disscussion}

The demonstration of the classical $H$-theorem usually starts by assuming that the gas,
despite being initially out of equilibrium, can be described by a
spatially homogeneous distribution function. 
Subsequently, the time-evolution of the system occurs in such a manner that $dH/dt \leq 0$. 
Therefore, this approach does not cover the evolution to equilibrium of systems with spatial inhomogeneities.
To address this issue, in this article we
proposed a framework that may be useful to describe the time-evolution of systems
initially non-homogeneous. To this end, we divided the
system into small cells, so as to conceive a system wherein the local equilibrium
hypothesis is valid in each cell, but in such a manner that the total system is not homogeneous. Systems
that satisfy the previous conditions will evolve towards equilibrium, and the evolution
occurs according to $d\mathcal{H}'/dt\leq0$, Eq.~(\ref{eq:cHdef}),
and $d\mathcal{H}/dt\leq0$, Eq.~(\ref{eq:qHdef}), for classical and quantum gases, respectively.
Consequently, this approach can be considered as an extension of the $H$-theorem
for more realistic out-of-equilibrium systems.

The classical and quantum $H$-functionals, $\mathcal{H}'$ and $\mathcal{H}$, respectively,
correctly recover the most-probable distribution
functions both in out-of-equilibrium states (locally) and when the system attains the global
equilibrium state. The relaxation process of the system is described by
monotonic functions that account for deviations from the global equilibrium.

It is clear that for describing the relaxation process of a concrete system,
it is necessary to know, at least with some approximation level, the specific forms of the
monotonic functions $g'_M$ and $g_{nM}$, for classical and quantum systems.
Whereas the complete analysis of these functions is out of the scope of the present work,
some of their properties can be foreseen, \textit{e.g.} they must
be consistent both with the system relaxation times and with the mechanisms of 
energy transfer between cells.

An important aspect of the framework proposed in this work is related to the entropy of systems
out-of-equlibrium. Because the functionals $\mathcal{H}'$ and $\mathcal{H}$ can be related to
the entropy of dilute gases, either classical or quantum, the fact that these functionals are defined
over a system divided into cells enables their use for defining the entropy
of out-of-equilibrium systems, other than dilute gases. In particular, and 
derived from our previous work (\textit{e.g.} \cite{bib:nicolas2020,bib:nicolas2016}), the
$\mathcal H'$ and $\mathcal H$ functionals may serve to describe the entropy,
as well as the entropy generation, occurring during the
growth of complex physical systems, such as fractals. Possibly, studying these systems might also
shed light on the explicit functional form of $g'_M$ and $g_{nM}$.

In summary, we proposed a variational procedure to demonstrate the classical and
quantum $H$-theorems, which made possible to describe, at a mesoscopic local view
(cell-scale), the time-evolution of an out-of-equilibrium and spatially
non-homogeneous system, towards the equilibrium condition. In principle, this
approach would render possible to investigate the transport phenomena, inherent to the
equilibration process, occurring in a system that started from an spatially inhomogeneous
out-of-equilibrium initial condition. 

%-----------------------------------------------

