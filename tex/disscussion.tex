%--------------------------------------
% Me parece que sera necesario incluir la discusin sobre los procesos de relajacin.
%Sin esa parte el contenido original del artculo podria parecer insuficiente para
%una buena publicacion%

\color{red}
\section{Discussions and remarks}\label{sec:disscussion}

Traditionally, the proof of the classical $H$-theorem involves assuming that the
distribution function is spatially homogeneous, and the proof starts with
considering particular systems that are out of equilibrium, but in such a manner
that their distribution functions are still homogeneous. This out-of-equilibrium systems
are conceived considering the micro-states of the system, and the $H$-theorem determines
the microscopic evolution of the system towards a state with the most-probable distribution function,
given a thermodynamic state. 
This approach does not cover systems with spatial inhomogeneities.
To address this issue, we
proposed a framework that may be useful to describe the time-evolution of systems
initially non-homogeneous. For this, we divided the
system into small cells, so as to conceive a system wherein we accept the local equilibrium
hypothesis to be valid for each cell, but in such a manner that the global system is not homogeneous. Systems
that satisfy the previous conditions will evolve towards global equilibrium, and the evolution is
determined by the relations $d\mathcal{H}'/dt\leq0$, Eq.~(\ref{eq:cHdef}),
and $d\mathcal{H}/dt\leq0$, Eq.~(\ref{eq:qHdef}), for classical and quantum gases, respectively.
In some aspects, this can be considered as an extension or as a potential equivalent of the $H$ theorem
for systems out-of-equilibrium in the spatial sense, \textit{i.e.} systems whose thermodynamic
variables are not spatially homogeneous.

The classical and quantum $H$-functionals, $\mathcal{H}'$ and $\mathcal{H}$, respectively,
correctly recover the most-probable distribution
functions both in out-of-equilibrium states (locally) and when the system attains the global
equilibrium state. This enables the use of all the machinery of thermodynamics, including the
proper definition of state variables as functions of time for each cell. In addition, our approach
does not require to know the specific mechanisms by which energy or particles are transferred from one
cell to its neighbors. Instead, the time-evolution from the out-of-equilibrium to equilibrium states are
governed by monotonic functions that account for deviations from the global equilibrium.

The specific forms of the functions $g'_M$ and $g_{nM}$ for classical and quantum systems will be the
subject of our future work. However, some of their properties can be foreseen, \textit{e.g.} they must
be consistent with the system relaxation times, they must be consistent with the mechanisms of 
energy transfer between open systems, there must be related to the cell thermodynamic variables, etc.

An important aspect of the framework proposed in this work is related to the entropy of systems
out-of-equlibrium. Because the functionals $\mathcal{H}'$ and $\mathcal{H}$ can be related to
the entropy of diluted gases, either classical or quantum, the fact that these functionals are defined
over a system divided into cells enables their use for defining the entropy
of other out-of-equilibrium systems. In particular, and 
derived from our previous work (\textit{e.g.} \cite{bib:nicolas2020,bib:nicolas2016}), the
$\mathcal H'$ and $\mathcal H$ functionals may serve to describe the entropy,
as well as the entropy generation, occurring during the
growth of complex physical systems, such as fractals. Possibly, studying these systems might also
shed light on the functional form of $g'_M$ and $g_{nM}$.

In conclusion, we have proposed a framework for studying classical and quantum out-of-equilibrium
diluted gases, whose initial
state is inhomogeneous in nature. This framework consists, briefly, of dividing the system into
a set of small cells, and assuming that each cell is in equilibrium. Subsequently,
$H$-functionals, $\mathcal H$, can be defined
such that the time evolution towards equilibrium resembles the Boltzmann $H$-theorem: $d\mathcal H/dt\leq0$,
and the equality condition is satisfied when the system attains the global equlibrium.
The distribution functions,
stemming from the $\mathcal H$ functionals, at any time, $t$, are the respective equilibrium distribution functions,
which can be Maxwell-Boltzmann, Fermi-Dirac or Bose-Einstein, depending on the classical or quantum nature
of the gas under consideration. Furthermore,
the distribution function of each cell evolves so as to match the global distribution function in equilibrium.
A proof of the relation $d\mathcal H/dt\leq0$ is performed through the variational method,
the functional form of the distribution
functions, in terms of a monotonic function $g$, are defined, and some of the properties of $g$ are discussed.
The functionals $\mathcal H$ can be associated with the entropy of the system under consideration, hence we
refer to this entropy as the variational entropy of the system. Finally,
a potential application of this framework is outlined briefly. 



\section{Remarks and conclusions}\label{sec:disscussion}

In the literature, the Boltzmann's $H$-theorem is proved using a spatially homogeneous distribution function. 
Here, we considered inhomogeneities in the system through the division of the volume $V$ in cells 
and the non-homogeneous distribution function assumption in the total volume. 
We have shown that the Boltzmann's $H$-theorem still holds when the system includes 
inhomogeneities, of course assuming that the local equilibrium hypothesis holds. 
We demostrated the classical Boltzmann's $H$-theorem without using explicity the Boltzmann's transport equation. 
Instead of that, it needed to specify the behavior of the deviation $g$ and its time derivative. 
\\
In the quantum version, we used the same assumptions to prove the $H$-theorem, those were the 
behavior of the deviation $g_{nM}$ and its derivative, the local equilibrium hypothesis, and the 
non-homogeneous distribution function assumption. 
We have corroborated the correctness of the $f$ expression in equilibrium recovering 
the MB, FD and BE distributions and in recovering the expression of the entropy of an ideal gas. 
We demostrated as well that the functional $H$ leads to the proper fulfilling of the quantum classical correspondence.
\\
\\
We also have shown that variational entropy is proportional to the Boltzmann's $H$-functional in the non-degenerated limit. 
This is a straightfowars conclusion. One may infer from the expression for the Boltzmann's $H$-functional in equilibrium that
\begin{equation}
    H_{B}=-\frac{S}{Vk},
\end{equation}
where $S$ is the entropy of the system, and $k$ is the Boltzmann's constant, we can identify that variational entropy is proportional to the entropy of the system.\\
\begin{equation}
    \Ss\propto S.
\end{equation}
It is important to remark that we find that $\frac{d\mathcal{H}'}{dt}\leq 0$ and $\frac{d\mathcal{H}}{dt}\leq 0$ 
setting the expecting behavior of the deviation $g$. 
This correct behavior results from the molecular chaos hypothesis in the classical case and 
the random a priori phases hypothesis in the quantum case. 
Specifically those hypothesis are included in the assumed behavior of $\dot{g}_{nJ}^{+}$ and $\dot{g}_{nJ}^{-}$. 
Those time derivatives correspond to the expected behavior in a meantime, that 
is, $\dot{g}_{nJ}^{+}$ always decreases and $\dot{g}_{nJ}^{-}$ always increases.\\ 
According to the classical $H$-theorem, if the molecular chaos hypothesis holds, 
although within small times of the order of the \textit{relaxation time}, as a consequence of the 
stochastic nature of the collisions, $H$ function evolves through a sequential series of 
molecular chaos and non-molecular chaos, but a larger times $\frac{dH}{dt}\leq 0$.\\
In analogy with this, we see that in our procedure this behavior is introduced by 
$\dot{g}_{nJ}^{+}$ and $\dot{g}_{nJ}^{-}$ to obtain the expected behavior in a relaxation time. 
So, molecular chaos hypothesis and a random a priori phases hypothesis are implicit
 in the expected behavior of the deviation from the equilibrium of the $H$ function such that 
$\frac{d\mathcal{H}'}{dt}\leq 0$ and $\frac{d\mathcal{H}}{dt}\leq 0$.
\\
\\
In summaty, we propose a variational procedure to demostrate the classical and quantum $H$ theorem, that
make possible to describe at a mesoscopic local view (cell-scale), the time evolution 
of an out-of-equilibrium, spatially non-homogeneous, system to the equilibrium condition.
In principle, this approach would make possible investigate the transport phenomena inherent to the
equilibrium process, starting from an arbitrarily far-from-equilibrium initial condition.

%-----------------------------------------------

