%--------------------------------------
% Me parece que sera necesario incluir la discusin sobre los procesos de relajacin.
%Sin esa parte el contenido original del artculo podria parecer insuficiente para
%una buena publicacion%

\section{Discussions and remarks}\label{sec:disscussion}

In the literature, the Boltzmann's $H$-theorem is proved using a spatially homogeneous distribution function. In this work, we used the local equilibrium hypothesis and as a consequence of this, it was necessary to include the homogeneous distribution function hypothesis. However, we included inhomogeneities in the system through the division of the volume $V$ in cells and the non-homogeneous distribution function assumption in the total volume. Moreover, the Boltzmann's $H$-theorem still holds when the system includes inhomogeneities, only if the local equilibrium hypothesis holds. We can include the fact that we proved the Boltzmann's $H$-theorem without using the Boltzmann's transport equation. It was only necessary to specify the behavior of the deviation $g$ and its derivative in time. Also, we obtained the Boltzmann's transport equation in equilibrium from the Boltzmann's $H$-functional through the variational method. Nevertheless, in order to obtain the complete Boltzmann's transport equation, it was necessary to include the collision term because of the variation of the $H$-functional is not zero when the system is out of equilibrium. This is a consequence of the $H$-theorem.\\
\\
In the quantum version, we used the same assumptions to prove the $H$-theorem, those were the behavior of the deviation $g_{nM}$ and its derivative, the local equilibrium hypothesis, and the non-homogeneous distribution function assumption. With the proof of the quantum version of the $H$-theorem, we can say that a quantum gas inside a volume $V$ that is out of equilibrium with inhomogeneities, in the first-order approximation the system evolves to the equilibrium state as long as the local equilibrium hypothesis holds. In addition to this, the system is in the equilibrium state when the derivative of the variational entropy with respect to time is equal to zero, and as a consequence of this, the distribution function corresponds to the Bose-Einstein or Fermi-Dirac distribution.\\
On the other hand, the equation of motion obtained from the variational entropy could be seen as a quantum transport equation. However, in order to develop the transport phenomena (in analogy to theoretical work developed by Boltzmann) is necessary to find an expression for the collision term. This problem may be future work.\\
\\
We also find that variational entropy is proportional to the Boltzmann's $H$-functional in the limit of high energy levels. This conclusion is trivial if we compare (\ref{h-quantic4}) with (\ref{reduce-h}). Moreover, if we see the expression for the Boltzmann's $H$-functional in equilibrium
\begin{equation}
    H_{B}=-\frac{S}{Vk},
\end{equation}
where $S$ is the entropy of the system, and $k$ is the Boltzmann's constant, we can identify that variational entropy is proportional to the entropy of the system.\\
\begin{equation}
    \Ss\propto S.
\end{equation}
\textcolor{red}{Besides, we find an expression for the entropy density (\ref{H-entropy}). This expression will be useful to describe quantum gases in terms of entropy.\\
It is important to remark that we find that $\frac{dH'}{dt}\leq 0$ and $\frac{d\Ss}{dt}\leq 0$ setting the expecting behavior of the deviation $g$. This correct behavior involves the molecular chaos hypothesis in the classical case and the random a priori phases hypothesis in the quantum case. Specifically those hypothesis are included in $\dot{g}_{nJ}^{+}$ and $\dot{g}_{nJ}^{-}$. Those time derivatives correspond to the expected behavior in a meantime, that is, $\dot{g}_{nJ}^{+}$ always decreases and $\dot{g}_{nJ}^{-}$ always increases.\\ 
According to the classical analysis of the $H$-theorem, the expected behavior of the distribution function is presented in the system if the molecular chaos hypothesis holds, but in a meantime, called \textit{relaxation time}, the distribution function behaves as we expect, such as the distribution function evolves to the equilibrium state keeping that $\frac{dH}{dt}\leq 0$.\\
In the same way as the previous statement, we proposed that $\dot{g}_{nJ}^{+}$ and $\dot{g}_{nJ}^{-}$ have the expected behavior in a relaxation time. Also molecular chaos hypothesis and a random a priori phases hypothesis are implicit in the expected behavior of the deviation such as $\frac{dH'}{dt}\leq 0$ and $\frac{d\Ss}{dt}\leq 0$.}

%-----------------------------------------------

