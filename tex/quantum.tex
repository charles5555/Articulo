\section{Quantum scheme}\label{sec:quantum}

The classical $H$-theorem is even to these days one of the pillars whereon the classical statistical
mechanics is founded. Unfortunately, regardless of several attempts have been made
\commr{[CITE Q. H-THEOREM EXAMPLES]}, 
the generality of the classical $H$-theorem has no quantum match. In this section, we shall
propose and analize an alternative quantum $H$-functional using the variational method.
We start by briefly outline a typical textbook demonstration of the quantum $H$-theorem
\cite{bib:tolman}, followed by the analysis of our proposed $H$-functional.

\subsection{$H$-theorem and the Fermi-Dirac and Bose-Einstein distribution functions}

Consider a diluted gas of $N$ non-interacting quantum particles (either bosons or fermions),
contained by a vessel of volume
$V$, at some temperature $T$, and total energy E. Starting from the Boltzmann definition
of entropy, the quantum $H$ functional is
%
\begin{equation}
   H_T=-\ln G,
\end{equation}
%
where $G$ is the total number of possible quantum states of the gas that satisfy the
above conditions \cite{bib:tolman}.
The quantum $H$-theorem can be demostrated as follows.  $G$ can be divided into groups of neighboring states,
$g_k$, and a certain occupation numbers, $n_k$, can be associated with each of these
groups. Thus, the above functional takes the form:
%
\begin{equation}\label{eq:quantumh}
    H_T=\sum_i n_i \ln n_i -(n_i\pm g_i)\ln (g_i \pm n_i)\pm g_i\ln g_i, 
\end{equation}
%
where the upper and lower signs are for bosons and fermions, respectively.
Thus the time derivative of Eq.~(\ref{eq:quantumh}) is
%
\begin{equation}\label{eq:quantumdHdt}
\frac{dH_T}{dt}=\sum_{\kappa}\left[\ln n_{\kappa}-\ln\left(g_{\kappa}\pm n_{\kappa}\right)\right]
\frac{dn_{\kappa}}{dt}.
\end{equation}
%

Arguing that the energy exchange between particles is dominated by collisions between particles,
and using perturbation theory, the rate of change in the number of particles in a group $\kappa$ is
%
\begin{eqnarray}\label{eq:changen}
    \frac{d n_{\kappa}}{dt}&=&-\sum_{\lambda,(\mu \nu)}A_{\kappa\lambda,\mu\nu} n_{\kappa}n_{\lambda}(g_{\mu}\pm n_{\mu})(g_{\nu}\pm n_{\nu})\nonumber \\
    &&+\sum_{\lambda,(\mu \nu)}A_{\mu\nu,\kappa\lambda} n_{\mu}n_{\nu}(g_{\kappa}\pm n_{\kappa})(g_{\lambda}\pm n_{\lambda}).
\end{eqnarray}
%
Here $A_{\kappa\lambda,\mu\nu}n_{\kappa}n_{\lambda}(g_{\mu}\pm n_{\mu})(g_{\nu}\pm n_{\nu})$ is the
expected number of collisions per unit time, in which two particles will be moved from groups
$\kappa$, $\lambda$ to $\mu$, $\nu$, and the tensor $A_{\kappa\lambda,\mu\nu}n_{\kappa}$ is given by
%
\begin{equation}\label{eq:amnkldef}
  A_{\kappa\lambda,\mu\nu}=\frac{4\pi^{2}}{h}\frac{|I_1\pm I_2|^2}{\Delta \epsilon}.
\end{equation}
%
In Eq.~(\ref{eq:amnkldef}), $\Delta\epsilon$ is the net energy change occuring during the
collision and $|I_1-I_2|^2=|V_{mn,kl}|^2$ where
$V_{mn,kl}$ is the element of the transition matrix of a binary collision.
It is important to remark that in deriving Eq.~(\ref{eq:changen}), the equal a priori
probabilities and the random a priori phases hypotheses were surmised, and that 
the random a priori phases hypothesis can be
considered an analogous to the molecular chaos hypothesis
\cite{bib:das2018}.

Substituting Eq.~(\ref{eq:changen}) into Eq.~(\ref{eq:quantumdHdt}), and performing
the algebra, it can be proven that
%
\begin{equation}
    \frac{dH_T}{dt}\leq 0.
\end{equation}
%

In equilibrium (at $t\to\infty$), $dn_{\kappa}/dt=0$, hence from Eq.~(\ref{eq:changen})
%
\begin{equation}\label{eq:boseiferdirdistfnctsrc}
    \ln \frac{n_{\kappa}}{g_{\kappa}\pm n_{\kappa}}+\ln \frac{n_{\lambda}}{g_{\lambda}\pm n_{\lambda}}=\ln \frac{n_{\mu}}{g_{\mu}\pm n_{\nu}}+\ln \frac{n_{\nu}}{g_{\nu}\pm n_{\nu}}.
\end{equation}
%
Considering that the energy is conserved during the collsion, the Bose-Einstein or Fermi-Dirac distribution
functions can be recovered from Eq.~(\ref{eq:boseiferdirdistfnctsrc}):
%
\begin{equation}
   n_{\kappa}=\frac{g_{\kappa}}{\exp(\alpha+\beta\epsilon_{\kappa}\mp1)}.
\end{equation}
%

\textit{I.e.}, in equilibrium $dH_T/dt=0$ and the distribution function obtained from
Eq.~(\ref{eq:quantumh}) is the expected distribution function.

%----------------------------------------------
\begin{comment}
In addition, Tolman also showed that (\ref{eq:quantumh}) can be reduced, in the
high-energy limit, to
%
\begin{equation}
    H_T = \sum_{\kappa} (n_{\kappa} \ln n_{\kappa} - n_{\kappa} \ln g_{\kappa}). \label{reduce-h}
\end{equation}
%
This expression also can be obtained also from the Boltzmann $H$-functional through defining
%
\begin{equation}
    f=\frac{n_{\kappa}}{ g_{\kappa}}.
\end{equation}
\end{comment}
%
%--------------------------



\subsection{Out of equilibrium, non-homogeneous quantum systems}

Consider a diluted gas enclosed by a perfectly isolated vessel of volume $V$, with
total energy $E$, and total number of quantum particles $N$, which can be free fermions or
bosons. For our purposes, the volume $V$ is divided into $K$ small cells, each of which
has a constant volume $\delta V_M=V/K$ ($M=1,\,\dots\,,K$), a
local temperature $T_M$, a local energy $\epsilon_M$,
a local number of particles $\mathcal{N}_M$, and a local set of distribution functions,
$\{f_{Mn}(t)\}$. Hereafter we will use the short-hand notation:
%
\begin{equation}
   f_{Mn}(t)\equiv f(\vec r_M,\epsilon_{n},t),
\end{equation}
%
where $\vec r_M$ is the radius vector pointing at the center of the $M$-th cell.
$f_{Mn}(t)$ represent the number
of particles contained in the $M$-th cell that occupy the energy level $\epsilon_n$ at
time $t$. Since the particles are considered to be free, the energy levels should not
depend on the cell properties, \textit{i.e.} the energy spectrum, $\{\epsilon_n\}$, is the same for all
cells; thus there is no need to label $\epsilon_n$ with an index $M$.

We propose the following functional as an alternative $H$-functional
for quantum non-homogeneous diluted gases:
%
\begin{eqnarray}
    \mathcal{H} (t)&=&\sum_{M=1}^{K} \sum_{n} \bigg[ f_{Mn}(t) \ln f_{Mn}(t)\bigg.\nonumber \\
    &&\bigg.\pm \Big(1 \mp f_{Mn}(t)) \ln (1 \mp f_{Mn}(t)\Big) \Big]   \delta V_M\label{entropy}.
\end{eqnarray}
%
Here, the upper and lower sign refer to bosons and fermions respectively.
In addition, when needed, each cell will have an associated local chemical potential,
$\alpha_M$, and its local $H$-functional, which is defined by:
%
\begin{equation}\label{eq:qHMdef}
   \mathcal{H}_M(t)=\sum_{n} \bigg[ f_{Mn}(t) \ln f_{Mn}(t)
   \pm \Big(1 \mp f_{Mn}(t)\Big) \ln \Big(1 \mp f_{Mn}(t)\Big) \bigg] \de V_M.
\end{equation}
%
Therefore, $\mathcal{N}_M$ and $\mathcal{E}_M$ as functions of time are given by:
%
\begin{subequations}
\begin{equation}
    {\mathcal{N}}_M(t) = \sum_{n}f_{Mn}(t) \delta V_M
\end{equation}
%
and
%
\begin{equation}
\mathcal{E}_M(t) = \sum_{n}f_{Mn}(t)\epsilon_{n} \de V_M
\end{equation}
\end{subequations}
%
which are, in addtion, constrained by the global microcanonical restrictions
%
\begin{subequations}
\begin{equation}\label{eq:qGlobRestN}
    \sum_{M=1}^{K}\left[\sum_{n}f_{Mn}(t)\right]\delta  V_M
    =\sum_{M=1}^{K}\mathcal{N}_{M}(t)\delta V_M=N
\end{equation}
%
and
%
\begin{equation}\label{eq:qGlobRestE}
    \sum_{M=1}^{K}\left[\sum_{n}f_{Mn}(t)\epsilon_{n}\right]\delta V_M
    =\sum_{M=1}^{K}\mathcal{E}_M(t)\delta V_M=E. 
\end{equation}
\end{subequations}
%

Applying the variational method to $\mathcal{H}$, and using the using the
Lagrange multipliers
$\{\alpha_M\}$ and $\{\beta_M\}$ it is straightforward to obtain
(see also text around Eq.~(\ref{eq:deltaHpdeltafpj})):
%
\begin{equation}\label{eq:relation}
\ln \left(\frac{1\mp f_{Mn}(t)}{f_{Mn}(t)} \right)=-\alpha_M(t)-\beta_M(t) \epsilon_{n},
\end{equation}
%
and solving for $f_{Mn}(t)$ yields
%
\begin{equation}\label{eq:qfMn}
f_{Mn}(t)=\frac{1}{\exp\big(-\alpha_M(t)-\beta_M(t) \epsilon_{n}\big)\pm 1}.
\end{equation}

%============================================================
\subsubsection{Properties of $\mathcal{H}$ in equilibrium}
%============================================================
If the system is in equilibrium (at $t\to\infty$), the local number of particles and the local
energy do not depend on the cell number and they should be homogeneous, this is \commr{[This needs a rationale\dots]}
%
\begin{subequations}
\begin{equation}
   {\mathcal{N}}_M(t\to\infty)\equiv \bar{\mathcal{N}}=N/K
\end{equation}
%
and
%
\begin{equation}
	{\mathcal{E}}_M(t\to\infty)\equiv \bar{\mathcal{E}}=E/K.
\end{equation}
%
\end{subequations}
%

\color{blue}

%
\commr{We can obtain
from the variational procedure that
%
\begin{equation}
    \alpha_M\propto \frac{\partial \ln \Omega_M}{\partial \mathcal{N}_M}, \ \ \ \beta_M\propto \frac{\partial \ln \Omega_M}{\partial \mathcal{E}_M},\label{multipliers}
\end{equation}
%
[What is $\Omega_M$?]
}

where we used that the $H$-functional in equilibrium is proportional to the
entropy of the system.
In other hand, in the expression (\ref{multipliers}), we expect that the
variation of the logarithm with the number of particles by cells is indifferent
from a particular cell. With this, we conclude 
%
\begin{equation}
    \alpha_M=\alpha; \ \ \ \ \beta_M=\beta.
\end{equation}
%
In equilibrium, the energy and the Lagrange multiplier do not depend on the
cell number and time, therefore
%
\begin{equation}
    \bar f_{Mn}(\epsilon_{n},t)=\bar f_n(\epsilon_{n}) =\frac{1}{e^{-\alpha-\beta \epsilon_n}\pm 1}.
\end{equation}
%
If one compares those multipliers with the mean energy and the mean particle
number from the statistical mechanic's results, we obtain
%
\begin{equation}
    \alpha=\frac{\mu}{kT}\equiv \bar{\alpha}; \ \ \ \ \beta=-\frac{1}{kT}\equiv \bar{\beta},
\end{equation}
%
and finally
%
\begin{eqnarray}
    \bar{f}_{n}(\epsilon_{n})&=&\frac{1}{e^{(\frac{{\epsilon_n}-\bar{\mu}}{kT})}\pm 1}\equiv \bar{f}_{n}.
\end{eqnarray}
%
We proved, when the system is in equilibrium, the Lagrange multipliers don't
depend on the cell number. In other words, Lagrange multipliers are homogeneous
in the system.

We can calculate the entropy from an ideal system, in other words, if one
substitute (\ref{distributionequilibrium}) to (\ref{entropy}) we obtain
%
\begin{eqnarray}
      \mathcal{H}(t)&=&\sum_{M=1}^{K} \sum_n  \left[\left(\frac{1}{e^{-\bar{\alpha}-\bar{\beta}\epsilon_{n}}\pm 1} \right)\ln \left(\frac{1}{e^{-\bar{\alpha}-\bar{\beta}\epsilon_{n}}} \right) \right]\nonumber \\
      &&\pm  \ln \left[\prod_{M=1}^{K} \prod_{n}\left(1 \mp \frac{1}{e^{-\bar{\alpha}-\bar{\beta}\epsilon_{n}}\pm 1} \right) \right] \de V_M \delta \epsilon_n\label{H-entropy}.
  \end{eqnarray}
%
This result allows us to calculate the entropy of a quantum ideal gases. This
entropy could be used as a thermodynamic variable to describe those systems.
 
 
%\subsection{The quantum analogous of the Boltzmann's $H$-theorem}
We rewrite the \textit{variational entropy} (\ref{entropy}) as
%
\begin{equation}
    \mathcal{H}(t)=\sum_{M=1}^{K} \sum_{n} \left[ f_{Mn} \ln f_{Mn} \pm (1 \mp f_{Mn}) \ln (1 \mp f_{Mn}) \right] \de V_M \delta \epsilon_n \label{entropy2},
\end{equation}
%
where, for simplicity, we have defined $f(\epsilon_{n},t)\equiv f_{Mn}$.

This distribution function must satisfy the local equilibrium hypothesis in
each cell ($f_{nM}$ must be spatial-homogeneous for each $M$), the
non-homogeneous distribution function assumption in the total volume ($f_{nM}$
must be different among all cells), and the following restrictions
%
\begin{eqnarray}
        \sum_{n}f_{Mn} \delta \epsilon_n=\bar{\mathcal{N}}+\Delta_M(t); \ \ \ \ \sum_{n}\epsilon_{n}f_{Mn} \delta \epsilon_n=\bar{\mathcal{E}}+ \delta_M(t), \label{restrictionoutside}
  \end{eqnarray}
%
where $\bar {\mathcal{N}}$ and $\bar{\mathcal{E}}$ are the local particle
number and the local energy in equilibrium
%
\begin{equation}
      \bar{\mathcal{N}}= \sum_n \bar{f}_n \delta \epsilon_n; \ \ \ \ \bar{\mathcal{E}}= \sum_n \epsilon_n\bar{f}_n \delta \epsilon_n,
  \end{equation}
%
and $\Delta_M,\delta_M$ could be seen as a deviation from $\bar{\mathcal{N}}$
and $\bar{\mathcal{E}}$ respectively, with $\Delta_M(t)\ll \bar{\mathcal{N}}$
and $\delta_M(t) \ll \bar{\mathcal{E}}$. The previous conditions and
restrictions are the quantum analogous case of the classical case. Also, this
system will suffer, as in the classical case, a \textit{relaxation process}.

  The quantities $\Delta_M,\delta_M$ are sufficiently big to be different from
fluctuations in the system but sufficiently small to consider the system not
far from the equilibrium state, as in the classical case. 

If we perform the derivative of the variational entropy with respect to time,
we obtain
%
\begin{equation}
   \frac{d \mathcal{H} (t)}{dt}= \sum_n \sum_{M=1}^{K} \dot{f}_{nM}(t)\ln \left[ \frac{f_{nM}(t)}{1\mp f_{nM}(t)} \right] \de V_M \delta \epsilon_n.\label{deltaH}
\end{equation}
%
Using the first-order approximation (\ref{firstorder}), we get
%
\begin{eqnarray}
    \frac{d\mathcal{H} (t)}{dt}&=&\sum_n \sum_{M=1}^{K} \bar{f}_{n}\ln \left[ \frac{\bar{f}_{n}(1+g_{nM})}{1\mp \bar{f}_{n} (1+ g_{nM})} \right]\dot{g}_{nM} \de V_M\delta \epsilon_n \nonumber \\
    &=&\sum_n \sum_{M=1}^{K} \bar{f}_n \left \{ \ln [\bar{f}_n+\bar{f}_n g_{nM}]\dot{g}_{nM}-\ln [1\mp\bar{f}_n\mp\bar{f}_n g_{nM}]\dot{g}_{nM}  \right \}\de V_M \delta \epsilon_n.\nonumber \\
    \label{cambioH1}
\end{eqnarray}
%
It's necessary to remark that, in the case of fermions
%
\begin{eqnarray}
   1-\bar f_n -\bar f_n g_{nM}>0 \ \ \Rightarrow \ \ \frac{1}{\bar f_n}>1+g_{nM}. \label{fermionrestriction}
\end{eqnarray}
%
The previous expression establishes that the value of $g_{nM}$ is determined by
$\bar f_n$. Furthermore, whether $\bar f_{n}=1$, this is, all lower energy
levels are occupied, then the system will not have inhomogeneities because of
the exclusion principle, and as a consequence of this,  $g_{nM}=0$. This result
corresponds to the system in the zero temperature condition. We can observe
that in this condition, (\ref{fermionrestriction}) is violated. Due to this,
the zero temperature condition is excluded.


We can observe in the case of bosons, if $\bar{f}_n \gg \bar{f}_n |g_{nM}|$
then $1+\bar{f}_n \gg \bar{f}_n |g_{nM}|$. In fermions, in the case of a
non-extremely degenerated condition, it is true that $1-\bar{f}_n \gg \bar{f}_n
|g_{nM}|$. With those relations, we can approximate the logarithm functions in
(\ref{cambioH1}) to their first-order Taylor series around $\bar f_n g_{nM}=0$
%
\begin{equation}
    \ln [\bar{f}_n+\bar{f}_n g_{nM}] \approx \ln [\bar{f}_n]+ g_{nM}; \ \ \ \ \ln[1\mp\bar{f}_n\mp\bar{f}_n g_{nM}] \approx \ln[1\mp\bar{f}_n]\mp\frac{\bar{f}_n}{1\mp\bar{f}_{n}} g_{nM}. \label{lnapproximation}
\end{equation}
%
With the previous approximation, (\ref{cambioH1}) will be
%
\begin{eqnarray}
    \frac{d\mathcal{H}}{dt}&=&\sum_n \sum_{M=1}^{K} \bar{f}_n\left \{ (\ln \bar{f}_n+ g_{nM})\dot{g}_{nM}\right\} \de V_M \delta \epsilon_n \nonumber \\
    &&-\sum_{n}\sum_{M=1}^{K}\bar f_n\left\{ \left( \ln[1\mp\bar{f}_n]\mp \left[\frac{\bar{f}_n}{1\mp\bar{f}_n} \right] g_{nM}\right)\dot{g}_{nM} \right \}\de V_M \delta \epsilon_n.\label{cambioH2}
\end{eqnarray}
%
Making use of the expression (\ref{relation}),
(\ref{cambioH2}) casts into
%
\begin{eqnarray}
    \frac{d\mathcal{H}}{dt}&=&\sum_n \sum_{M=1}^{K} \bar{f}_n\left \{ (\bar{\alpha}+\bar{\beta}{\epsilon}_n)\dot{g}_{nM}+ g_{nM}\left(1\pm e^{\bar{\alpha}+\bar{\beta}{\epsilon}_n}\right)\dot{g}_{nM} \right \} \de V_M \delta \epsilon_n. \nonumber \\
    \label{cambioH3}
\end{eqnarray}
%
On the other hand, we obtain from the restrictions the following expressions
%
\begin{eqnarray}
    &&\sum_n \bar{f}_n g_{nM} \delta \epsilon_n=\Delta_M(t) \ \  \Rightarrow \ \  \sum_n \bar{f}_n \dot{g}_{nM} \delta \epsilon_n=\dot{\Delta}_M(t), \nonumber \\
    &&\sum_n  \bar{f}_n g_{nM}\epsilon_n \delta \epsilon_n=\delta_M(t) \ \  \Rightarrow \ \  \sum_n \bar{f}_n \dot{g}_{nM}\epsilon_n \delta \epsilon_n=\dot{\delta}_M(t)
\end{eqnarray}
%
and as a consequence of $\sum_{M=1}^{K} \Delta_M(t) \de V_M  =\sum_{M=1}^{K} \delta_M(t) \de V_M =0$, we find
%
\begin{equation}
    \sum_{M=1}^{K} \dot{\Delta}_M(t) \de V_M =\sum_{M=1}^{K} \dot{\delta}_{M}(t) \de V_M=0.
\end{equation}
%
Substituting the previous expression to (\ref{cambioH3}) we obtain
%
\begin{eqnarray}
   \frac{d\mathcal{H}}{dt}&=&  \sum_n e^{\bar{\alpha}+\bar{\beta}\epsilon_n}\sum_M  g_{nM}\dot{g}_{nM} \de V_M\delta \epsilon_n. \nonumber \\ \label{cambioH4}
\end{eqnarray}
%
Now, the summation over $M$ will be expressed in two summations, such as in the
classical case
%
\begin{equation}
    \frac{d\mathcal{H}}{dt}=\sum_n  e^{\bar{\alpha}+\bar{\beta}\epsilon_n}\left(\sum_J ^{L} g^{+}_{nJ}\dot{g}^{+}_{nJ}\de V_J \delta \epsilon_n+\sum^{P}_J  g^{-}_{nJ}\dot{g}^{-}_{nJ} \de V_J\delta \epsilon_n \right), \label{cambioH5}
\end{equation}
%
where $L+P=K$. Besides, $\dot{g}^{+}_{J}$ represents the change on the
deviation on cells that have an excess of particles or energy while
$\dot{g}^{-}_{J}$  represents the change on the deviations on cells that have
missing particles or energy.
On the other hand, $g^{+}_{J}$  represents the value of the deviation on cells
that have an excess of particles or energy. In contrast, $g^{-}_{J}$ represents
the value of the deviation on cells that have missing particles or energy.

Also, on the one hand, $\dot{g}^{+}_{nJ}<0$ describes the loss of particles
and/or energy and so, $g^{+}_{nJ}>0$. On the other hand, $\dot{g}^{-}_{nJ}>0$
describes the gain of particles and/or energy and therefore $g^{-}_{nJ}<0$.

We sort the previous ideas in the following form
%
\begin{eqnarray}
   &&g^{+}_{nJ}=+|g^{+}_{nJ}|; \ \ \  \dot{g}^{+}_{nJ}=-|\dot{g}^{+}_{nJ}| \nonumber \\
   &&g^{-}_{nJ}=-|g^{-}_{nJ}|; \ \ \ \dot{g}^{-}_{nJ}=+|\dot{g}^{-}_{nJ}| \label{separacion},
\end{eqnarray}
%


and consequently, (\ref{cambioH5}) obtains the following form
%
\begin{equation}
    \frac{d\mathcal{H}}{dt}=-\sum_n  e^{\bar{\alpha}+\bar{\beta}\epsilon_n}\left(\sum_J ^{L} |g^{+}_{nJ}||\dot{g}^{+}_{nJ}|\de V_J \delta \epsilon_n+\sum^{P}_J  |g^{-}_{nJ}||\dot{g}^{-}_{nJ}| \de V_J \delta \epsilon_n \right), \label{cambioH6}
\end{equation}
%
and due to $e^{\bar{\alpha}+\bar{\beta}\epsilon_n}$ is always positive, then
$\frac{d\mathcal{H}}{dt}<0$. With this, we proved that any quantum ideal gas perturbed
(out of equilibrium but not so far from it), evolves such that
$\frac{d\mathcal{H}}{dt}<0$.

When the system is in equilibrium, $g_{nM}=0$ and in consequence of that, $\dot
g_{nM}=0$ and finally from (\ref{cambioH1}) $\frac{d\mathcal{H}}{dt}=0$. Then, we can
say that any ideal quantum gas perturbed always evolves to the equilibrium
state, such as the classical $H$-theorem. Then, we obtained the quantum version
of the Boltzmann's $H$-theorem including inhomogeneities in the system.

We obtained some results from the Statistical Mechanic in Quantum Mechanics
scheme. However, the correspondence principle is important to recover all
results from Classical Mechanics. This correspondence principle will be
discussed in the next section.

  %-----------------------------------------
 

%---------------------------------------------

