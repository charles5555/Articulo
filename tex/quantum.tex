\section{Quantum scheme}\label{sec:quantum}

The classical $H$-theorem is even to these days one of the pillars whereon the classical statistical
physics is founded. Unfortunately, regardless of several attempts have been made \cite{bib:silva2010,bib:deroeck2006,bib:grabert1974,bib:han2015,bib:das2018,bib:vonneumann2010}, 
the generality of the classical $H$-theorem has no an equally robust quantum match. In this section, we shall
propose and analize an alternative quantum $H$-functional using the variational method.
We start by briefly outline a typical textbook demonstration of the quantum $H$-theorem
\cite{bib:tolman}, then, we present the analysis of our proposed $H$-functional.

\subsection{$H$-theorem and the Fermi-Dirac and Bose-Einstein distribution functions}

Consider a diluted gas of $N$ non-interacting quantum particles (either bosons or fermions),
contained by a vessel of volume
$V$, at some temperature $T$, and total energy E. Starting from the Boltzmann definition
of entropy, the quantum $H$ functional is
%
\begin{equation}
   H_T=-\ln G,
\end{equation}
%
where $G$ describes the total number of accessible quantum states of the gas that satisfy the
above mentioned conditions \cite{bib:tolman}.
The quantum $H$-theorem can be demostrated as follows.  $G$ can be divided into groups of neighboring states,
$g_k$, and a certain occupation numbers, $n_k$, can be associated with each of these
groups. Thus, the above functional takes the form:
%
\begin{equation}\label{eq:quantumh}
    H_T=\sum_i n_i \ln n_i -(n_i\pm g_i)\ln (g_i \pm n_i)\pm g_i\ln g_i, 
\end{equation}
%
where the upper and lower signs are for bosons and fermions, respectively.
Thus the time derivative of Eq.~(\ref{eq:quantumh}) is
%
\begin{equation}\label{eq:quantumdHdt}
\frac{dH_T}{dt}=\sum_{\kappa}\left[\ln n_{\kappa}-\ln\left(g_{\kappa}\pm n_{\kappa}\right)\right]
\frac{dn_{\kappa}}{dt}.
\end{equation}
%

Assuming that the energy exchange between particles is produced by collisions between particles.
By using perturbation theory, the rate of change in the number of particles in a group $\kappa$ is
%
\begin{eqnarray}\label{eq:changen}
    \frac{d n_{\kappa}}{dt}&=&-\sum_{\lambda,(\mu \nu)}A_{\kappa\lambda,\mu\nu} n_{\kappa}n_{\lambda}(g_{\mu}\pm n_{\mu})(g_{\nu}\pm n_{\nu})\nonumber \\
    &&+\sum_{\lambda,(\mu \nu)}A_{\mu\nu,\kappa\lambda} n_{\mu}n_{\nu}(g_{\kappa}\pm n_{\kappa})(g_{\lambda}\pm n_{\lambda}).
\end{eqnarray}
%
Here $A_{\kappa\lambda,\mu\nu}n_{\kappa}n_{\lambda}(g_{\mu}\pm n_{\mu})(g_{\nu}\pm n_{\nu})$ is the
expected number of collisions per unit time, in which two particles will be moved from groups
($\kappa$, $\lambda$) to ($\mu$, $\nu$), and the tensor $A_{\kappa\lambda,\mu\nu}$ is given by
%
\begin{equation}\label{eq:amnkldef}
  A_{\kappa\lambda,\mu\nu}=\frac{4\pi^{2}}{h}\frac{|I_1\pm I_2|^2}{\Delta \epsilon}.
\end{equation}
%
In Eq.~(\ref{eq:amnkldef}), $\Delta\epsilon$ is the net energy change occuring during the
collision and $|I_1-I_2|^2=|V_{mn,kl}|^2$ where
$V_{mn,kl}$ is the element of the transition matrix of a binary collision.
It is important to remark that in deriving Eq.~(\ref{eq:changen}), the equal a priori
probabilities and the random a priori phases hypotheses were taken valid. The 
the random a priori phases hypothesis can be
considered an analogous to the molecular chaos hypothesis
\cite{bib:das2018} as the mechanism to introduce stochasticity into the system.

Substituting Eq.~(\ref{eq:changen}) into Eq.~(\ref{eq:quantumdHdt}) after some algebra, it can be proven that
%
\begin{equation}
    \frac{dH_T}{dt}\leq 0. \label{eq:H-theorem-tolman}
\end{equation}
%

In equilibrium (at $t\to\infty$), $dn_{\kappa}/dt=0$, hence from Eq.~(\ref{eq:changen})
%
\begin{equation}\label{eq:boseiferdirdistfnctsrc}
    \ln \frac{n_{\kappa}}{g_{\kappa}\pm n_{\kappa}}+\ln \frac{n_{\lambda}}{g_{\lambda}\pm n_{\lambda}}=\ln \frac{n_{\mu}}{g_{\mu}\pm n_{\nu}}+\ln \frac{n_{\nu}}{g_{\nu}\pm n_{\nu}}.
\end{equation}
%
Considering that the energy is conserved during the collsion, the Bose-Einstein or Fermi-Dirac distribution
functions can be recovered from Eq.~(\ref{eq:boseiferdirdistfnctsrc}):
%
\begin{equation}
   n_{\kappa}=\frac{g_{\kappa}}{\exp(\alpha+\beta\epsilon_{\kappa})\mp1}.
\end{equation}
%

\textit{I.e.}, in equilibrium $dH_T/dt=0$ and the distribution function obtained from
Eq.~(\ref{eq:quantumh}) is the expected distribution function.

%----------------------------------------------
\begin{comment}
In addition, Tolman also showed that (\ref{eq:quantumh}) can be reduced, in the
high-energy limit, to
%
\begin{equation}
    H_T = \sum_{\kappa} (n_{\kappa} \ln n_{\kappa} - n_{\kappa} \ln g_{\kappa}). \label{reduce-h}
\end{equation}
%
This expression also can be obtained also from the Boltzmann $H$-functional through defining
%
\begin{equation}
    f=\frac{n_{\kappa}}{ g_{\kappa}}.
\end{equation}
\end{comment}
%
%--------------------------



\subsection{Out of equilibrium, non-homogeneous quantum systems}

Consider a diluted gas enclosed by a perfectly isolated vessel of volume $V$, with
total energy $E$, and total number of quantum particles $N$, which can be free fermions or
bosons. For our purposes, the volume $V$ is divided into $K$ small cells, each of which
has a constant volume $\delta V_M=V/K$ ($M=1,\,\dots\,,K$), a temperature $T_M$, an energy $\epsilon_M$,
a number of particles $\mathcal{N}_M$, and a distribution function,
$\{f_{Mn}(t)\}$. Hereafter we will use a short-hand notation:
%
\begin{equation}
   f_{Mn}(t)\equiv f(\vec r_M,\epsilon_{n},t),
\end{equation}
%
where $\vec r_M$ is the radius vector pointing at the center of the $M$-th cell.
$f_{Mn}(t)$ represent the number
of particles contained in the $M$-th cell that occupy the energy level $\epsilon_n$ at
time $t$. Since the particles are considered to be free, the energy levels should not
depend on the cell properties, \textit{i.e.} the energy spectrum, $\{\epsilon_n\}$, is the same for all
cells; thus there is no need to label $\epsilon_n$ with an index $M$.

We propose the following functional as an alternative $H$-functional
for quantum non-homogeneous diluted gases:
%
\begin{eqnarray}\label{eq:qHdef}
    \mathcal{H} (t)&=&\sum_{M=1}^{K} \sum_{n} \bigg[ f_{Mn}(t) \ln f_{Mn}(t)\bigg.\nonumber \\
    &&\qquad\qquad\bigg.\pm \Big(1 \mp f_{Mn}(t)) \ln (1 \mp f_{Mn}(t)\Big) \Big]   \delta V_M.
\end{eqnarray}
%
Here, the upper and lower sign refer to \commr{bosons and fermions respectively}.
In addition, when needed, each cell will have an associated local chemical potential,
$\alpha_M$, and a local $H$-functional, which is defined by:
%
\begin{equation}\label{eq:qHMdef}
   \mathcal{H}_M(t)=\sum_{n} \bigg[ f_{Mn}(t) \ln f_{Mn}(t)
   \pm \Big(1 \mp f_{Mn}(t)\Big) \ln \Big(1 \mp f_{Mn}(t)\Big) \bigg] \de V_M.
\end{equation}
%
Therefore, $\mathcal{N}_M$ and $\mathcal{E}_M$ as functions of time are given by:
%
\begin{subequations}
\begin{equation}
    {\mathcal{N}}_M(t) = \sum_{n}f_{Mn}(t) \delta V_M
\end{equation}
%
and
%
\begin{equation}
\mathcal{E}_M(t) = \sum_{n}f_{Mn}(t)\epsilon_{n} \de V_M,
\end{equation}
\end{subequations}
%
which are, for the whole system, constrained by the microcanonical restrictions
%
\begin{subequations}\label{eq:qGlobRest}
\begin{equation}\label{eq:qGlobRestN}
    \sum_{M=1}^{K}\left[\sum_{n}f_{Mn}(t)\right]\delta  V_M
    =\sum_{M=1}^{K}\mathcal{N}_{M}(t)\delta V_M=N,
\end{equation}
%
and
%
\begin{equation}\label{eq:qGlobRestE}
    \sum_{M=1}^{K}\left[\sum_{n}f_{Mn}(t)\epsilon_{n}\right]\delta V_M
    =\sum_{M=1}^{K}\mathcal{E}_M(t)\delta V_M=E. 
\end{equation}
\end{subequations}
%

Applying the variational method to $\mathcal{H}$, and using the
Lagrange multipliers
$\{\alpha_M\}$ and $\{\beta_M\}$, it is straightforward to obtain
(see also the discussion of Eq.~(\ref{eq:deltaHpdeltafpj})):
%
\begin{equation}\label{eq:relation}
\ln \left(\frac{1\mp f_{Mn}(t)}{f_{Mn}(t)} \right)=-\alpha_M(t)-\beta_M(t) \epsilon_{n},
\end{equation}
%
and solving for $f_{Mn}(t)$ yields
%
\begin{equation}\label{eq:qfMn}
f_{Mn}(t)=\frac{1}{\exp\big(-\alpha_M(t)-\beta_M(t) \epsilon_{n}\big)\pm 1}.
\end{equation}
%
Thus, in this zero-order approximation the form of equilibrium distribution functions are conserved.

%============================================================
\subsubsection{Properties of $\mathcal{H}$ for systems in equilibrium}
%============================================================
If the system is in equilibrium (we will show that it will also be valid at $t\to\infty$), the temperature becomes homogeneous throughout
the complete system. Also, the local number of particles, the local
energy, and the Lagrange multipliers do not depend on the cell number and they should be homogeneous,
this is 
%
\begin{subequations}\label{eq:qEqRestrictions}
\begin{equation}
   {\mathcal{N}}_M(t\to\infty)\equiv \bar{\mathcal{N}}=N/K,
\end{equation}
%
%
\begin{equation}
	{\mathcal{E}}_M(t\to\infty)\equiv \bar{\mathcal{E}}=E/K,
\end{equation}
\end{subequations}
%
%
\begin{subequations}\label{eq:qEqAlphaBeta}
\begin{equation}
	\alpha_M=\bar\alpha,\quad\forall M,
\end{equation}
%
and
\begin{equation}
	\beta_M=\bar\beta, \quad\forall M.
\end{equation}
\end{subequations}
%
Substituting Eqs.~(\ref{eq:qEqAlphaBeta}) into Eq.~(\ref{eq:qfMn}) yields the distribution
function of each cell in equilibrium to be
%
\begin{equation}\label{eq:qfneq}
    \bar f_{Mn}=\bar f_n =\frac{1}{\exp\big(-\bar\alpha-\bar\beta \epsilon_n\big)\pm 1},\quad\forall M.
\end{equation}
%
By means of the above equation, we can recover the distribution function and the entropy of a diluted quantum gas in equilibrium as follows. Setting $\bar\alpha=\mu/kT$ and $\bar\beta=-1/kT$,
and substituting them in Eq.~(\ref{eq:qfneq}), it renders the Fermi-Dirac and Bose-Einstein
distribution functions:
%
\begin{equation}\label{eq:qfneqtwo}
    \bar{f}_{n}=\frac{1}{\exp\big(\frac{{\epsilon_n}-\mu}{kT}\big)\pm 1},
\end{equation}
%
and substituting Eq.~(\ref{eq:qfneq}) into the negative of Eq.(\ref{eq:qHdef}), 
the entropy of a quantum ideal gas is obtained
%
\begin{eqnarray}\label{eq:varEntropyDef}
      S&=&\sum_{M=1}^{K}
        \sum_n  \left[
        		\left(\frac{1}{\exp\big(-\bar{\alpha}-\bar{\beta}\epsilon_{n}\big)\pm 1} \right)
           \commr{\ln \left(\frac{1}{\exp\big(-\bar{\alpha}-\bar{\beta}\epsilon_{n}\big)\pm 1} \right)}
         \right]\nonumber \\
      &&\quad\pm  \ln \left[\prod_{M=1}^{K} \prod_{n}
         \left(1 \mp \frac{1}{\exp\big(-\bar{\alpha}-\bar{\beta}\epsilon_{n}\big)\pm 1} \right)
         \right] \delta V_M.
\end{eqnarray}
%
This quantity is what we refer to as ``variational entropy'',
and this name reflects the fact that it was obtained \textit{via}
the variational method. {\color{blue} We will use it at the end of this article to analyze relaxation processes.}

%=============================================
\subsubsection{Proof of the quantum $H$-theorem for non-homogeneous systems}
%=============================================

For quantum systems, we also accept the validity of the local equilibrium hypothesis
for every cell in the system. This allows us to define non-homogeneous systems,
wherein thermodynamic quantities are well-defined on a cell-per-cell basis,
and in terms of the equilibrium properties, we have

%
\begin{subequations}\label{eq:restrictionoutside}
\begin{equation}
        \mathcal{N}_M(t)=\sum_{n}f_{Mn}(t)=\bar{\mathcal{N}}+\Delta_M(t) 
\end{equation}
  and
\begin{equation}
        \mathcal{E}_M(t)=\sum_{n}\epsilon_{n}f_{Mn}(t)=\bar{\mathcal{E}}+ \delta_M(t).
\end{equation}
\end{subequations}
%
In Eqs.~(\ref{eq:restrictionoutside}) $\bar {\mathcal{N}}$ and $\bar{\mathcal{E}}$ are the cell particle
number and the cell energy in equilibrium, which are given by Eqs.~(\ref{eq:qEqRestrictions}),
and $\Delta_M$ and $\delta_M$ are deviations from $\bar{\mathcal{N}}$
and $\bar{\mathcal{E}}$, respectively, with $\Delta_M(t)\ll \bar{\mathcal{N}}$
and $\delta_M(t) \ll \bar{\mathcal{E}}$. 


In the present context, $|\Delta_M|$ and $|\delta_M|$ are sufficiently large so they are not
fluctuations of the system, but enough small so that the local equilibrium hypothesis is
valid for $t>0$ (we set $t_0=0$, and $t_0$ is the initial time at which the system is prepared).
Therefore, it is reasonable to re-write the distribution functions as:
%
\begin{equation}\label{eq:qFirstOrd}
   f_{Mn}(t)=\bar{f}_n(1+g_{Mn}(t)),\quad
   1\gg|g_{Mn}(t)|,
\end{equation}
%
from which it follows, by substituting Eq.~(\ref{eq:qFirstOrd}) into
Eq.~(\ref{eq:restrictionoutside}), that $\Delta_M$ and $\delta_M$ satisfy:
%
\begin{subequations}
\begin{equation}
    \Delta_M(t)=\sum_n \bar{f}_n g_{nM}
\end{equation}
%
and
%
\begin{equation}
	\delta_M(t)=\sum_n  \bar{f}_n g_{nM}\epsilon_n.
\end{equation}
\end{subequations}
%

\commr{
An additional consideration is necessary for treating gases composed of fermions,
which must obey the relation $f_{Mn}(t)\leq1$, hence
%
\begin{eqnarray}\label{eq:fermionrestriction}
   1-\bar f_n -\bar f_n g_{nM}\geq0 \ \ \Rightarrow \ \ \frac{1}{\bar f_n}\geq1+g_{nM}.
\end{eqnarray}
%
$\bar f_{n}=1$ is certainly satisfied if the system is at zero temperature. In this state,
all energy levels below and including the Fermi energy are occupied because of
the exclusion principle, therefore the system will
necessarily be homogeneous , and consequently $g_{nM}=0$. In this article, we will
omit this scenario and will only discuss Fermi gases whose temperature is non zero.
}


%\begin{proof}
To proof the quantum $H$-theorem, start by taking the time-derivative of Eq. (\ref{eq:qHdef}):
%
\begin{equation}\label{eq:deltaH}
   \frac{d \mathcal{H} (t)}{dt}= \sum_n \sum_{M=1}^{K} \dot{f}_{nM}(t)\ln \left[ \frac{f_{nM}(t)}{1\mp f_{nM}(t)} \right] \de V_M.
\end{equation}
%
Subsequently, substitute Eq.~(\ref{eq:qFirstOrd}) in the above equation to obtain:
%
\begin{eqnarray}\label{eq:cambioH1}
    \frac{d\mathcal{H} (t)}{dt}&=&
      \sum_n \sum_{M=1}^{K} \bar{f}_{n}\ln \left[
        \frac{\bar{f}_{n}(1+g_{nM})}{1\mp \bar{f}_{n} (1+ g_{nM})}
      \right]\dot{g}_{nM} \delta V_M \nonumber \\
    &=&\sum_n \sum_{M=1}^{K} \bar{f}_n \left \{
      \ln [\bar{f}_n+\bar{f}_n g_{nM}]\dot{g}_{nM}
      -\ln [1\mp\bar{f}_n\mp\bar{f}_n g_{nM}]\dot{g}_{nM}
    \right \}\delta V_M.
\end{eqnarray}
%
Approximate the logarithmic terms, corresponding to
Fermi and Bose gases, through a Taylor series around $\bar f_n g_{nM}=0$:
%
\begin{subequations}\label{eq:qlnApprox}
\begin{equation}\label{eq:qlnApproxFermions}
	\ln[1\mp\bar{f}_n\mp\bar{f}_n g_{nM}]
    	\approx \ln[1\mp\bar{f}_n]\mp\frac{\bar{f}_n}{1\mp\bar{f}_{n}} g_{nM}
\end{equation}
%
and
%
\begin{equation}\label{eq:qlnApproxBosons}
    \ln [\bar{f}_n+\bar{f}_n g_{nM}] \approx \ln [\bar{f}_n]+ g_{nM},
\end{equation}
\end{subequations}
%
respectively. Eq.~(\ref{eq:qlnApproxFermions}) is valid because, for non-extremely degenerated Fermi gases, 
$1-\bar{f}_n \gg \bar{f}_n|g_{nM}|$, and Eq.~(\ref{eq:qlnApproxBosons}) is fulfilled because,
for Boson gases,
$1+\bar{f}_n \gg \bar{f}_n |g_{nM}|$ when $\bar{f}_n \gg \bar{f}_n |g_{nM}|$.

Combine Eqs. (\ref{eq:cambioH1}) and (\ref{eq:qlnApprox}):
%
\begin{eqnarray}\label{eq:cambioH2}
    \frac{d\mathcal{H}}{dt}&=&\sum_n \sum_{M=1}^{K} \bar{f}_n\left \{ (\ln \bar{f}_n+ g_{nM})\dot{g}_{nM}\right\} \de V_M \delta \epsilon_n \nonumber \\
    &&-\sum_{n}\sum_{M=1}^{K}\bar f_n\left\{ \left( \ln[1\mp\bar{f}_n]\mp \left[\frac{\bar{f}_n}{1\mp\bar{f}_n} \right] g_{nM}\right)\dot{g}_{nM} \right \}\de V_M,
\end{eqnarray}
%
and substitute Eq.~(\ref{eq:qfneq}) into Eq.~(\ref{eq:cambioH2}):
%
\begin{equation}\label{eq:cambioH3}
    \frac{d\mathcal{H}}{dt}=
       \sum_n \sum_{M=1}^{K} \bar{f}_n\left \{
          (\bar{\alpha}+\bar{\beta}{\epsilon}_n)\dot{g}_{nM}
           +g_{nM}\left(1\pm e^{\bar{\alpha}+\bar{\beta}{\epsilon}_n}\right)\dot{g}_{nM}
       \right \} \de V_M.
\end{equation}
%

Since both the total number of particles and the total energy 
of the system are constant, it follows,
from Eqs.~(\ref{eq:qGlobRest}) and (\ref{eq:restrictionoutside}) that:
%
\begin{subequations}
\begin{equation}
\frac{dN}{dt}=\sum_{M=1}^K\dot{\mathcal{N}}_M\delta V_M
   =\sum_{M=1}^K\sum_n \bar{f}_n \dot{g}_{nM}\delta V_M=\sum_{M=1}^K\dot{\Delta}_M(t)\delta V_M=0
\end{equation}
%
and
%
\begin{equation}
\frac{dE}{dt}=\sum_{M=1}^K\dot{\mathcal{E}}_M\delta V_M
   =\sum_{M=1}^K\sum_n \bar{f}_n \dot{g}_{nM}\epsilon_n\delta V_M=\sum_{M=1}^K\dot{\delta}_M(t)\delta V_M=0.
\end{equation}
\end{subequations}
%

Substitute the previous expression in Eq.~(\ref{eq:cambioH3}) to obtain:
%
\begin{equation}\label{eq:cambioH4}
   \frac{d\mathcal{H}}{dt}=\sum_n e^{\bar{\alpha}+\bar{\beta}\epsilon_n}
   \sum_M  g_{nM}\dot{g}_{nM} \delta V_M \leq 0. \qquad\qquad\textrm{QED.}
\end{equation}
%\end{proof}
%

For obtaining the right-most hand side of Eq.~(\ref{eq:cambioH4}), 
we have used the relation $g_{nM}\dot{g}_{nM}\leq0$ for $t>0$. This can be proven 
by simply arguing that, in the initial
state, if a cell is described by $g_{nM}(t_0)>0$ then $g_{nM}(t)\geq0$ and $\dot g_{nM}(t)\leq0$, and if
$g_{nM}(t_0)<0$ then $g_{nM}(t)\leq0$ and $\dot g_{nM}(t)\geq0$. Here we have used the facts that
the system in equilibrium is homogeneous, and that, by accepting the local equilibrium hypothesis, $g_{Mn}(t)$
is a monotonic function and $g_{nM}\to0$ as $t\to\infty$ as the system aproaches to the equilibrium state.
Another approach to prove Eq.~(\ref{eq:cambioH4}) consists of splitting the cells into two subsets,
just as we did in the classical scenario.

Succinctly, considering a diluted quantum gas contained in a vessel of volume $V$ (divided into
$K$ small cells), with total energy $E$ and $N$ quantum free particles, which initially
is out of equilibrium ---but in such a manner that the local equilibrium hypothesis is valid---,
the functional
%
\begin{eqnarray}\label{eq:qHfunctTheo}
    \mathcal{H} (t)&=&\sum_{M=1}^{K} \sum_{n} \bigg[ f_{Mn}(t) \ln f_{Mn}(t)\pm \Big(1 \mp f_{Mn}(t)) \ln (1 \mp f_{Mn}(t)\Big) \Big]   \delta V_M,
\end{eqnarray}
%
where $f_{Mn}$ is the $M$-th cell distribution function,
evolves in time such that $d\mathcal{H}/dt\leq0$, and the equality 
condition is attained when the system reaches
the equilibrium state. In Eq.~(\ref{eq:qHfunctTheo}), and for a Fermi (Bose)
gas, $f_{Mn}$ corresponds to the Fermi-Dirac (Bose-Einstein) distrubution function for each cell.
Locally, each cell is in equilibrium, although the complete system may be non-homogeneous,
and is caracterized by the respective $f_{Mn}$, number of particles $\mathcal{N}_M$, energy $\mathcal{E}_M$,
temperature $T_M$, and Legendre Multipliers $\alpha_M$ and $\beta_M$.


