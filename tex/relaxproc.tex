
  \section{Relaxation processes in degenerated quantum gases}\label{sec:relaxproc}
In order to obtain an equation of motion for the quantum gas, we evaluate 
$\Delta H=H(t_2)-H(t_1)$ where the $H$ functional is evaluated in diferent times such that $t_2 > t_1$.
We begin using the following notation $f_{nM}(t_2)\equiv f'_{nM}$ and 
$f_{nM}(t_1) \equiv f''_{nM}$. 
Now, using (\ref{eq:qHdef}) to calculate $\Delta H$ with the previous notation, we obtain
\begin{eqnarray} \label{eq:deltaHdef}
	\Delta H &=& \sum_{M}^{K}\sum_n [f'_{nM} \ln f'_{nM} - f''_{nM}	\ln f''_{nM}\nonumber \\ 
			&&\pm (1\mp f'_{nM}) \ln(1\mp f'_{nM}) \mp (1\mp f''_{nM}) \ln (1\mp f''_{nM})]\delta V.  
\end{eqnarray}
In order to obtain a general equation of motion for the distribution function 
$f_{nM}$, we only take the limit when $(t_2-t_1) \rightarrow 0 $. 
However, our current work is focused on the deviation $g_{nM}$. So, keeping the same workline
we apply the first order approximation (\ref{eq:qFirstOrd}) to (\ref{eq:deltaHdef}) to obtain the following expression
\begin{eqnarray}\label{eq:deltaH1}
	\Delta H &=& \sum_{M}^{K} \sum_n [\bar{f}_n(1+g'_{nM}) \ln \bar{f}_{n}(1+g'_{nM})  -  
			\bar{f}_n(1+g''_{nM}) \ln \bar{f}_n(1+g''_{nM}) \nonumber \\
		    &&  \pm (1 \mp \bar{f}_n \{1+g'_{nM}\}) \ln (1\mp \bar{f}_n\{1+g'_{nM} \})\nonumber \\
		    &&  \mp (1\mp \bar{f}_n \{ 1+g''_{nM} \}) \ln (1\mp \bar{f}_n \{1+g''_{nM} \})]\delta V.\nonumber \\
\end{eqnarray} 
The equation (\ref{eq:deltaH1}) can be rewritten, after some approximations on the logarithm functions, as
\begin{equation}\label{eq:deltaH2}
\Delta H = \sum_{M}^{K}\sum_n [\bar{f}_n(1+\ln \bar{f}_n)-\bar{f}_n \{\ln(1\mp \bar{f}_n)\mp 1 \}](g'_{nM}-g''_{nM})\delta V.
\end{equation}
The equation (\ref{eq:deltaH2}) must be analyzed taking the limit $(t_2-t_1) \rightarrow 0$ 
to obtain the equation of motion for the deviation $g_{nM}$.
This analysis will not be treated in this article.

Eq. 68 can be seen as a time-evolution equation for  g. The specific form of g'-g''
must be made in order to have a \dots


  
  %----------------------------------------


%-------------------------------------------------

