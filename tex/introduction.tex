\section{Introduction}

In the literature, the theoretical basis that allows us to describe equilibrium
systems in the classical scheme is well-established.
The procedures to describe
equilibrium systems are well-known and work in almost every system in nature
(Thermodynamics for a phenomenological description and statistical mechanics
for a microscopic description). In contrast, a unified mathematical model to
describe the evolution of any system to an equilibrium state has not developed.
Nevertheless, developments can be found in the literature and describe the
evolution to equilibrium to almost any system (the internal behavior of glasses
\cite{bib:zanotto2018} and the entropy measurement
\cite{bib:schmelzer2018,bib:nemilov2018} are some examples where the recent
developments do not work.) Examples of those developments can be the Onsager
formulation (called linear thermodynamics)
\cite{bib:keizer1987,bib:onsager1931}, and the Kinetic Theory of dilute gases
developed by Boltzmann.

In the Kinetic Theory of dilute gases, the behavior of a dilute classical gas
was described by Boltzmann through the Boltzmann transport equation.
Furthermore, the evolution of an equilibrium state for any system was proved by
means the Boltzmann's $H$-theorem (or classical $H$-theorem) and the
spatial-homogeneous distribution function hypothesis.
However, there are some aspects in the classical $H$-theorem and the
Boltzmann's $H$ functional that deserve some review to accomplish better
consistency, for example, the modification of the $H$-theorem in stochastic
trajectories, violations in the second law of thermodynamics, and the relation
between Shannon's measure of information and the Boltzmann's entropy %[6-10].
\cite{bib:nemilov2018,bib:keizer1987,bib:onsager1931,bib:brown2008,bib:dragoljub2009}.

To obtain the well-known correspondence principle between classical mechanics
and quantum mechanics, a quantum version of the Boltzmann's $H$-theorem (or
quantum $H$-theorem) and the quantum analogous to the Boltzmann transport
equation is necessary such that in the appropriate limit, the  classical $H$
theorem and the Boltzmann transport equation should be recovered.

One of the first developments in this field was developed by Tolman
\cite{bib:tolman}. He proposes one of the first quantum versions of the
Boltzmann's $H$-theorem using the probability transition relation, the random
phases hypothesis, and a homogeneous distribution function. The equation of
motion for the occupation number (necessary to prove the quantum $H$-theorem in
the same way that Boltzmann's procedure) was obtained by applying the time
perturbation theory. However, to obtain the flux relations in analogy to the
Boltzmann's method, a quantum version of the transport equation that includes a
non-homogeneous distribution function is required.

On the other hand, some works in the quantum operator formalism attempt to
describe quantum transport phenomena using the Hamiltonian of the system and
the master equation (analogous to the Boltzmann transport equation)
\cite{bib:gorban2014,bib:bennaim2017,bib:tolman,bib:li2019}. %[12-15]
Besides, some authors construct the $H$ functional and develop the proof of the
quantum $H$-theorem \cite{bib:silva2010,bib:deroeck2006,bib:grabert1974}.
However, the homogeneous distribution function hypothesis and the
correspondence principle are uncleared or not discussed. Moreover, the validity
of the quantum $H$-theorem and the second law of thermodynamics, and the
interpretation of the quantum entropy is still in discussion %[19-26].
\cite{bib:silva2010,bib:deroeck2006,bib:grabert1974,bib:acharya2019,%
bib:kastner2017,bib:gring2012,bib:han2015,bib:das2018}.

In order to contribute to the construction of a consistent classical and
quantum $H$-theorem, and a new formalism to describe out of equilibrium
systems, we develop a new theoretical basis to describe them. In particular, we
are interested in including a non-homogeneous distribution function in the
mathematical model, and so including non-homogeneous systems in the $H$
theorem. Besides, we are interested in developing the quantum analogous of
Boltzmann's $H$-theorem considering a non-spatial-homogeneous distribution
function and solve the problem of the correspondence principle mentioned above.

The organization of this article is as follows.
In Section~\ref{sec:classical}, we highlight, to our purposes, the most
fundamental assumptions required to proof the Boltzmann $H$-theorem,
we provide an alternative method to obtain the Maxwell-Boltzmann
distribution using the variational method, and an alternative demonstration of
the $H$-theorem for classical systems. In Section~\ref{sec:quantum}
we review the Tolman proposal for the quantum
version of the $H$-theorem (quantum $H$-theorem)
\commr{and how the Bose-Einstein and Fermi-Dirac distributions are treated within
this framework},
and our alternative proof of the quantum $H$-theorem. In 
Section~\ref{sec:qccorrespondence}
the classical-quantum correspondence between the
quantum $H$ functional and the classical $H$ functional is analyzed.
In Section~\ref{sec:relaxproc} we explore how relaxation processes occur
in a quantum ideal gas; here we provide a time-evolution equation that could be
considered the analogous of the classical Boltzmann transport equation. Finally,
we discuss some key ideas derived from our approach in Section~\ref{sec:disscussion},
and provide some conclusions in Section~\ref{sec:conclusions}.

