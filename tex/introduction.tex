\section{Introduction}

The theoretical bases and the procedures that allow us to describe equilibrium systems
are well-established. These procedures can be applied to a wide range of natural systems,
including both the macroscopic phenomenological methods (thermodynamics)
and the microscopic description (statistical mechanics).
(Out-of-equilibrium systems, of course, are still a challenge.)
For instance,
in the kinetic theory of gases, the behavior of a dilute classical gas
is described through the Boltzmann transport equation \cite{bib:huang},
and the time-evolution of a system towards equilibrium is finely accounted for
through the Boltzmann $H$-theorem.

However, for quantum out-of-equilibrium systems, the construction of a
kinetic framework with the same level of
success and universality as the classical version still presents some fundamental challenges. For instance,
to obtain a complete correspondence principle between classical mechanics
and quantum mechanics, the form of the quantum analogues of both the Boltzmann $H$-theorem
and the Boltzmann transport equation is inadequate.
In this context, Tolman was one of the earliest physicists to propose a quantum
$H$-theorem \cite{bib:tolman}, using a probability transition relationship, the random
phases hypothesis, and an $H$-functional defined in terms of a spatially homogeneous distribution function.
Tolman also proposed a potential quantum analogue of the transport equation,
in terms of the occupation numbers,
by applying time perturbation theory. Additional attempts, under quantum operator
formalism, have addressed the description of quantum transport phenomena through the Hamiltonian
of the system and the master equation (which is, in these works, the analogue
of the Boltzmann transport equation)
\cite{bib:grabert1974,%
bib:angel2017,
bib:amato2020,bib:nicacio2015,bib:hussein2014}. However, these approaches 
are not consistent with the classical-quantum correspondence principle.
Similarly, some authors have proposed $H$-functionals
and attempted to proof a
quantum $H$-theorem \cite{bib:gorban2014,bib:bennaim2017,bib:silva2010,%
bib:deroeck2006,bib:acharya2019,bib:kastner2017,bib:han2015,bib:das2018,bib:vonneumann2010}.
However, whether or not the homogeneous distribution function hypothesis is assumed or
if its framework fulfills the correspondence principle is unclear or not discussed.
Since the pioneering work of Tolman, at several stages, there has been some discussion regarding
the general validity of the quantum $H$-theorem,
some possible violations of the second law of thermodynamics, and the
interpretation of the quantum entropy  %[19-26].
\cite{bib:silva2010,bib:acharya2019,bib:kastner2017,bib:han2015,bib:brown2008,%
bib:dragoljub2009,bib:vonneumann2010,%
bib:syros1999,bib:lesovik2016,
bib:lesovik2019}.

Nonetheless, the framework to describe spatially non-homogeneous systems is still
under construction, although several approaches have been developed. For instance,
the celebrated Onsager formulation (linear thermodynamics)
\cite{bib:keizer1987,bib:onsager1931} has been successful in describing irreversible chemical 
and physical phenomena. However, some descriptions, such as those the internal behavior of gases
\cite{bib:zanotto2018} and the entropy measurement
\cite{bib:schmelzer2018,bib:nemilov2018}, cannot be completely addressed with linear thermodynamics.

In addition, some aspects regarding the classical $H$-theorem and the
Boltzmann $H$-functional require revision to improve their mutual consistency.
One example is the modification of the $H$-theorem to include phenomena stemming from stochastic
trajectories, violations of the second law of thermodynamics, the relationship
between Shannon’s measure of information and the Boltzmann’s entropy, 
and the calculation of thermodynamical quantities and thermalization of specific systems
\cite{bib:gorban2014,bib:li2019,bib:gring2012,bib:nemilov2018,bib:wang2014}.

To contribute to the construction of a consistent classical and
quantum $H$-theorem, within a formalism that describes out-of-equilibrium
non-homogeneous systems, we propose a new theoretical framework.
Specifically, for both classical and quantum systems, we include non-homogeneous distribution functions in
the $H$-functionals, and consider non-homogeneous systems in the proofs of the resulting
$H$-theorems. Our proposed $H$-functionals satisfy
the correspondence principle, but more importantly, these functionals describe
the time-evolution of spatially non-homogeneous
systems towards equilibrium.

The organization of this article is as follows.
In Section~\ref{sec:classical}, we highlight, for our purposes, the most
fundamental assumptions required to proof the Boltzmann $H$-theorem,
we provide an alternative method to obtain the Maxwell-Boltzmann
distribution using the variational method, propose
an alternative $H$-functional for classical systems, and demonstrate
the respective $H$-theorem. In Section~\ref{sec:quantum},
we review the Tolman proposal for the quantum
version of the $H$-theorem (quantum $H$-theorem)
and how the Bose-Einstein and Fermi-Dirac distributions are treated within
this framework. Subsequently, we present our proposal for a quantum $H$-functional
and the proof of the corresponding quantum $H$-theorem. In 
Section~\ref{sec:qccorrespondence}
we analyze the classical-quantum correspondence between the
quantum and classical $H$-functionals.
In Section~\ref{sec:relaxproc} we explore how relaxation processes occur
in a quantum ideal gas and, based on what we call variational entropy,
propose a time-evolution equation for the distribution function. Finally,
we discuss some key ideas resulting from our approach and close with a summary
in Section~\ref{sec:disscussion}.

