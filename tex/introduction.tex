\section{Introduction}

The theoretical basis and the procedures that allow to describe classical equilibrium systems
are well-stablished. These procedures can be applied to treat a wide range of the systems
present in nature, and include both the macroscopic phenomenological methods (thermodynamics)
as well as the microscopic description (statistical mechanics). For instance,
in the Kinetic Theory of gases, the behavior of a dilute classical gas
is described through the Boltzmann transport equation \cite{bib:huang},
and the time-evolution of a system towards the equilibrium is proved by
means of the Boltzmann $H$-theorem, wherein the
spatial-homogeneous distribution function hypothesis plays a crucial role.

In contrast, for quantum equilibrium systems, the construction of a framework,
with the same level of
success and universality as the classical version, is still an open problem. For instance,
in order to obtain a complete correspondence principle, between classical mechanics
and quantum mechanics, the quantum analogous of both the Boltzmann $H$-theorem
and the Boltzmann transport equation have not attained a fully accepted form.
In this context, Tolman was one of the earliest physicists to propose a quantum
$H$-theorem \cite{bib:tolman}, using probability transition relations, the random
phases hypothesis, and a spatially homogeneous distribution function.
Also, Tolman proposed a potential quantum analogous of the transport equation,
in terms of the occupation number,
by applying time perturbation theory. Further attempts, under the quantum operator
formalism, have addressed the description of quantum transport phenomena by means of the Hamiltonian
of the system and the master equation (which here is the analogous
of the Boltzmann transport equation)
\cite{bib:grabert1974,bib:wang2014,%
bib:angel2017,
bib:amato2020,bib:nicacio2015,bib:hussein2014}. However, these approaches 
are not consistent with the classical-quantum correspondence principle.
In the same line of thought, some authors have proposed $H$-functionals
and attempted to proof a
quantum $H$-theorem \cite{bib:gorban2014,bib:bennaim2017,bib:silva2010,%
bib:deroeck2006,bib:acharya2019,bib:kastner2017,bib:han2015,bib:das2018,bib:vonneumann2010}.
However, whether or not the homogeneous distribution function hypothesis is assumed, or
if their framework obeys the correspondence principle is unclear or not discussed whatsoever.
Furthermore, the general validity
of the quantum $H$-theorem, the second law of thermodynamics, and the
interpretation of the quantum entropy is still in discussion %[19-26].
\cite{bib:silva2010,bib:acharya2019,bib:kastner2017,bib:han2015,bib:brown2008,%
bib:dragoljub2009,bib:vonneumann2010,%
bib:syros1999,bib:lesovik2016,
bib:lesovik2019}.

On the other hand, the framework to describe spatially non-homogeneous systems is still
under construction, although several approaches have been developed. For instance,
the celebrated Onsager formulation (\textit{aka} linear thermodynamics)
\cite{bib:keizer1987,bib:onsager1931} have been successful in describing irreversible chemical 
and physical phenomena. Nonetheless, some aspects such as describing the internal behavior of glasses
\cite{bib:zanotto2018} and the entropy measurement
\cite{bib:schmelzer2018,bib:nemilov2018} cannot be completely addressed with linear thermodynamics.

In addition, some aspects regarding the classical $H$-theorem and the
Boltzmann $H$-functional require to be revised, for the purpose of improving their mutual consistency.
For example, the modification of the $H$-theorem to include phenomena stemming from stochastic
trajectories, violations of the second law of thermodynamics, the relation
between Shannon's measure of information and the Boltzmann's entropy, 
and calculation of thermodynamical quantities and thermalization for some systems
\cite{bib:gorban2014,bib:li2019,bib:gring2012,bib:nemilov2018,bib:wang2014}.

In order to contribute to the construction of a consistent classical and
quantum $H$-theorem, within a formalism that describes out-of-equilibrium
non-homogeneous systems, in this article we propose a new theoretical framework.
Specifically, for both classical and quantum systems, we include non-homogeneous distribution functions in
the $H$-functionals, and consider non-homogeneous systems in the proofs of the resulting
$H$-theorems. Our proposed $H$-functionals obey
the correspondence principle, but more importantly, these functionals describe
the time-evolution of spatially non-homogeneous
systems towards equilibrium.

The organization of this article is as follows.
In Section~\ref{sec:classical}, we highlight, to our purposes, the most
fundamental assumptions required to proof the Boltzmann $H$-theorem,
we provide an alternative method to obtain the Maxwell-Boltzmann
distribution using the variational method, and the demonstration of
an alternative $H$-theorem for classical systems. In Section~\ref{sec:quantum}
we review the Tolman proposal for the quantum
version of the $H$-theorem (quantum $H$-theorem)
\commr{and how the Bose-Einstein and Fermi-Dirac distributions are treated within
this framework},
and our alternative proof of the quantum $H$-theorem. In 
Section~\ref{sec:qccorrespondence}
the classical-quantum correspondence between the
quantum $H$-functional and the classical $H$-functional is analyzed.
In Section~\ref{sec:relaxproc} we explore how relaxation processes occur
in a quantum ideal gas; here we provide a time-evolution equation that could be
considered the analogous of the classical Boltzmann transport equation. Finally,
we discuss some key ideas derived from our approach and close with a brief summary
in Section~\ref{sec:disscussion}.


