%\section{Classical}
%\subsection{Boltzmann transport equation}
%\subsection{H theorem and the Maxwell-Boltzmann distribution}
%Todo esto de arriba en la forma tradicional y muy resumido. Solo remarcando lo fundamental.
%\subsection{Out of equilibrium, non-homogeneous distributions}.
%Entrar a la generalizacion, ahi si de manera muy detallada y con amplitud discutida
%fisicamente, del tratamiento con las celdas. Es decir, la demostracion del teorema H
%clasico pero con el enfoque de funciones de distribucion no homogeneas y por tanto
%con el tratamiento de celdas.

%----------------------------------

\section{Classical scheme}

\subsection{The Boltzmann transport equation}

In the classical version, the Boltzmann transport equation is required to show
the $H$-theorem. That equation is
%
\begin{equation}\label{eq:transport}
  \left(
    \frac{\partial}{\partial t}+\vec{v}_1 \cdot\nabla_r
    +\frac{\vec{F}}{m}\cdot\nabla_{v_1}
  \right)f_1=\int\mathrm{d}\Omega\int
    \mathrm{d}^{3}v_2\sigma(\Omega)|\vec{v}_1-\vec{v}_2|(f_2'f_1'-f_2f_1),
\end{equation}
%
where $\Omega$ is the solid angle, $\sigma$ is the scattering section, $F$ the
external force applied to the system, $f_1$, $f_2$ are the distribution
functions of the particles 1 and 2 respectively before the collision,
and $f'_1$, $f'_2$ are the distribution function of the particles 1 and 2
respectively after the collision.

On the one hand, dynamics, inhomogeneities, and external forces are described
by the left-hand side of the equation (\ref{eq:transport}). On the other hand, the
binary collisions among particles are involved in the right-hand side of the
previous equation. However, the \textit{molecular chaos hypothesis} was applied
to obtain the right-hand side of (\ref{eq:transport}). The \textit{molecular chaos
hypothesis} establishes that spatial and velocity coordinates of the particles
are not correlated.

%-------------------------
\subsection{H-theorem and the Maxwell-Boltzmann distribution}
The $H$ theorem can be proved using the $H$ functional defined by Boltzmann 
%
\begin{equation}\label{eq:hfunctional}
   H=\int f(\vec{v},t) \ln f(\vec{v},t) \mathrm{d}^{3}v,
\end{equation} 
%
where $f(\vec{v},t)$ is a homogeneous distribution function.
This $H$ functional describes a system with a global energy $E$, a global
number of free classical particles $N$, global volume $V$, and a temperature $T$.
Besides, the Boltzmann's $H$ functional correspond to a density of entropy
when the system arrives at the equilibrium state.

The $H$ theorem establishes that if $f(\vec{v},t)$ satisfies the Boltzmann
transport equation and the molecular chaos hypothesis, then
$dH/dt\leq0$ and the system arrives to the equilibrium state when
$dH/dt=0$.
The theorem can be proved by applying the time derivative
to Eq. (\ref{eq:hfunctional}) and using the Boltzmann transport equation
(\ref{eq:transport}). As a consequence of this theorem, the distribution function
that describes the system in equilibrium corresponds to the Maxwell-Boltzmann
distribution. This distribution can be obtained by solving a sufficient
condition for $f$ to obtain the equilibrium distribution function $f_0$:
%
\begin{equation}
    f_0(\vec{v}'_2)f_0(\vec{v}'_1)- f_0(\vec{v}_2)f_0(\vec{v}_1)=0.
\end{equation}
%

This condition can be obtained from the binary collision term of the Boltzmann
transport equation (\ref{eq:transport}).

The $H$-theorem fixes the time evolution direction of systems in nature, such
as the second law of thermodynamics does. {\color{blue}footnote: }The molecular hypothesis is %
fundamental to determine the physical meaning of the $H$ theorem. For more %
information, see \cite{bib:huang}.

%---------------------------
\subsection{Out of equilibrium, non-homogeneous distributions}

The Boltzmann $H$-functional (\ref{eq:hfunctional}) is defined with a spatially
homogeneous distribution $f(\vec{v},t)$. However, we want to consider a $H$
functional with a non-homogeneous distribution function in order to generalize
the Boltzmann $H$ theorem. Because of this, we introduce a modified $H$
functional (denoted as $H'$) defined as
%
\begin{equation}\label{eq:CH2}
   H'(t)=\sum_{M=1}^{K}\int f_M(\vec{v},t) \ln f_M(\vec{v},t)\mathrm{d}^3v.
\end{equation}
%
This $H'$ describes a system with a global energy $E$, a global number of free
classical particles $N$, global volume $V$, and a temperature $T$. Nevertheless,
the global system, in turn, consists of a set of $K$ small cells with a constant
volume $\delta V_M$, whose local homogeneous distribution functions is
$f_{M}(\vec{v},t)$.

The distribution function will depend on the position of the cells, the velocity
$\vec{v}$, and time $t$. Without loss of generality, we choose these cells to
have local identical volumes (\textit{i.e.} $\delta V_M = V/K, \forall M$).
Each cell has its local chemical potential, $\mu_M$, local
$H$ functional, $H'_M$, local number of particles, $\mathcal{N}'_M$, and local
energy, $\mathcal{E}'_M$. In terms of the local distribution 
function and the energy $\epsilon(\vec{v})$, the last three properties are
given by:
%
\begin{eqnarray}
    H_M' & = &  \int f_M(\vec{v},t) \ln f_{M}(\vec{v},t)
      \mathrm{d}^{3}v \label{Hcell},\nonumber\\
    \mathcal{N}_M' & = & \int f_{M}(\vec{v} ,t) \mathrm{d}^{3}v, \nonumber\\
    \mathcal{E}_M' & = & \int f_{M}(\vec{v},t)\epsilon(\vec{v}) \mathrm{d}^{3}v.
\end{eqnarray}
%

When the system is in equilibrium, the local number of particles and the local
energy do not depend on the cell number, this is
%
\begin{equation}
   \mathcal{N}'_M=\mathcal{N}'\equiv \bar{\mathcal{N}}';
   \quad\mathcal{E}'_M=\mathcal{E}'\equiv\bar{\mathcal{E}}'.
\end{equation}
%
The particles in the system are considered to be free, \textit{i.e.} the
particles do not interact with each other. Besides, the local distribution
function must satisfy the following microcanonical restrictions 
%
\begin{equation}\label{micro}
    \sum_{M=1}^{K}\int f_M(\vec{v},t)\mathrm{d}^3v =N,
    \quad\sum_{M=1}^{K}\int f_M(\vec{v},t)\epsilon(\vec{v})\mathrm{d}^3v=E.
\end{equation}
%

We can see that the functional (\ref{eq:CH2}) contains a non-homogeneous
distribution function but keeps the same form of the Boltzmann's $H$ functional.

Inspired by the Hamilton principle, $H'$ will be maximized with those
restrictions to obtain the equilibrium distribution function for a classical gas
using the Lagrange multipliers method obtaining
%
\begin{eqnarray}
    \frac{\delta H'}{\delta f_J(\vec{v}')} & = & \sum_{M=1}^{K}\int
      \frac{\delta}{\delta f_J(\vec{v}')}\left[
        f_M(\vec{v})\ln f_M(\vec{v})
        \right]
       \mathrm{d}^3v -\sum_{M=1}^{K}\alpha_M\int
       	\frac{\delta f_M(\vec{v})}{\delta f_J(\vec{v}')}
      \mathrm{d}^3v\nonumber\\
    & & -\sum_{M=1}^{K}\beta_M\int\epsilon(\vec{v})
    	\frac{\delta f_M(\vec{v})}{\delta f_J(\vec{v}')}
      \mathrm{d}^3v \nonumber\\
    & = & \ln f_J(\vec{v}')+1-\alpha_J-\beta_J \epsilon(\vec{v}')=0.
\end{eqnarray}
%
Isolating the distribution function 
%
\begin{eqnarray}
    \ln f_J(\vec{v}') & = & \alpha_J+\beta_J \epsilon(\vec{v}')
       -1\quad\Rightarrow\quad f_J(\vec{v}')\\
    & = & e^{\alpha_J +\beta_J \epsilon(\vec{v}')-1}\nonumber\\
    & = & Ce^{\alpha_J+\beta_J \epsilon(\vec{v}') } \label{relacion1},
\end{eqnarray}
%
where $C$ is a constant. It is trivial that using the variational procedures we
obtain the form of the Maxwell-Boltzmann distribution. 

As $H'$ in equilibrium is proportional to entropy, it is trivial to think that
Lagrange multipliers do not depend on the position because of the system is in
equilibrium and consequently, the distribution function is homogeneous, that is
%
\begin{equation}
    f(\vec{v})=Ce^{\alpha+\beta \epsilon(\vec{v})}\equiv \bar{f}(\vec{v}).
\end{equation} 
%
The constant $C$ can be omitted defining the following $H$ functional
%
\begin{equation}
   H''(t)=\sum_{M=1}^{K}\int \left[f_M(\vec{v},t)
    \ln f_M(\vec{v},t)-f_M(\vec{v},t)\right]\mathrm{d}^3v  \label{CH3},
\end{equation}
%
but $\sum_{M=1}^{K} \int f_M(\vec{v},t)\mathrm{d}^3v =N$ where $N$ is the
total particle number. As $N$ is a constant and any $H$ functional has sense
when we calculate its time derivative, this constant can be omitted. 

On the other hand, if the distribution function is spatial-homogeneous, $f_M$ does not depend on the cell number $M$, then we rewrite (\ref{eq:CH2}) as
%
\begin{equation}
    H'(t)=\int \sum_{M=1}^{K} [f(\vec{v},t)\ln f(\vec{v},t)] \mathrm{d}^{3}v
         = K\int  f(\vec{v},t) \ln f(\vec{v},t) \mathrm{d}^{3}v= K H_{boltz}(t), 
\end{equation}
%
where
%
\begin{equation}
    H_{boltz}(t)=\int f(\vec{v},t) \ln f(\vec{v},t) \mathrm{d}^{3}v.
\end{equation}
%
Besides, if we analyze the following expression
%
\begin{equation}
    \int Kf(\vec{v},t) \ln [Kf(\vec{v},t)] \mathrm{d}^{3}v = \int [(K\ln K)f(\vec{v},t) + Kf(\vec{v},t) \ln f(\vec{v},t)]\mathrm{d}^{3}v, \label{sum-h}
\end{equation}
%
we can observe if the number of particles in the $\mu$-space $f(\vec{v},t)$ is to big compare with the number of cells $P$, then the first term in (\ref{sum-h}) is negligible, and consequently, we conclude that
%
\begin{equation}
    \int f'(\vec{v},t) \ln f'(\vec{v},t) \mathrm{d}^{3}v = K\int f(\vec{v},t) \ln f(\vec{v},t) \mathrm{d}^{3}v; \ \ \ f'(\vec{v},t)= Kf(\vec{v},t). \label{aditive-property} 
\end{equation}
%
We obtained the additive property when we supposed the distribution function is
spatial-homogeneous.


To prove the Boltzmann's $H$ theorem (using (\ref{eq:CH2}) as functional), we need
to assume that the distribution function $f_M(\vec{v},t)$ must satisfy the
following assumptions:
%
\begin{itemize}
  \item The local equilibrium hypothesis in each cell ($f_{M}$ must be
    spatial-homogeneous for each $M$).
  \item The non-homogeneous distribution function assumption in the total
    volume ($f_{M}$ must be different among all cells).
  \item The following restrictions
    %
    \begin{equation}\label{restrictionoutsideclassical}
      \int f_{M}(\vec{v},t) \mathrm{d}^{3}v=
      \bar{\mathcal{N}}+\Delta_M(t); \ \ \ \ 
      \int f_{M}(\vec{v},t) \epsilon(\vec{v}) \mathrm{d}^{3}v=
      \bar{\mathcal{E}}+ \delta_M(t),
    \end{equation}
    %
    where $\bar {\mathcal{N}}$ and $\bar{\mathcal{E}}$ are
    the local particle number and the local energy in equilibrium
    %
    \begin{equation}
      \bar{\mathcal{N}}=
      \int \bar{f}(\vec{v}) \mathrm{d}^{3}v ; \quad
      \bar{\mathcal{E}}=
      \int \bar{f}(\vec{v})\epsilon(\vec{v}) \mathrm{d}^{3}v,
    \end{equation}
    %
    and $\Delta_M$, $\delta_M$ could be seen as a deviation from
    $\bar{\mathcal{N}}$ and $\bar{\mathcal{E}}$ respectively,
    with $\Delta_M(t)\ll\bar{\mathcal{N}}$ and
    $\delta_M(t)\ll\bar{\mathcal{E}}$.
\end{itemize} 
%
Those conditions described a system out of equilibrium but no so far from it,
that is a perturbed system and will suffer a \textit{relaxation process, i.e.}
The time evolution of a perturbated system to the equilibrium state.

It is important to remark that the quantities $\Delta_M$, $\delta_M$ are
sufficiently big to be different from fluctuations in the system but too small
such that the system is not far from the equilibrium state.  

Performing the derivative of (\ref{eq:CH2}) with respect to time, we obtain
%
\begin{equation}\label{dH1}
    \frac{dH'}{dt}=\sum_{M=1}^{K}\int\left[
      1+\ln f_M(\vec{v},t)
    \right]\dot f_M(\vec{v},t)\mathrm{d}^3v.
\end{equation}
%
Using the first-order approximation
%
\begin{equation}\label{firstorder}
   f_{Mn}(\vec{v},t)=\bar{f}_{n}(\vec{v})(1+g_{Mn}(\vec{v},t)): \ \ \ 
   \bar{f}_{n}(\vec{v})\gg \bar{f}_{n}(\vec{v})|g_{Mn}(\vec{v},t)|,
\end{equation}
%
(\ref{dH1}) yields
%
\begin{equation}\label{dH1-1}
    \frac{dH'}{dt}=\sum_{M=1}^{K}\int\bar f(\vec{v}) \left[
      1+\ln \left\{
        \bar f(\vec{v})+\bar f(\vec{v})g_M(\vec{v},t)
      \right\}
    \right]\dot g_M(\vec{v},t)\mathrm{d}^3v.
\end{equation}
%

We can approximate the logarithm function in (\ref{dH1-1}) to its first-order
Taylor series around $\bar f_n(\vec{v}) g_{nM}(\vec{v},t)=0$ 
%
\begin{equation}\label{lnapproximationclassical}
    \ln [\bar{f}(\vec{v})+\bar{f}(\vec{v}) g_{M}(\vec{v},t)] \approx
    \ln [\bar{f}(\vec{v})]+ g_{M}(\vec{v},t).
\end{equation}
%
With the previous approximation, (\ref{dH1-1}) obtains the following form 
%
\begin{eqnarray}
    \frac{dH'}{dt}&=&\sum_{M=1}^{K} \int \bar f(\vec{v})\left[
      1+\ln \bar f(\vec{v})+g_M(\vec{v},t)
    \right]\dot g_M(\vec{v},t)\mathrm{d}^3v,
\end{eqnarray}
%
using (\ref{relacion1}) with homogeneous multipliers
%
\begin{equation}\label{dH1-2}
    \frac{dH'}{dt} = \sum_{M=1}^{K}\int\bar f(\vec{v})\left[
      \alpha+\beta \epsilon(\vec{v})
    \right]\dot g_M(\vec{v},t)\mathrm{d}^3v +\sum_{M=1}^{K}
    \int\bar f(\vec{v})g_M(\vec{v},t)\dot g_M(\vec{v},t)\mathrm{d}^3v . \nonumber 
\end{equation}
%
On the other hand, we obtain from the restrictions the following expressions
%
\begin{eqnarray}
    \int \bar{f}(\vec{v}) g_{M}(\vec{v},t) \mathrm{d}^{3}v=\Delta_M(t) \ \  &\Rightarrow&
    \ \  \int \bar{f}(\vec{v}) \dot{g}_{M}(\vec{v},t)\mathrm{d}^{3}v=\dot{\Delta}_M(t), \nonumber \\
    \int  \bar{f}(\vec{v}) g_{M}(\vec{v},t)\epsilon(\vec{v}) \mathrm{d}^{3}v=\delta_M(t) \ \  &\Rightarrow&
    \ \  \int \bar{f}(\vec{v}) \dot{g}_{M}(\vec{v},t)\epsilon(\vec{v}) \mathrm{d}^{3}v=\dot{\delta}_M(t), \nonumber 
\end{eqnarray}
%
and as a consequence of $\sum_{M=1}^{K} \Delta_M(t) =\sum_{M=1}^{K} \delta_M(t) =0$,
we find
%
\begin{equation}
    \sum_{M=1}^{K} \dot{\Delta}_M(t)  =\sum_{M=1}^{K} \dot{\delta}_{M}(t) =0.
\end{equation}
%
Then using the previous result to (\ref{dH1-2}), we find
%
\begin{equation}
  \frac{dH'}{dt} = \sum_{M=1}^{K}\int\bar f(\vec{v})g_M(\vec{v},t)\dot g_M(\vec{v},t)\mathrm{d}^3v.
\end{equation}
%
Now, the summation over $M$ will be expressed in two summations
%
\begin{equation}\label{classicalH3}
    \frac{dH'}{dt}=\sum_J^{L}\int
      \bar f(\vec{v})g_J^{+}(\vec{v},t)\dot g_J^{+}(\vec{v},t)\mathrm{d}^3v
      +\sum_J^{P}\int
        \bar f(\vec{v})g_J^{-}(\vec{v},t)\dot g_J^{-}(\vec{v},t)\mathrm{d}^3v.
\end{equation}
%
where $L+P=K$. The group of $L$ cells is such that cells have an excess of
particles or/and energy (from the mean value in equilibrium). In contrast,
the group of $P$ cells is such that cells have to miss particles or/and energy
(from the mean value in equilibrium). Besides, $\dot{g}^{+}_{J}$ represents the
change on the deviation on cells that have an excess of particles or energy
while $\dot{g}^{-}_{J}$  represents the change on the deviations on cells that
have missing particles or energy. 
On the other hand, $g^{+}_{J}$  represents the value of the deviation on cells
that have an excess of particles or energy. In contrast, $g^{-}_{J}$ represents
the value of the deviation on cells that have missing particles or energy.

Also, on the one hand, $\dot{g}^{+}_{J}<0$ describes the loss of particles
and/or energy and so, $g^{+}_{J}>0$. On the other hand, $\dot{g}^{-}_{J}>0$
describes the gain of particles and/or energy and therefore $g^{-}_{J}<0$. 

We sort the previous ideas in the following form
%
\begin{equation}\label{separacionclassical}
\begin{array}{rl}
  g^{+}_{J}=+|g^{+}_{J}|; & \dot{g}^{+}_{J}=-|\dot{g}^{+}_{J}|,\\
  g^{-}_{J}=-|g^{-}_{J}|; & \dot{g}^{-}_{J}=+|\dot{g}^{-}_{J}|,
 \end{array}
\end{equation}
%
and consequently, (\ref{classicalH3}) obtains the following form
%
\begin{equation}
    \frac{dH'}{dt}=-\left[
      \sum_J^{L}\int\bar f(\vec{v})|g_J^{+}(\vec{v},t)|
        |\dot g_J^{+}(\vec{v},t)|\mathrm{d}^3v
      +\sum_J^{P}\int\bar f(\vec{v})|g_J^{-}(\vec{v},t)|
      |\dot g_J^{-}(\vec{v},t)|\mathrm{d}^3v 
    \right]. \label{classicalH4}
\end{equation}
%
We can observe that $\bar{f}(\vec{v})$ in (\ref{classicalH4}) is always
positive. Also the deviation and its derivative are positive, then all the
expression is positive. However, the global sign makes the derivative of the
$H'$ with respect to time is always less to zero. With this we proved that
$\frac{dH'}{dt}<0$. If the system is in equilibrium,
$g_{J}(\vec{v},t)=\dot g_J(\vec{v},t)=0$, therefore $\frac{dH'}{dt}=0$.

Joining both results, we can say the following statement:
Consider a classical gas in a total volume $V$ (divided in $K$ cells of equal
volume elements), total energy $E$ and total number of free particles $N$.
Consider also the system has inhomogeneities and suffers a relaxation process.
If we define the following functional
%
\begin{equation}\label{CH3}
   H'(t)=\sum_{M=1}^{K}\int f_M(\vec{v},t) \ln f_M(\vec{v},t)\mathrm{d}^3v,
\end{equation}
%
where $f_M(\vec{v},t)$ is the local distribution function of each cell in the
system, then in the first-order approximation
%
\begin{equation}
    \frac{dH'}{dt} \leq 0.
\end{equation}
%
This statement will be the classical $H$ theorem with inhomogeneities.

In the next section, we present a proposal quantum version of the
$H$ theorem defined by Tolman and our proposal quantum version of the $H$
theorem applying the method of the volume divided into cells.


