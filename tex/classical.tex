%\section{Classical}
%\subsection{Boltzmann transport equation}
%\subsection{H theorem and the Maxwell-Boltzmann distribution}
%Todo esto de arriba en la forma tradicional y muy resumido. Solo remarcando lo fundamental.
%\subsection{Out of equilibrium, non-homogeneous distributions}.
%Entrar a la generalizacion, ahi si de manera muy detallada y con amplitud discutida
%fisicamente, del tratamiento con las celdas. Es decir, la demostracion del teorema H
%clasico pero con el enfoque de funciones de distribucion no homogeneas y por tanto
%con el tratamiento de celdas.

%----------------------------------

\section{Classical scheme}\label{sec:classical}

Classical statistical mechanics is theoretically founded on the Boltzmann transport
equation and the hypo of molecular chaos. The fundamental connection
between the microscopic nature of matter and the phenomenological laws of thermodynamics
is the $H$-theorem.

\subsection{The Boltzmann transport equation}

From the Boltzmann kinetic theory of gases, the probability for a particle to be in a given
state of position and velocity at a time $t$,
$f(\vec r,\vec v,t)$ (\textit{aka} the distribution function), must satify
the transport equation (Eq.~(\ref{eq:transport}), below).
This equation is traditionally obtained by exploiting the volume-invariance of the $\mu$-space
(\textit{i.e.} the space formed by the direct product of position and velocity)
and considering that elastic collisions constitute the predominant mechanism
for exchanging energy among particles \cite{bib:huang}.
%
\begin{equation}\label{eq:transport}
	%\left(\frac{\partial f_1}{\partial t}\right)_{\mathrm{coll}}=
  \left(
    \frac{\partial}{\partial t}+\vec{v}_1 \cdot\nabla_{\vec r}
    +\frac{\vec{F}}{m}\cdot\nabla_{\vec v_1}
  \right)f_1=\int\mathrm{d}\Omega\int
    \mathrm{d}^{3}v_2\sigma(\Omega)|\vec{v}_1-\vec{v}_2|(\tilde f_2\tilde f_1-f_2f_1).
\end{equation}
%
In Eq. (\ref{eq:transport}), $\Omega$ is the solid angle, $\sigma$ is the
scattering cross section, $\vec F$ the
external force applied to the system, and $f_1$ and $f_2$ ($\tilde f_1$ and $\tilde f_2$) are the distribution
functions of particles 1 and 2, respectively, before (after) the collision.

Particle dynamics and external forces are described
by the left-hand side of Eq.~(\ref{eq:transport}), and the right-hand side
is derived considering binary collisions between particles and
the \textit{molecular chaos hypothesis}, \textit{i.e.} it is assumed that
the coordinates and velocities of the particles are not time-correlated.


%-------------------------
\subsection{A brief summary of the H-theorem and the Maxwell-Boltzmann distribution}

The evolution of a dilute gas towards thermodynamic equilibrium is frequently
addressed by first defining the functional \cite{bib:tolman,bib:huang}
%
\begin{equation}\label{eq:hbfunctional}
   H_{B}=\int f(\vec{v},t) \ln f(\vec{v},t) \mathrm{d}^{3}v.
\end{equation} 
%
Here, $f(\vec{v},t)$ is a homogeneous distribution function, \textit{i.e.}
$f$ is assumed to be independent of $\vec r$.
The functional $H_B$,
originally introduced by Boltzmann in 1872,
describes a diluted gas, at temperature $T$, with a total energy $E$, a total
number of free classical particles $N$, and occupying a volume $V$.
(We deviate from the custommary notation in order to
clearly distinguish the Boltzmann functional $H$ from our proposals, and
$f_B$ to denote the Maxwell-Boltzmann distribution function.)

Under this scheme, the time-evolution of the gas towards equilibrium is obtained through the
$H$-theorem. This theorem establishes that if $f(\vec{v},t)$ satisfies the Boltzmann
transport equation and the molecular chaos hypothesis is valid, then
$dH_B/dt\leq0$ in general, and if $dH_B/dt=0$, then the system is in the equilibrium state.
The $H$-theorem is proven by taking the time derivative
of Eq. (\ref{eq:hbfunctional}) and using the Boltzmann transport equation
(\ref{eq:transport}) \cite{bib:huang}. The $H$-theorem fixes the direction of time-evolution
of systems in nature, which is consistent with the second law of thermodynamics; in fact,
$H_B$ can be associated with an entropy density.

On the other hand, considering a diluted gas in equilibrium (\textit{i.e.}
$(\partial f/\partial t)=0$),
with no applied external forces (\textit{i.e.} $f$ is independent of $\vec r$), 
it can be proven that the equilibrium distribution function obtained from
Eq.~(\ref{eq:transport}) is precisely the Maxwell-Boltzmann
distribution function. The proof of the above commences by finding the sufficient
condition for $f$ to render a null r.h.s. of Eq. (\ref{eq:transport}). Such an
$f$, which we denote here as $f_0$, must satisfy:
%
\begin{equation}\label{eq:neccondf0}
    f_0(\vec{v}'_2)f_0(\vec{v}'_1)- f_0(\vec{v}_2)f_0(\vec{v}_1)=0.
\end{equation}
%
Subsequently, the Maxwell-Boltzmann distrubution function can be obtained by
taking the logarithm to Eq.~(\ref{eq:neccondf0}), and exploiting mechanical
conserved quantities (see \cite[ch. 4.2]{bib:huang}).

%---------------------------
\subsection{Alternative classical $H$-functional}

As we saw in the previous section,
the validity of the $H$-theorem relies significantly on assuming
that the distribution function is homogeneous and that the hypothesis of
molecular chaos is surmised. However, in order to consider systems whose
distribution functions are not necessarily homogeneous, which
allows for studying systems out of equilibrium,
we introduce a modified $H$-functional as follows. Hereafter, we will
use the term homogeneous distribution function to refer to a distribution
function thad does not depend on the coordinates, and will use primed functions
and quantities to differentiate the classical case (this Section) \textit{vs} the quantum
case (below, Section \ref{sec:quantum}).

Our proposed functional, denoted as $\mathcal{H}'$, describes a diluted classical
gas occupying a volume, $V$,
divided into $K$ cells, which, without loss of generality,
have local identical volumes (\textit{i.e.} $\delta V_M = V/K,\ M=1,\,\dots\,,K$).
Each cell has its own local
$H$-functional, $\mathcal{H}'_M$, and local homogeneous distribution function,
$f'_{M}(\vec{v},t)$; local number of particles, $\mathcal{N}'_M$;
local temperature, $T'_M$; and a local energy, $\mathcal{E}'_M$. As a whole, the system has a
global energy $E$, and a global number of free
classical particles $N$, and it is considered to be perfectly isolated (\textit{i.e.} the
global system is a microcanonical ensemble). We also assume that the number of particles
in each cell is sufficiently large, so as to obtain accurate averages.
The functional takes the form
%
\begin{equation}\label{eq:CH2}
   \mathcal{H}'(t)=\sum_{M=1}^{K}\int_{\delta V_M} f'_M(\vec{v},t) \ln f'_M(\vec{v},t)\mathrm{d}^3v.
\end{equation}
%


The distribution functions, $\{f'_M(\vec{v},t)\}$, depend implicitly on the position of the cells,
relative to the global system, and on the velocity
$\vec{v}$ and time $t$. Each $f'_M$ can be formally extended to the complete coordinate space
by defining each $f'_M$ to be zero outside the $M$-th cell, in such a manner that
the distribution function of the complete system is a piece-wise sum
of $\{f'_M(\vec v,t)\}$:
%
\begin{equation}
   f'(\vec r,\vec v,t)=\sum_{M=1}^Kf'(\vec r_M,\vec v,t)=\sum_{M=1}^Kf'_M(\vec v,t).
\end{equation}
%
Here $\vec r_M$ is the center of the cell of index $M$, and $f'_M(\vec v,t)\neq f'_N(\vec v,t)$
for $M\neq N$. This extended definition
allows us to omit the symbol $\delta V_M$ on all integrals performed over the cell volume.
In terms of $f'_M(\vec{v},t)$ and the energy, $\epsilon(\vec{v})$, we have:
%
\begin{subequations}\label{eq:cellrestrictions}
\begin{eqnarray}
    H_M' & = &  \int f'_M(\vec{v},t) \ln f'_{M}(\vec{v},t)
      \mathrm{d}^{3}v \label{eq:Hcell},\\
    \mathcal{N}_M' & = & \int f'_{M}(\vec{v} ,t) \mathrm{d}^{3}v,\label{eq:Ncell}\\
    \mathcal{E}_M' & = & \int f'_{M}(\vec{v},t)\epsilon(\vec{v}) \mathrm{d}^{3}v\label{eq:Ecell}.
\end{eqnarray}
\end{subequations}
%
Notice that assuming homogeneous distribution functions per cell means that we are surmising the 
local equilibrium hypothesis. In addition, the set $\{f'_{M}(\vec{v},t)\}$ must satisfy the following
microcanonical restrictions:
%
\begin{subequations}\label{eq:micro}
\begin{equation}\label{eq:micron}
    \sum_{M=1}^{K}\int f'_M(\vec{v},t)\mathrm{d}^3v =N
\end{equation}
and
\begin{equation}\label{eq:microe}
    \sum_{M=1}^{K}\int f'_M(\vec{v},t)\epsilon(\vec{v})\mathrm{d}^3v=E.
\end{equation}
\end{subequations}
%

We now use the variational method in order to maximize $H'$, applying the restrictions
given in Eqs.~(\ref{eq:cellrestrictions}) and (\ref{eq:micro}) through Lagrange
multipliers ($\{\alpha_M\}$ and $\{\beta_M\}$). This yields
%
\begin{eqnarray}
    \frac{\delta H'}{\delta f'_J(\vec{v}')} & = & \sum_{M=1}^{K}\int
      \frac{\delta}{\delta f'_J(\vec{v}')}\left[
        f'_M(\vec{v})\ln f'_M(\vec{v})
        \right]
       \mathrm{d}^3v -\sum_{M=1}^{K}\alpha_M\int
       	\frac{\delta f'_M(\vec{v})}{\delta f'_J(\vec{v}')}
      \mathrm{d}^3v\nonumber\\
    & & -\sum_{M=1}^{K}\beta_M\int\epsilon(\vec{v})
    	\frac{\delta f'_M(\vec{v})}{\delta f'_J(\vec{v}')}
      \mathrm{d}^3v \nonumber\\
    & = & \ln f'_J(\vec{v}')+1-\alpha_J-\beta_J \epsilon(\vec{v}')=0.
\end{eqnarray}
%
Solving the last line for $f'_J(\vec{v}')$ renders
%
\begin{comment}
\begin{eqnarray}
    \ln f'_J(\vec{v}') & = & \alpha_J+\beta_J \epsilon(\vec{v}')
       -1\quad\Rightarrow\quad f_J(\vec{v}')\\
    & = & e^{\alpha_J +\beta_J \epsilon(\vec{v}')-1}\nonumber\\
    & = & Ce^{\alpha_J+\beta_J \epsilon(\vec{v}') } \label{eq:relacion1},
\end{eqnarray}
\end{comment}
\begin{equation}\label{eq:relacion1}
	f'_J(\vec{v}') = C\exp\left({\alpha_J+\beta_J \epsilon(\vec{v}') }\right),
\end{equation}
%
where $C$ is a constant. We notice that using the variational procedure on
$H'$ predicts that the
distribution function of each cell has the form of the Maxwell-Boltzmann
distribution function, which is consistent with the local equilibrium assumption.

\subsubsection{Properties of $H'$ for systems in equilibrium}

If the complete system is in equilibrium without external forces applied to
the gas, from classical thermodynamics of
systems in equilibrium, we ascertain that the local number
of particles and the local
energy do not depend on the cell number, this is
%
\begin{subequations}\label{eq:eqNandE}
\begin{equation}
   \mathcal{N}'_M=\mathcal{N}'\equiv \bar{\mathcal{N}}'
\end{equation}
%
and
%
\begin{equation}
   \quad\mathcal{E}'_M=\mathcal{E}'\equiv\bar{\mathcal{E}}'.
\end{equation}
\end{subequations}
%
In Eqs. (\ref{eq:eqNandE}) the bar implies averaged properties over the complete system.
Moreover, the distribution function is homogeneous, hence $f'_M$
does not depend on the cell number $M$ (\textit{i.e.} $f'_M(\vec v,t)=f'(\vec v,t)$,
$\forall M$). Several properties follow directly from this, \textit{e.g.}
from Eqs.~(\ref{eq:eqNandE}) $E=\sum_M\mathcal{E}'_M=K\bar{\mathcal{E}}'$, $N=\sum_M\mathcal{N}'_M=K\bar{\mathcal{N}}'$ (here we have used Eqs. (\ref{eq:Ecell})
and (\ref{eq:Ncell})),
and Eq.~(\ref{eq:CH2}), in terms of Eq.~(\ref{eq:hbfunctional}), reduces to
%
\begin{equation}
    H'(t)=\int \sum_{M=1}^{K} [f'(\vec{v},t)\ln f'(\vec{v},t)] \mathrm{d}^{3}v
         = K\int  f'(\vec{v},t) \ln f'(\vec{v},t) \mathrm{d}^{3}v= K H_{B}(t).
\end{equation}
%
In order to identify $H'$ with the entropy, it remains to show that $H'(t)$ is extensive,
with respect to $Kf'_(\vec v,t)$.
This is shown by analyzing the following expression
%
\begin{eqnarray}\label{eq:sum-h}
    \int \big[Kf'(\vec{v},t)\big] \ln \big[Kf'(\vec{v},t)\big] \mathrm{d}^{3}v & = & 
        \int \big[(K\ln K)f'(\vec{v},t) + Kf'(\vec{v},t) \ln f'(\vec{v},t)\big]\mathrm{d}^{3}v\nonumber\\
        & = & K\int\Big\{ f'(\vec{v},t)\big[\ln K + \ln f'(\vec{v},t)\big] \Big\}\mathrm{d}^{3}v.
\end{eqnarray}
%
We observe that if the number of particles in the $\mu$-space, $f'(\vec{v},t)$, is much greater than
the number of cells $K$, then the first term of Eq.~(\ref{eq:sum-h}) is negligible, and consequently:
%
\begin{equation}\label{aditive-property}
    \int \big[Kf'(\vec{v},t)\big] \ln \big[Kf'(\vec{v},t)\big] \mathrm{d}^{3}v \simeq
    K\int f'(\vec{v},t) \ln f'(\vec{v},t) \mathrm{d}^{3}v,
\end{equation}
%
\textit{i.e.} $H_M'$ is extensive, and the sum $\sum_MH'_M$ is the $H$-functional of the complete
system, which reduces to the Boltzmann $H$-functional. Hence, $H'$ can also be identified with the
system's entropy.

Furthermore, in equilibrium the Lagrange multipliers are position- and time-independent,
thus $f'_M(\vec{v})$
reduces to
%
\begin{equation}\label{eq:fbardef}
    f'_M(\vec{v})=C\exp\left(\alpha+\beta \epsilon(\vec{v})\right)
    \equiv \bar{f}'(\vec{v}),\quad M=1,\,\dots\,,K.
\end{equation} 
%
The constant $C$ can be omitted, which is shown by defining the following $H''$
functional:
%
\begin{equation}
   H''(t)=\sum_{M=1}^{K}\int \left[f'_M(\vec{v},t)
    \ln f'_M(\vec{v},t)-f'_M(\vec{v},t)\right]\mathrm{d}^3v  \label{CH3},
\end{equation}
%
however $\sum_{M=1}^{K} \int f'_M(\vec{v},t)\mathrm{d}^3v =N$.
Since $N$ is a constant and we are mainly intersted in the time-derivative of $H''$,
$C$ can be conveniently omitted. In other words, $H'$ renders the Maxwell-Boltzmann
distribution function for systems in equilibrium.

\subsubsection{Alternative proof of the classical $H$-theorem}

Throughout this section, we will consider that the classical gas
starts at some time as an out-of-equilibrium system,
but in such a manner that it is intuitively not too far from equilibrium,
which ensures that the local equilibrium hypothesis remains valid during
the time evolution of the system.
Also we will use the following definitions:
\begin{subequations}\label{eq:restrictionoutsideclassical}
%
\begin{equation}
      \mathcal{N}'_M(t)=\int f'_{M}(\vec{v},t) \mathrm{d}^{3}v=
      \bar{\mathcal{N}}'+\Delta'_M(t)
\end{equation}
%
and
%
\begin{equation}
      \mathcal{E}'_M(t)=\int f'_{M}(\vec{v},t) \epsilon(\vec{v}) \mathrm{d}^{3}v=
      \bar{\mathcal{E}}'+ \delta'_M(t).
\end{equation}
\end{subequations}
%
Here $\bar {\mathcal{N}}'\ (=N/K)$ and $\bar{\mathcal{E}}'\ (=E/K)$ are
the local particle number and the local energy in equilibrium, respectively, which are given by:
%
    \begin{equation}
      \bar{\mathcal{N}}'=
      \int \bar{f}'(\vec{v}) \mathrm{d}^{3}v\quad\textrm{and} \quad
      \bar{\mathcal{E}}'=
      \int \bar{f}'(\vec{v})\epsilon(\vec{v}) \mathrm{d}^{3}v,
    \end{equation}
    %
where we have used $\bar{f}'(\vec{v})$ as defined in Eq.~(\ref{eq:fbardef}).
In Eqs.~(\ref{eq:restrictionoutsideclassical}), $\Delta'_M$ and $\delta'_M$
are considered deviations relative to
$\bar{\mathcal{N}}'$ and $\bar{\mathcal{E}}'$, respectively, and for systems that are not
too far away from equilibrium, it is reasonable to expect on the one hand that
$\Delta'_M(t)\ll\bar{\mathcal{N}}'$ and
$\delta'_M(t)\ll\bar{\mathcal{E}}'$, and on the other hand that $\Delta'_M$ and
$\delta'_M$ are sufficiently large compared with fluctuations of $\bar{\mathcal{N}}'$
and $\bar{\mathcal{E}}'$. Under the same perspective, we can assume that every local
distribution function can be written as
\begin{equation}\label{eq:firstorder}
   f'_{Mn}(\vec{v},t)=\bar{f}'_{n}(\vec{v})(1+g'_{Mn}(\vec{v},t)),\quad
   1\gg|g'_{Mn}(\vec{v},t)|.
\end{equation}
%

With the previous considerations, in the following we provide an alternative demonstration
of the $H$-theorem, considering the $H$-functional defined by Eq.~(\ref{eq:CH2}).

We commence by differentiating Eq.~(\ref{eq:CH2}) with respect to time,
%
\begin{equation}\label{eq:dH1}
    \frac{dH'}{dt}=\sum_{M=1}^{K}\int\left[
      1+\ln f'_M(\vec{v},t)
    \right]\dot f'_M(\vec{v},t)\mathrm{d}^3v.
\end{equation}
%
(Here and in what follows we use the standard notation $(d/dt)h\equiv\dot h$
for an arbitrary function $h$.) Substituting Eq.~(\ref{eq:firstorder}) into Eq.~(\ref{eq:dH1}) yields
%
\begin{equation}\label{eq:dH1-1}
    \frac{dH'}{dt}=\sum_{M=1}^{K}\int\bar f'(\vec{v}) \left[
      1+\ln \left\{
        \bar f'(\vec{v})+\bar f'(\vec{v})g'_M(\vec{v},t)
      \right\}
    \right]\dot g'_M(\vec{v},t)\mathrm{d}^3v.
\end{equation}
%
The logarithmic term of Eq.~(\ref{eq:dH1-1}) expanded up to the first-order term
of its Taylor series, around $g'_{nM}(\vec{v},t)=0$, is
%
\begin{equation}\label{lnapproximationclassical}
    \ln [\bar{f}'(\vec{v})+\bar{f}'(\vec{v}) g'_{M}(\vec{v},t)] \approx
    \ln [\bar{f}'(\vec{v})]+ g'_{M}(\vec{v},t),
\end{equation}
%
and and substituting this into Eq.~(\ref{eq:dH1-1}) gives
%
\begin{eqnarray}
    \frac{dH'}{dt}&=&\sum_{M=1}^{K} \int \bar f'(\vec{v})\left[
      1+\ln \bar f'(\vec{v})+g'_M(\vec{v},t)
    \right]\dot g'_M(\vec{v},t)\mathrm{d}^3v.
\end{eqnarray}
%

\commr{ CHECKPOINT So far, we have not restricted the initial condition of the system to be
of any particular form.}

using (\ref{eq:relacion1}) with homogeneous multipliers
%
\begin{equation}\label{dH1-2}
    \frac{dH'}{dt} = \sum_{M=1}^{K}\int\bar f(\vec{v})\left[
      \alpha+\beta \epsilon(\vec{v})
    \right]\dot g_M(\vec{v},t)\mathrm{d}^3v +\sum_{M=1}^{K}
    \int\bar f(\vec{v})g_M(\vec{v},t)\dot g_M(\vec{v},t)\mathrm{d}^3v . \nonumber 
\end{equation}
%
On the other hand, we obtain from the restrictions the following expressions
%
\begin{eqnarray}
    \int \bar{f}(\vec{v}) g_{M}(\vec{v},t) \mathrm{d}^{3}v=\Delta_M(t) \ \  &\Rightarrow&
    \ \  \int \bar{f}(\vec{v}) \dot{g}_{M}(\vec{v},t)\mathrm{d}^{3}v=\dot{\Delta}_M(t), \nonumber \\
    \int  \bar{f}(\vec{v}) g_{M}(\vec{v},t)\epsilon(\vec{v}) \mathrm{d}^{3}v=\delta_M(t) \ \  &\Rightarrow&
    \ \  \int \bar{f}(\vec{v}) \dot{g}_{M}(\vec{v},t)\epsilon(\vec{v}) \mathrm{d}^{3}v=\dot{\delta}_M(t), \nonumber 
\end{eqnarray}
%
and as a consequence of $\sum_{M=1}^{K} \Delta_M(t) =\sum_{M=1}^{K} \delta_M(t) =0$,
we find
%
\begin{equation}
    \sum_{M=1}^{K} \dot{\Delta}_M(t)  =\sum_{M=1}^{K} \dot{\delta}_{M}(t) =0.
\end{equation}
%
Then using the previous result to (\ref{dH1-2}), we find
%
\begin{equation}
  \frac{dH'}{dt} = \sum_{M=1}^{K}\int\bar f(\vec{v})g_M(\vec{v},t)\dot g_M(\vec{v},t)\mathrm{d}^3v.
\end{equation}
%
Now, the summation over $M$ will be expressed in two summations
%
\begin{equation}\label{classicalH3}
    \frac{dH'}{dt}=\sum_J^{L}\int
      \bar f(\vec{v})g_J^{+}(\vec{v},t)\dot g_J^{+}(\vec{v},t)\mathrm{d}^3v
      +\sum_J^{P}\int
        \bar f(\vec{v})g_J^{-}(\vec{v},t)\dot g_J^{-}(\vec{v},t)\mathrm{d}^3v.
\end{equation}
%
where $L+P=K$. The group of $L$ cells is such that cells have an excess of
particles or/and energy (from the mean value in equilibrium). In contrast,
the group of $P$ cells is such that cells have to miss particles or/and energy
(from the mean value in equilibrium). Besides, $\dot{g}^{+}_{J}$ represents the
change on the deviation on cells that have an excess of particles or energy
while $\dot{g}^{-}_{J}$  represents the change on the deviations on cells that
have missing particles or energy. 
On the other hand, $g^{+}_{J}$  represents the value of the deviation on cells
that have an excess of particles or energy. In contrast, $g^{-}_{J}$ represents
the value of the deviation on cells that have missing particles or energy.

Also, on the one hand, $\dot{g}^{+}_{J}<0$ describes the loss of particles
and/or energy and so, $g^{+}_{J}>0$. On the other hand, $\dot{g}^{-}_{J}>0$
describes the gain of particles and/or energy and therefore $g^{-}_{J}<0$. 

We sort the previous ideas in the following form
%
\begin{equation}\label{separacionclassical}
\begin{array}{rl}
  g^{+}_{J}=+|g^{+}_{J}|; & \dot{g}^{+}_{J}=-|\dot{g}^{+}_{J}|,\\
  g^{-}_{J}=-|g^{-}_{J}|; & \dot{g}^{-}_{J}=+|\dot{g}^{-}_{J}|,
 \end{array}
\end{equation}
%
and consequently, (\ref{classicalH3}) obtains the following form
%
\begin{equation}
    \frac{dH'}{dt}=-\left[
      \sum_J^{L}\int\bar f(\vec{v})|g_J^{+}(\vec{v},t)|
        |\dot g_J^{+}(\vec{v},t)|\mathrm{d}^3v
      +\sum_J^{P}\int\bar f(\vec{v})|g_J^{-}(\vec{v},t)|
      |\dot g_J^{-}(\vec{v},t)|\mathrm{d}^3v 
    \right]. \label{classicalH4}
\end{equation}
%
We can observe that $\bar{f}(\vec{v})$ in (\ref{classicalH4}) is always
positive. Also the deviation and its derivative are positive, then all the
expression is positive. However, the global sign makes the derivative of the
$H'$ with respect to time is always less to zero. With this we proved that
$\frac{dH'}{dt}<0$. If the system is in equilibrium,
$g_{J}(\vec{v},t)=\dot g_J(\vec{v},t)=0$, therefore $\frac{dH'}{dt}=0$.

Joining both results, we can say the following statement:
Consider a classical gas in a total volume $V$ (divided in $K$ cells of equal
volume elements), total energy $E$ and total number of free particles $N$.
Consider also the system has inhomogeneities and suffers a relaxation process.
If we define the following functional
%
\begin{equation}\label{CH3}
   H'(t)=\sum_{M=1}^{K}\int f_M(\vec{v},t) \ln f_M(\vec{v},t)\mathrm{d}^3v,
\end{equation}
%
where $f_M(\vec{v},t)$ is the local distribution function of each cell in the
system, then in the first-order approximation
%
\begin{equation}
    \frac{dH'}{dt} \leq 0.
\end{equation}
%
This statement will be the classical $H$-theorem with inhomogeneities.

In the next section, we present a proposal quantum version of the
$H$-theorem defined by Tolman and our proposal quantum version of the $H$
theorem applying the method of the volume divided into cells.


