\section{Classical scheme}\label{sec:classical}

The Boltzmann kinetic theory of gases represents a fundamental connection
between the microscopic nature of matter and the phenomenological macroscopic laws of classical thermodynamics.
The stochasticity introduced by the molecular chaos hypothesis in the otherwise deterministic kinetics of the
particles allows for the demonstration 
of the celebrated Boltzmann $H$-theorem. In contrast, in this article, we propose an alternative approach 
developed using a variational procedure applied to an $H$-functional.
We start this section by briefly accounting for the important elements of the 
standard derivation of the Boltzmann transport equation and demonstrating the $H$-theorem,
such as they are presented in classical textbooks \cite{bib:huang}. 

\subsection{The Boltzmann transport equation}

The first step in the Boltzmann kinetic theory of gases is defining the distribution function, 
$f(\vec{r},\vec{v},t)$, as the average number of molecules that, at time $t$, have 
position $\vec r$ and velocity $\vec v$, and are contained in a $\mu$-space volume element $d^3rd^3v$.
Assuming a deterministic Newtonian description of molecular motion, 
as well as the invariance of the $\mu$-space volume measure, one arrives at the Boltzmann rate equation:
%
\begin{equation}
	f(\vec{r}+\vec{v}\delta t, \vec{v}+\vec{F} \delta t, t+\delta t)=f(\vec{r},\vec{v},t)+\left( \frac{\partial f}{\partial t} \right)_{\textrm{coll}} \delta t.
\end{equation}
%
Here, the term $\left(\partial f/\partial t\right)_{\textrm{coll}}$
describes the in and out fluxes from and towards the volume element, due to the collisions.
Subsequently, from the previous equation, the integro-differential Boltzmann transport equation is obtained:
%
\begin{equation}\label{eq:transport}
	%\left(\frac{\partial f_1}{\partial t}\right)_{\mathrm{coll}}=
  \left(
    \frac{\partial}{\partial t}+\vec{v}_1 \cdot\nabla_{\vec r}
    +\frac{\vec{F}}{m}\cdot\nabla_{\vec v_1}
  \right)f_1=\int\mathrm{d}\Omega\int
    \mathrm{d}^{3}v_2\sigma(\Omega)|\vec{v}_1-\vec{v}_2|(\tilde f_2\tilde f_1-f_2f_1).
\end{equation}
%
In Eq. (\ref{eq:transport}), $\Omega$ is the solid angle, $\sigma$ is the
scattering cross section, $\vec F$ the
external force applied to the system, and $f_1$ and $f_2$ ($\tilde f_1$ and $\tilde f_2$) are the distribution
functions of particles 1 and 2, respectively, before (and after) the collision.

Particle dynamics and the effects of external forces are described
by the left-hand side of Eq.~(\ref{eq:transport}). The right-hand side
is derived by considering binary collisions between particles and accepting
the \textit{molecular chaos hypothesis}, \textit{i.e.} it is assumed that
the positions and velocities of the particles are not time-correlated.


%-------------------------
\subsection{A summary of the H-theorem and the Maxwell-Boltzmann distribution}

The evolution of a dilute gas towards thermodynamic equilibrium is frequently
addressed by first defining the $H$-functional \cite{bib:tolman,bib:huang}:
%
\begin{equation}\label{eq:hbfunctional}
   H_{B}=\int f(\vec{v},t) \ln f(\vec{v},t) \mathrm{d}^{3}v.
\end{equation} 
%
Notice that $f_B(\vec{v},t)$ is a spatially homogeneous distribution function.
The functional $H_B$,
originally introduced by Boltzmann in 1872,
describes a dilute gas occupying a volume $V$, at temperature $T$, with total energy $E$, and total
number of free classical particles $N$.
To clearly distinguish the Boltzmann functional $H_B$, we denote hereafter
the Maxwell-Boltzmann distribution function as $f_B$.

The physically correct spontaneous time-evolution of an out of equilibrium dilute gas 
is corroborated by the $H$-theorem.
This theorem establishes that if a) the homogeneous
function $f(\vec{v},t)$ satisfies the Boltzmann
transport equation and b) the molecular chaos hypothesis is valid, then
the system evolves in such a manner that $dH_B/dt\leq0$, and if $dH_B/dt=0$, 
then the system is in the equilibrium state. 
The $H$-theorem is straightforward to prove using Eq.~(\ref{eq:transport}) \cite{bib:huang}, 
and it assures the consistency between our microscopic approach to describe the system’s spontaneous 
time-evolution and the phenomenological observations
established by the second law of classical thermodynamics; in fact,
$H_B$ can be associated with an entropy density.

On the other hand, considering a dilute gas in equilibrium
with no applied external forces, \textit{i.e.}
$(\partial f/\partial t)=0$ and $f$ is independent of $\vec r$, 
we can directly prove that the equilibrium distribution function obtained from
Eq.~(\ref{eq:transport}) is precisely the Maxwell-Boltzmann
distribution function. The proof of the above first requires identification of the sufficient
condition for $f$ to render a null r.h.s. of Eq. (\ref{eq:transport}). Such an
$f$, which we denote here as $f_0$, must satisfy
%
\begin{equation}\label{eq:neccondf0}
    f_0(\vec{v}'_2)f_0(\vec{v}'_1)- f_0(\vec{v}_2)f_0(\vec{v}_1)=0.
\end{equation}
%
Subsequently, the Maxwell-Boltzmann distribution function can be obtained by
taking the logarithm of Eq.~(\ref{eq:neccondf0}) and conserved mechanical
quantities (see \cite[ch. 4.2]{bib:huang}).

Before introducing our proposed $H$-functional, we must state that in defining
$H_B$, Eq.~(\ref{eq:hbfunctional}), it is assumed that the distribution function 
$f$ is spatially homogeneous. This assumption simplifies the demonstration 
of the $H$ theorem. However, it also introduces a limited conception
of the out-of-equilibrium condition of the gas. Given the relatively simple nature of a dilute 
gas, one of the salient features of an out-of-equilibrium condition is the existence of
inhomogeneities in the system, which is not considered in the above.

%---------------------------
\subsection{Non-homogeneous classical $H$-functional}

As we saw in the previous section,
the validity of the $H$-theorem relies significantly on assuming
that the distribution function is homogeneous and the
molecular chaos hypothesis is fulfilled. To extend the previous procedure to systems with non-homogeneous
distribution functions, which might allow for the study of systems
in a more general out-of-equilibrium condition, we introduce a modified $H$-functional. 
For the sake of clarity and simplicity, we will use primed functions
and quantities to denote the classical case to differentiate them from the quantum
analogues.

Our proposed classical $H$-functional, denoted as $\mathcal{H}’ $, describes a dilute classical
gas occupying a volume $V$. In our theoretical treatment, we divide this volume
into $K$ cells, which, without loss of generality,
have identical volumes, $\delta V_M = V/K,\ M=1,\,\dots\,,K$.
Each cell of index $M$ has the following local functions, properties, and variables:
an $H$-functional, $\mathcal{H}'_M$, a homogeneous distribution function, $f'_{M}(\vec{v},t)$,
number of particles, $\mathcal{N}'_M$, temperature, $T'_M$, and 
energy, $\mathcal{E}'_M$. Taken as a whole, the system has an
energy $E$, and a global number of free classical particles $N$. We also assume that the system is 
perfectly isolated, and that the number of particles
in each cell is sufficiently large, so as to obtain accurate averages.
We start our analysis by proposing the following inhomogeneous $H$-functional:
%
\begin{equation}\label{eq:cHdef}
   \mathcal{H}'(t)=\sum_{M=1}^{K}\int_{\delta V_M} f'_M(\vec{v},t) \ln f'_M(\vec{v},t)\mathrm{d}^3v.
\end{equation}
%


The distribution functions, $\{f'_M(\vec{v},t)\}$, depend implicitly on the position of the cells,
relative to the global system, and on the velocity
$\vec{v}$ and time $t$. Notice that each $f'_M$ can be formally extended to the complete coordinate space
by defining each $f'_M$ to be zero outside the $M$-th cell, in such a manner that
the distribution function of the complete system is a piece-wise sum
of $\{f'_M(\vec v,t)\}$:
%
\begin{equation}
   f'(\vec r,\vec v,t)=\sum_{M=1}^Kf'(\vec r_M,\vec v,t)=\sum_{M=1}^Kf'_M(\vec v,t).
\end{equation}
%
Here, $\vec r_M$ is the center of the cell of index $M$, and $f'_M(\vec v,t)\neq f'_N(\vec v,t)$
for $M\neq N$. This extended definition
allows us to omit the symbol $\delta V_M$ in all integrals performed over the cell volume.
In terms of $f'_M(\vec{v},t)$ and a local variable of energy, $\epsilon(\vec{v})$, we have
%
\begin{subequations}\label{eq:cellrestrictions}
\begin{eqnarray}
    \mathcal{H}_M' & = &  \int f'_M(\vec{v},t) \ln f'_{M}(\vec{v},t)
      \mathrm{d}^{3}v \label{eq:Hcell},\\
    \mathcal{N}_M' & = & \int f'_{M}(\vec{v} ,t) \mathrm{d}^{3}v,\label{eq:Ncell}\\
    \mathcal{E}_M' & = & \int f'_{M}(\vec{v},t)\epsilon(\vec{v}) \mathrm{d}^{3}v\label{eq:Ecell}.
\end{eqnarray}
\end{subequations}
%
Notice that assuming every $f'_M$ to be homogeneous implies that we are accepting the validity of the
local equilibrium hypothesis. In addition, the set $\{f'_{M}(\vec{v},t)\}$ must satisfy the following restrictions:
%
\begin{subequations}\label{eq:micro}
\begin{equation}\label{eq:micron}
    \sum_{M=1}^{K}\int f'_M(\vec{v},t)\mathrm{d}^3v =N
\end{equation}
and
\begin{equation}\label{eq:microe}
    \sum_{M=1}^{K}\int f'_M(\vec{v},t)\epsilon(\vec{v})\mathrm{d}^3v=E.
\end{equation}
\end{subequations}
%

We now use the variational method to find the extremal of $\mathcal{H}'$,
consistent with restrictions
(\ref{eq:cellrestrictions}) and (\ref{eq:micro}) together with the corresponding Lagrange
multipliers $\{\alpha_M\}$ and $\{\beta_M\}$. This yields
%
\begin{eqnarray}\label{eq:deltaHpdeltafpj}
    \frac{\delta \mathcal{H}'}{\delta f'_J(\vec{v}')} & = & \sum_{M=1}^{K}\int
      \frac{\delta}{\delta f'_J(\vec{v}')}\left[
        f'_M(\vec{v})\ln f'_M(\vec{v})
        \right]
       \mathrm{d}^3v -\sum_{M=1}^{K}\alpha_M\int
       	\frac{\delta f'_M(\vec{v})}{\delta f'_J(\vec{v}')}
      \mathrm{d}^3v\nonumber\\
    & & -\sum_{M=1}^{K}\beta_M\int\epsilon(\vec{v})
    	\frac{\delta f'_M(\vec{v})}{\delta f'_J(\vec{v}')}
      \mathrm{d}^3v \nonumber\\
    & = & \ln f'_J(\vec{v}')+1-\alpha_J-\beta_J \epsilon(\vec{v}')=0.
\end{eqnarray}
%
Solving the last line for $f'_J(\vec{v}')$ renders
\begin{equation}\label{eq:relacion1}
	f'_J(\vec{v}') = C\exp\left({\alpha_J+\beta_J \epsilon(\vec{v}') }\right)
\end{equation}
%
where $C$ is a constant. We notice that by applying the variational procedure on
$\mathcal{H} ‘$, we predict that when equilibrium is reached, the
distribution function of each cell has the form of the Maxwell-Boltzmann
distribution function, which is consistent with the local equilibrium assumption.

\subsubsection{Properties of $\mathcal{H}'$ for systems in equilibrium}

If the complete system is in equilibrium without external forces applied to
the gas, from classical thermodynamics of
systems in equilibrium, we ascertain that the local number
of particles and the local
energy do not depend on the cell number. In a statistical sense, this is
%
\begin{subequations}\label{eq:eqNandE}
\begin{equation}
   \mathcal{N}'_M=\mathcal{N}'\equiv \bar{\mathcal{N}}'
\end{equation}
%
and
%
\begin{equation}
   \quad\mathcal{E}'_M=\mathcal{E}'\equiv\bar{\mathcal{E}}'.
\end{equation}
\end{subequations}
%
In Eqs. (\ref{eq:eqNandE}) the bar implies averaged properties over the complete system.
Moreover, the global distribution function is homogeneous, hence $f'_M$
does not depend on the cell number $M$ (\textit{i.e.} $f'_M(\vec v,t)=f'(\vec v,t)$,
$\forall M$). Several properties arise directly from this, \textit{e.g.}
from Eqs.~(\ref{eq:eqNandE}) $E=\sum_M\mathcal{E}'_M=K\bar{\mathcal{E}}'$, $N=\sum_M\mathcal{N}'_M=K\bar{\mathcal{N}}'$. Here we have used Eqs. (\ref{eq:Ecell}) and (\ref{eq:Ncell}).
Eq.~(\ref{eq:cHdef}), in terms of Eq.~(\ref{eq:hbfunctional}), can be rewritten as:
%
\begin{equation}
    \mathcal{H}'(t)=\int \sum_{M=1}^{K} [f'(\vec{v},t)\ln f'(\vec{v},t)] \mathrm{d}^{3}v
         = K\int  f'(\vec{v},t) \ln f'(\vec{v},t) \mathrm{d}^{3}v= K H_{B}(t).
\end{equation}
%

To identify $\mathcal{H}'$ with the entropy, we need to show that $\mathcal{H}'(t)$ is extensive,
with respect to $Kf'(\vec v,t)$.
This is shown by analyzing the following expression:
%
\begin{eqnarray}\label{eq:sum-h}
    \int \big[Kf'(\vec{v},t)\big] \ln \big[Kf'(\vec{v},t)\big] \mathrm{d}^{3}v & = & 
        \int \big[(K\ln K)f'(\vec{v},t) + Kf'(\vec{v},t) \ln f'(\vec{v},t)\big]\mathrm{d}^{3}v\nonumber\\
        & = & K\int\Big\{ f'(\vec{v},t)\big[\ln K + \ln f'(\vec{v},t)\big] \Big\}\mathrm{d}^{3}v.
\end{eqnarray}
%
We observe that if the number of particles in the $\mu$-space, $f'(\vec{v},t)$, is much larger than
the number of cells, $K$, then the first term of Eq.~(\ref{eq:sum-h}) is negligible, and consequently:
%
\begin{equation}\label{aditive-property}
    \int \big[Kf'(\vec{v},t)\big] \ln \big[Kf'(\vec{v},t)\big] \mathrm{d}^{3}v \approx
    K\int f'(\vec{v},t) \ln f'(\vec{v},t) \mathrm{d}^{3}v,
\end{equation}
%
\textit{i.e.}, $\mathcal{H}'_M$ is extensive, and the sum $\sum_M\mathcal{H}'_M$ is the $H$-functional of the complete
system, which reduces to the Boltzmann $H$-functional.
Therefore, $\mathcal{H}’ $ can be identified with the entropy density of the system.

Furthermore, in equilibrium, the Lagrange multipliers are position- and time-independent,
thus $f'_M(\vec{v})$
reduces to
%
\begin{equation}\label{eq:fbardef}
    f'_M(\vec{v})=C\exp\left(\alpha+\beta \epsilon(\vec{v})\right)
    \equiv \bar{f}'(\vec{v}),\quad M=1,\,\dots\,,K.
\end{equation} 
%
The constant $C$ can be omitted, which is shown by defining the following $\mathcal{H}''$
functional:
%
\begin{equation}\label{CH3}
   \mathcal{H}''(t)=\sum_{M=1}^{K}\int \left[f'_M(\vec{v},t)
    \ln f'_M(\vec{v},t)-f'_M(\vec{v},t)\right]\mathrm{d}^3v.
\end{equation}
%
Since $\sum_{M=1}^{K} \int f'_M(\vec{v},t)\mathrm{d}^3v =N$ (a constant),
and because we are mainly interested in the time-derivative of $\mathcal{H}''$,
$C$ can be conveniently omitted. In other words, $\mathcal{H}'$ leads to the Maxwell-Boltzmann
distribution function of systems in equilibrium.

\subsubsection{Proof of the $H$-theorem for non-homogeneous distributions}

Throughout this section, we will consider a classical gas
with an initial condition close to the equilibrium,
which ensures that the local equilibrium hypothesis remains valid during
the time-evolution of the system.
Also, we will use the following definitions for the deviations of concentration 
and energy, relative to the equilibrium values:
\begin{subequations}\label{eq:restrictionoutsideclassical}
%
\begin{equation}
      \mathcal{N}'_M(t)=\int f'_{M}(\vec{v},t) \mathrm{d}^{3}v=
      \bar{\mathcal{N}}'+\Delta'_M(t)
\end{equation}
%
and
%
\begin{equation}
      \mathcal{E}'_M(t)=\int f'_{M}(\vec{v},t) \epsilon(\vec{v}) \mathrm{d}^{3}v=
      \bar{\mathcal{E}}'+ \delta'_M(t).
\end{equation}
\end{subequations}
%
Here $\bar {\mathcal{N}}' =N/K$ and $\bar{\mathcal{E}}'=E/K$ are
the cell particle number and the cell energy in equilibrium, respectively, which are given by
%
    \begin{equation}
      \bar{\mathcal{N}}'=
      \int \bar{f}'(\vec{v}) \mathrm{d}^{3}v\quad\textrm{and} \quad
      \bar{\mathcal{E}}'=
      \int \bar{f}'(\vec{v})\epsilon(\vec{v}) \mathrm{d}^{3}v,
    \end{equation}
    %
where we have used $\bar{f}'(\vec{v})$ as defined in Eq.~(\ref{eq:fbardef}).
In Eqs.~(\ref{eq:restrictionoutsideclassical}), $\Delta'_M$ and $\delta'_M$
are considered deviations relative to
$\bar{\mathcal{N}}'$ and $\bar{\mathcal{E}}'$, respectively. For systems that are
sufficiently close to equilibrium, it is reasonable to expect first that
$\Delta'_M(t)\ll\bar{\mathcal{N}}'$ and
$\delta'_M(t)\ll\bar{\mathcal{E}}'$, and second that $\Delta'_M$ and
$\delta'_M$ are sufficiently large compared to the fluctuations of $\bar{\mathcal{N}}'$
and $\bar{\mathcal{E}}'$. Similarly, we can assume that every local
distribution function can be written as
\begin{equation}\label{eq:firstorder}
   f'_{M}(\vec{v},t)=\bar{f}'(\vec{v})(1+g'_{M}(\vec{v},t)),\quad
   1\gg|g'_{M}(\vec{v},t)|.
\end{equation}
%

With the previous considerations, in the following, we proof an alternative
$H$-theorem, considering the $H$-functional, $\mathcal{H}'$, defined by Eq.~(\ref{eq:cHdef}).

We commence by differentiating Eq.~(\ref{eq:cHdef}) with respect to time:
%
\begin{equation}\label{eq:dH1}
    \frac{d\mathcal{H}'}{dt}=\sum_{M=1}^{K}\int\left[
      1+\ln f'_M(\vec{v},t)
    \right]\dot f'_M(\vec{v},t)\mathrm{d}^3v.
\end{equation}
%
(Starting here, we use the standard notation $\dot h\equiv(dh/dt)$).
Substituting Eq.~(\ref{eq:firstorder}) into Eq.~(\ref{eq:dH1}) yields
%
\begin{equation}\label{eq:dH1-1}
    \frac{d\mathcal{H}'}{dt}=\sum_{M=1}^{K}\int\bar f'(\vec{v}) \left[
      1+\ln \left\{
        \bar f'(\vec{v})+\bar f'(\vec{v})g'_M(\vec{v},t)
      \right\}
    \right]\dot g'_M(\vec{v},t)\mathrm{d}^3v.
\end{equation}
%
The logarithmic term of Eq.~(\ref{eq:dH1-1}) expanded up to the first-order term
of its Taylor series, around $g'_{M}(\vec{v},t)=0$, is
%
\begin{equation}\label{lnapproximationclassical}
    \ln [\bar{f}'(\vec{v})+\bar{f}'(\vec{v}) g'_{M}(\vec{v},t)] \approx
    \ln [\bar{f}'(\vec{v})]+ g'_{M}(\vec{v},t),
\end{equation}
%
and substituting this into Eq.~(\ref{eq:dH1-1}) gives
%
\begin{eqnarray}
    \frac{d\mathcal{H}'}{dt}&=&\sum_{M=1}^{K} \int \bar f'(\vec{v})\left[
      1+\ln \bar f'(\vec{v})+g'_M(\vec{v},t)
    \right]\dot g'_M(\vec{v},t)\mathrm{d}^3v.
\end{eqnarray}
%
Substituting $\ln \bar f'(\vec{v})=\exp\left(\alpha+\beta \epsilon(\vec{v})\right)$,
see Eq.~(\ref{eq:fbardef}) and subsequent text, and omitting $C$ we obtain
%
\begin{equation}\label{eq:dH1-2}
    \frac{d\mathcal{H}'}{dt} = \sum_{M=1}^{K}\int\bar f(\vec{v})\left[
      \alpha+\beta \epsilon(\vec{v})
    \right]\dot g_M(\vec{v},t)\mathrm{d}^3v +\sum_{M=1}^{K}
    \int\bar f(\vec{v})g_M(\vec{v},t)\dot g_M(\vec{v},t)\mathrm{d}^3v .
\end{equation}
%
From the definitions of $\mathcal{N}'_M$ and $\mathcal{E}'_M$ 
---Eqs.~(\ref{eq:restrictionoutsideclassical})---and $f'_{M}(\vec{v},t)$
---Eq.~(\ref{eq:firstorder})---it is straightforward to show that
%
\begin{subequations}\label{eq:classsumdotdeltaseq0}
\begin{eqnarray}
    \int \bar{f}(\vec{v}) g_{M}(\vec{v},t) \mathrm{d}^{3}v=\Delta_M(t) \ \  &\Rightarrow&
    \ \  \int \bar{f}(\vec{v}) \dot{g}_{M}(\vec{v},t)\mathrm{d}^{3}v=\dot{\Delta}_M(t), \\
    \int  \bar{f}(\vec{v}) g_{M}(\vec{v},t)\epsilon(\vec{v}) \mathrm{d}^{3}v=\delta_M(t) \ \  &\Rightarrow&
    \ \  \int \bar{f}(\vec{v}) \dot{g}_{M}(\vec{v},t)\epsilon(\vec{v}) \mathrm{d}^{3}v=\dot{\delta}_M(t),
\end{eqnarray}
\end{subequations}
%
and as a consequence of $\sum_{M=1}^{K} \Delta_M(t) =\sum_{M=1}^{K} \delta_M(t) =0$,
we find
%
\begin{equation}\label{eq:sumderdeltaseq0}
    \sum_{M=1}^{K} \dot{\Delta}_M(t)  =\sum_{M=1}^{K} \dot{\delta}_{M}(t) =0.
\end{equation}
%
Therefore, due to
Eqs.~(\ref{eq:classsumdotdeltaseq0}) and (\ref{eq:sumderdeltaseq0}), Eq.~(\ref{eq:dH1-2}) simplifies to
%
\begin{equation}\label{eq:dHpdtbefgdecclass}
  \frac{d\mathcal{H}'}{dt} = \sum_{M=1}^{K}\int\bar f(\vec{v})g_M(\vec{v},t)\dot g_M(\vec{v},t)\mathrm{d}^3v.
\end{equation}%
To clearly determine the time-evolution of Eq.~(\ref{eq:dHpdtbefgdecclass}),
we split the summation over $M$ into two terms:
%
\begin{equation}\label{eq:classicalH3}
    \frac{d\mathcal{H}'}{dt}=\sum_J^{L}\int
      \bar f'(\vec{v}){g'_I}^{+}(\vec{v},t){{\dot{g}}'}_I\phantom{.}\!^{+}(\vec{v},t)\mathrm{d}^3v
        +\sum_J^{P}\int
      \bar f'(\vec{v}){g'_J}^{-}(\vec{v},t){{\dot{g}}'}_J\phantom{.}\!^{-}(\vec{v},t)\mathrm{d}^3v
\end{equation}
%
where $L+P=K$. The above split is made based on the assumption that for any given initial
state of the system, at $t_0$, some cells will have either a $g'_I(\vec{v},t_0)\geq0$
or a $g'_J(\vec{v},t_0)<0$,
which we denote as ${{\dot{g}}'}_I\phantom{.}\!^{+}(\vec{v},t)$
or ${{\dot{g}}'}_J\phantom{.}\!^{-}(\vec{v},t)$, respectively.

If the system's initial state is sufficiently close to equilibrium, it is
physically appropriate
to assume that $\left|g'_M(\vec{v},t_0)\right|\to 0$ as $t\to\infty$ in a monotonous manner,
thus ${{\dot{g}}'}_I\phantom{.}\!^{+}(\vec{v},t)\leq0$ and
${{\dot{g}}'}_J\phantom{.}\!^{-}(\vec{v},t)>0$, for $t\geq t_0$. Consequently,
Eq.~(\ref{eq:classicalH3}) can be re-written as
%
\begin{equation}\label{eq:classicalH4}
    \frac{d\mathcal{H}'}{dt}=-\left[
      \sum_J^{L}\int\bar f(\vec{v})|g_J^{+}(\vec{v},t)|
        |\dot g_J^{+}(\vec{v},t)|\mathrm{d}^3v
      +\sum_J^{P}\int\bar f(\vec{v})|g_J^{-}(\vec{v},t)|
      |\dot g_J^{-}(\vec{v},t)|\mathrm{d}^3v 
    \right].
\end{equation}
%
Since every integrand in Eq.~(\ref{eq:classicalH4}) is positive,
for all $t$ and $\vec v$, and
$\left|g'_M(\vec{v},t_0)\right|\to 0$ as $t\to\infty$, it follows that
%
\begin{equation}\label{eq:dHpdtleq0}
 \frac{d\mathcal{H}'}{dt}\leq0.\qquad\qquad\textrm{QED.}
\end{equation}
%

In summary, considering a gas occupying a volume V (which is divided into $K$ small cells), with a
total energy $E$ and $N$ classical free particles, whose initial state is not in equilibrium,
but sufficiently close to equilibrium, then the functional
%
\begin{equation}\label{eq:CH3}
   \mathcal{H}'(t)=\sum_{M=1}^{K}\int f'_M(\vec{v},t) \ln f'_M(\vec{v},t)\mathrm{d}^3v
\end{equation}
%
---where $f'_M(\vec{v},t)$ is the cell distribution function---
satisfies $d\mathcal{H}'/dt\leq0$, and the equality relation is attained at $t\to\infty$.
In Eq.~(\ref{eq:CH3}), $f'_M(\vec{v},t)$ is the Maxwell-Boltzmann distribution
function, which in general is different for different cells---\textit{i.e.}, the complete system
can be non-homogeneous---and each $f'_M(\vec v,t)$ is compatible
with the cell properties, such as number of particles, $\mathcal{N}'_M$, energy, $\mathcal{E}'_M$, temperature,
$T_M$, and Legendre multipliers $\alpha_M$ and $\beta_M$. 

